\documentclass{ximera}
\graphicspath{
{./}
{volumes/}
{arclengths/}
{centroids/}
{techniques/}
{applications/}
{series/}
{powerseries/}
{odes/}
{lessons/}
}
\usepackage{booktabs}

\newcommand{\bigmath}[1]{$\displaystyle #1$}
\newcommand{\choicebreak}{}
\newenvironment{type}{}{}
\newenvironment{notes}{}{}
\newenvironment{keywords}{}{}
\newcommand{\offline}{}
\newenvironment{comments}{\begin{feedback}}{\end{feedback}}
\newenvironment{multiplechoice}{\begin{multipleChoice}}{\end{multipleChoice}}
\title{Centers of Mass and Centroids}
\author{Philip T. Gressman}

\begin{document}
\begin{abstract}
We practice setting up calculations for centers of mass and centroids.
\end{abstract}
\maketitle

\section*{(Video) Calculus: Single Variable}
\textbf{Note: It is important to get as much of an intuitive sense as you can about what the double integrals (which appear around the 3:30 mark) mean, but do not worry about precisely what they represent.}
\youtube{OLeYrY4AZXk}

\section*{Examples}

\begin{example}%[CentroidQuad46]
Compute the centroid of the region bounded by the inequalities $$0 \leq x \leq 2 \qquad \mbox{ and } \qquad {-\frac{9}{2}x^2+x} \leq y \leq {\frac{9}{2}x^2+x}.$$
\begin{itemize}
\item The term ``centroid'' refers to the geometric center of a region. Practically speaking, this means we may assume constant density (e.g., density $1$).
\item First we compute the ``mass'' of the region, which in this case is simply the area between curves:
\[ M = \int_{\answer{0}}^{\answer{2}} \left[ \left(\answer{\frac{9}{2}x^2+x}\right) - \left(\answer{-\frac{9}{2}x^2+x}\right) \right] dx  = \answer{24}. \]
\item Next we compute the moments about the $y$ and $x$ axes. This always involves multiplying the integrand above by $\tilde x$ and $\tilde y$, respectively (note the reversal), where $(\tilde x,\tilde y)$ are the coordinates of the geometric center of a typical slice.
\item Using $x$ as the slicing variable, slices are \wordChoice{\choice{horizontal}\choice[correct]{vertical}} and consequently the $x$-coordinate of the geometric center of a slice is just $x$ (but note that this would be different if $y$ were the slicing variable).  Thus
\[ M_y= \int_{\answer{0}}^{\answer{2}} x \left[ \answer{9 x^2} \right] dx = \answer{36}. \]
\item The $y$-coordinate of the geometric center of a slice will be the average of $y$-coordinates at the top and bottom of a slice. Therefore
\[ \tilde y = \answer{x}. \]
Thus
\[ M_x = \int_{\answer{0}}^{\answer{2}}  \answer{9x^3}  dx = \answer{36}. \]
\item Thus
\[ \begin{aligned}
 \overline{x} &  = \frac{M_y}{M} = \answer{\frac{3}{2}}, \\
 \overline{y} &  = \frac{M_x}{M} = \answer{\frac{3}{2}}.
 \end{aligned} \]
 \end{itemize}
\end{example}


\begin{example}%[CentroidQuad38]
Compute the centroid of the region given by ${-\frac{3}{2}y^2+2y} \leq x \leq {\frac{3}{2}y^2+2y}$ between $y = 0$ and $y = 2$.
\begin{itemize}
\item In this example, we should use $y$ as the slicing variable, so the roles of $x$ and $y$ are largely switched in comparison to the previous example.
\item First compute the mass:
\[ M = \int_{0}^{2} \left[ \left(\answer{\frac{3}{2}y^2+2y}\right) - \left(\answer{-\frac{3}{2}y^2+2y}\right) \right] dy = \answer{8}. \]
\item In this case, $\tilde y = y$ since slices are \wordChoice{\choice[correct]{horizontal}\choice{vertical}}, so
\[ M_x = \int_{0}^{2} y \left[ \answer{3 y^2} \right] dy  = \answer{12}. \]
\item Likewise, $\tilde x$ is the average of $x$-coordinates of endpoints of a slice. Thus
\[ \tilde x = \answer{ 2y}. \]
Therefore
\[ M_y =  \int_{\answer{0}}^{\answer{2}} \left[ \answer{6y^3} \right] dy = \answer{24} \]
\item To conclude,
$$ \begin{aligned}
 \overline{x} &  = \frac{M_y}{M} = \answer{3}, \\
 \overline{y} &  = \frac{M_x}{M} = \answer{\frac{3}{2}}.
\end{aligned}$$
\end{itemize}
\end{example}




\end{document}
