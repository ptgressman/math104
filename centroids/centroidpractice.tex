\documentclass{ximera}
\graphicspath{
{./}
{volumes/}
{arclengths/}
{centroids/}
{techniques/}
{applications/}
{series/}
{powerseries/}
{odes/}
}

\newcommand{\bigmath}[1]{$\displaystyle #1$}
\newcommand{\choicebreak}{}
\newenvironment{type}{}{}
\newenvironment{notes}{}{}
\newenvironment{keywords}{}{}
\newcommand{\offline}{}
\newenvironment{comments}{\begin{feedback}}{\end{feedback}}
\newenvironment{multiplechoice}{\begin{multipleChoice}}{\end{multipleChoice}}
\title{Exercises: Centers of Mass and Centroids}
\author{Philip T. Gressman}

\begin{document}
\begin{abstract}
Various questions relating to centers of mass and centroids.
\end{abstract}
\maketitle


\section*{Sample Quiz Questions}
\begin{question}[CentroidQuad01]
\begin{type}
multiplechoice
\end{type}
Compute the centroid of the region bounded by the inequalities \[-2 \leq x \leq 0 \qquad \text{ and } \qquad {3x^2-\frac{13}{2}} \leq y \leq {3x^2-\frac{11}{2}}.\]
\begin{multiplechoice}
\choice{\(\displaystyle \left(-\frac{3}{2},-3 \right)\)}
\choice{\(\displaystyle \left(-1,-3 \right)\)}
\choice{\(\displaystyle \left(-\frac{1}{2},-3 \right)\)}
\choice{\(\displaystyle \left(-\frac{3}{2},-2 \right)\)}
\choice[correct]{\(\displaystyle \left(-1,-2 \right)\)}
\choice{\(\displaystyle \left(-\frac{1}{2},-2 \right)\)}
\end{multiplechoice}
\begin{feedback}
The key calculations are as follows: 
\[ \begin{aligned}
M & = \int_{-2}^{0} \left[ \left({3x^2-\frac{11}{2}}\right) - \left({3x^2-\frac{13}{2}}\right) \right] dx = \int_{-2}^{0} \left[{1}\right] dx = 2, \\
M_y & = \int_{-2}^{0} x \left[ \left({3x^2-\frac{11}{2}}\right) - \left({3x^2-\frac{13}{2}}\right) \right] dx = \int_{-2}^{0} x \left[{1}\right] dx = -2, \\
M_x & = \frac{1}{2} \int_{-2}^{0} \left[ \left({3x^2-\frac{11}{2}}\right)^2 - \left({3x^2-\frac{13}{2}}\right)^2 \right] dx = \int_{-2}^{0} \left[{3x^2-6}\right] = -4, \\
 \overline{x} &  = \frac{M_y}{M} = -1, \\
 \overline{y} &  = \frac{M_x}{M} = -2.\end{aligned}\]
\end{feedback}
\begin{keywords}
center of mass,quadratic,polynomial,generation2
\end{keywords}
\end{question}

\section*{Sample Exam Questions}

\begin{question}
Which of the following is  the centroid of the region given by $x^2 + y^2 \leq 1$, $x \geq 0$, and $y \geq 0$? Justify your response.  {\bf Without doing the calculation,} is $\overline{x}$ greater than or less than $\frac{1}{2}$? Give a brief geometric explanation.
\begin{multipleChoice}
\choice[correct]{\bigmath{\left(\frac{4}{3 \pi},\frac{4}{3 \pi}\right)}}
\choice{\bigmath{\left(\frac{1}{2},\frac{4}{3 \pi}\right)}}
\choice{\bigmath{\left(\frac{4}{3 \pi}, \frac{1}{2}\right)}}
\choice{\bigmath{\left(\frac{4}{3 \pi} ,\frac{5}{3 \pi}\right)}}
\choice{\bigmath{\left(\frac{1}{2},\frac{1}{2}\right)}}
\choice{\bigmath{\left(\frac{5}{3 \pi},\frac{4}{3 \pi}\right)}}
\end{multipleChoice}

 \begin{feedback}\[ M = \frac{\pi}{4} \]
\[ \int_0^1 x \sqrt{1-x^2} dx = \left. - \frac{1}{3} (1-x^2)^\frac{3}{2} \right|_{0}^1 \]
The value of $\overline{x}$ should be less than $\frac{1}{2}$ since, when compared symmetrically about the axis $x=\frac{1}{2}$, the quarter circle has longer slices to the left of the axis than it does to the right.
\end{feedback}
\end{question}

\begin{question}[2015C.12]
\begin{type}
multiplechoice
\end{type}
Find the \(y\)-coordinate of the centroid of the region bounded by the \(x\)-axis, the \(y\)-axis, and the graph of \(y = \cos x\) for \(0 \leq x \leq \pi/2\) if the density is constant.
\begin{multiplechoice}
\choice{\(\displaystyle \frac{\pi}{18}\)}
\choice{\(\displaystyle \frac{\pi}{12}\)}
\choice[correct]{\(\displaystyle \frac{\pi}{8}\)}
\choice{\(\displaystyle \frac{\pi}{6}\)}
\choice{\(\displaystyle \frac{\pi}{4}\)}
\choice{\(\displaystyle \frac{\pi}{2}\)}
\end{multiplechoice}
\begin{feedback}
The area of the region is given by
\[ M = \int_0^{\frac{\pi}{2}} \cos x~dx = 1 \]
and 
\[\begin{aligned}
 M_x & = \int_0^{\frac{\pi}{2}} \frac{0 + \cos x}{2} \cos x~ dx = \frac{1}{2} \int_0^{\frac{\pi}{2}} \cos^2 x ~ dx \\
& \frac{1}{2} \int_0^{\frac{\pi}{2}} \frac{1 + \cos 2x}{2} dx = \frac{\pi}{8}. 
\end{aligned}\]
Therefore \(\overline{y} = M_x / M = \pi/8\).
\end{feedback}
\begin{keywords}
center of mass,trigonometric integrals
\end{keywords}
\end{question}

\begin{question}[2016C.02]
\begin{type}
multiplechoice
\end{type}
Find the \(y\)-coordinate of the centroid of the region in the upper half-plane (i.e., for \(y > 0\)) bounded by the semicircle \(y = \sqrt{1-x^2}\). (It is easiest to use a geometric formula to find the area of the region.)
\begin{multiplechoice}
\choice{\(\displaystyle \frac{4 \pi}{3}\)}
\choice[correct]{\(\displaystyle \frac{4}{3 \pi}\)}
\choice{\(\displaystyle \frac{7 \pi}{3}\)}
\choice{\(\displaystyle \frac{7}{3 \pi}\)}
\choice{\(\displaystyle \frac{28 \pi}{9}\)}
\choice{\(\displaystyle \frac{28}{9 \pi}\)}
\end{multiplechoice}
\begin{keywords}
center of mass
\end{keywords}
\end{question}

\end{document}
