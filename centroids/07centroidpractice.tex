\documentclass{ximera}
\graphicspath{
{./}
{volumes/}
{arclengths/}
{centroids/}
{techniques/}
{applications/}
{series/}
{powerseries/}
{odes/}
{lessons/}
}
\usepackage{booktabs}

\newcommand{\bigmath}[1]{$\displaystyle #1$}
\newcommand{\choicebreak}{}
\newenvironment{type}{}{}
\newenvironment{notes}{}{}
\newenvironment{keywords}{}{}
\newcommand{\offline}{}
\newenvironment{comments}{\begin{feedback}}{\end{feedback}}
\newenvironment{multiplechoice}{\begin{multipleChoice}}{\end{multipleChoice}}
\title{Exercises: Centers of Mass and Centroids}
\author{Philip T. Gressman}

\begin{document}
\begin{abstract}
Various questions relating to centers of mass and centroids.
\end{abstract}
\maketitle

%\begin{exercise}%%Community Calculus Section 09.06
%A thin plate fills the upper half of the unit circle $x^2 + y^2 = 1$. Find the centroid.
%\[ \overline{x} = \answer{0} \text{ and } \overline{y} = \answer{\frac{4}{3 \pi}}. \]
%\end{exercise}

\begin{exercise}%%Community Calculus 09.06
Find the centroid of the region bounded above by $y = x$ and below by $y = x^2$.
\[ \overline{x} = \answer{\frac{1}{2}} \text{ and } \overline{y} = \answer{\frac{2}{5}}. \]
\end{exercise}

\begin{exercise}%%Community Calculus 09.06
Find the centroid of the region bounded above by $y = 4-x^2$ and below by $y = 0$.
\[ \overline{x} = \answer{0} \text{ and } \overline{y} = \answer{\frac{8}{5}}. \]
\end{exercise}

\begin{exercise}
A thin plate in the plane defined by $x^2 \leq y \leq 1$ and $x \geq 0$ has density $y$ at the point $(x,y)$. Compute the center of mass.
\begin{hint}
Use $y$ as the variable of slicing. The center of mass of a single slice $(\tilde x, \tilde y)$ is then $(\sqrt{y}/2,y)$.
\end{hint}
\[ \overline{x} = \answer{\frac{1}{2}} \text{ and } \overline{y} = \answer{\frac{5}{7}}. \]
\end{exercise}

\begin{exercise}
A thin plate in the plane defined by $x^2 \leq y \leq 2x^2$ and $0 \leq x \leq 1$ has density $x$ at the point $(x,y)$. Compute the center of mass.
\begin{hint}
Use $x$ as the variable of slicing. The center of mass of a single slice $(\tilde x, \tilde y)$ is then $(x,3x^2/2)$.
\end{hint}
\[ \overline{x} = \answer{\frac{4}{5}} \text{ and } \overline{y} = \answer{1}. \]
\end{exercise}

\begin{exercise}
The same thin plate as above ($x^2 \leq y \leq 2x^2$ and $0 \leq x \leq 1$) now has density $x^{-2}$ at the point $(x,y)$.  Because the density of the plate is now higher near the origin than in the previous problem, this suggests that the center of mass will shift \wordChoice{\choice{away from}\choice[correct]{towards}} the origin relative to the previous exercise.

Compute the center of mass.
\[ \overline{x} = \answer{\frac{1}{2}} \text{ and } \overline{y} = \answer{\frac{1}{2}}. \]
\end{exercise}

\begin{exercise}
Compute the centroid of a thin wire along the graph $y = \sqrt{1-x^2}$ between $x = 0$ and $x=1$. 
\begin{hint}
Recall that 
\[ M = \int ds \]
\[ \overline{x} = \frac{1}{M} \int x ds \]
\[ \overline{y} = \frac{1}{M} \int y ds \]
where $ds$ is the arc length element. We also know that
\[ \int \frac{dx}{\sqrt{1-x^2}} = \arcsin x + C. \]
\end{hint}
\[ \overline{x} = \answer{\frac{2}{\pi}} \text{ and } \overline{y} = \answer{\frac{2}{\pi}}. \]
\end{exercise}


\section*{Sample Quiz Questions}
\begin{question}%%%%%[CentroidQuad01]

Compute the centroid of the region bounded by the inequalities \[-2 \leq x \leq 0 \qquad \text{ and } \qquad {3x^2-\frac{13}{2}} \leq y \leq {3x^2-\frac{11}{2}}.\]
(Hints won't be revealed until after you choose a response.)
\begin{multiplechoice}
\choice{\(\displaystyle \left(-\frac{3}{2},-3 \right)\)}
\choice{\(\displaystyle \left(-1,-3 \right)\)}
\choice{\(\displaystyle \left(-\frac{1}{2},-3 \right)\)}
\choice{\(\displaystyle \left(-\frac{3}{2},-2 \right)\)}
\choice[correct]{\(\displaystyle \left(-1,-2 \right)\)}
\choice{\(\displaystyle \left(-\frac{1}{2},-2 \right)\)}
\end{multiplechoice}
\begin{feedback}
The key calculations are as follows: 
\[ \begin{aligned}
M & = \int_{-2}^{0} \left[ \left({3x^2-\frac{11}{2}}\right) - \left({3x^2-\frac{13}{2}}\right) \right] dx , \\
M_y & = \int_{-2}^{0} x \left[ \left({3x^2-\frac{11}{2}}\right) - \left({3x^2-\frac{13}{2}}\right) \right] dx, \\
M_x & = \frac{1}{2} \int_{-2}^{0} \left[ \left({3x^2-\frac{11}{2}}\right)^2 - \left({3x^2-\frac{13}{2}}\right)^2 \right] dx.
\end{aligned}\]
 \begin{hint}
 \[ \begin{aligned}
M & = \int_{-2}^{0} \left[ \left({3x^2-\frac{11}{2}}\right) - \left({3x^2-\frac{13}{2}}\right) \right] dx = \int_{-2}^{0} \left[{1}\right] dx = 2, \\
M_y & = \int_{-2}^{0} x \left[ \left({3x^2-\frac{11}{2}}\right) - \left({3x^2-\frac{13}{2}}\right) \right] dx = \int_{-2}^{0} x \left[{1}\right] dx = -2, \\
M_x & = \frac{1}{2} \int_{-2}^{0} \left[ \left({3x^2-\frac{11}{2}}\right)^2 - \left({3x^2-\frac{13}{2}}\right)^2 \right] dx = \int_{-2}^{0} \left[{3x^2-6}\right] = -4, \\
 \overline{x} &  = \frac{M_y}{M} = -1, \\
 \overline{y} &  = \frac{M_x}{M} = -2.\end{aligned}\]
 \end{hint}
\end{feedback}

\end{question}

\section*{Sample Exam Questions}




\begin{question}%%%%%[2015C.12]

Find the \(y\)-coordinate of the centroid of the region bounded by the \(x\)-axis, the \(y\)-axis, and the graph of \(y = \cos x\) for \(0 \leq x \leq \pi/2\) if the density is constant.
\begin{hint}
Use the identity \[ \cos^2 x  = \frac{1 + \cos 2x}{2} \]
to calculate the integral of $\cos^2 x$.
\end{hint}
\begin{multiplechoice}
\choice{\(\displaystyle \frac{\pi}{18}\)}
\choice{\(\displaystyle \frac{\pi}{12}\)}
\choice[correct]{\(\displaystyle \frac{\pi}{8}\)}
\choice{\(\displaystyle \frac{\pi}{6}\)}
\choice{\(\displaystyle \frac{\pi}{4}\)}
\choice{\(\displaystyle \frac{\pi}{2}\)}
\end{multiplechoice}
\begin{feedback}
The area of the region is given by
\[ M = \int_0^{\frac{\pi}{2}} \cos x~dx = 1 \]
and 
\[\begin{aligned}
 M_x & = \int_0^{\frac{\pi}{2}} \frac{0 + \cos x}{2} \cos x~ dx = \frac{1}{2} \int_0^{\frac{\pi}{2}} \cos^2 x ~ dx \\
& \frac{1}{2} \int_0^{\frac{\pi}{2}} \frac{1 + \cos 2x}{2} dx = \frac{\pi}{8}. 
\end{aligned}\]
Therefore \(\overline{y} = M_x / M = \pi/8\).
\end{feedback}

\end{question}

\begin{question}%%%%%[2016C.02]

Find the \(y\)-coordinate of the centroid of the region in the upper half-plane (i.e., for \(y > 0\)) bounded by the semicircle \(y = \sqrt{1-x^2}\). (It is easiest to use a geometric formula to find the area of the region.)
\begin{multiplechoice}
\choice{\(\displaystyle \frac{4 \pi}{3}\)}
\choice[correct]{\(\displaystyle \frac{4}{3 \pi}\)}
\choice{\(\displaystyle \frac{7 \pi}{3}\)}
\choice{\(\displaystyle \frac{7}{3 \pi}\)}
\choice{\(\displaystyle \frac{28 \pi}{9}\)}
\choice{\(\displaystyle \frac{28}{9 \pi}\)}
\end{multiplechoice}

\end{question}

\end{document}
