\documentclass{ximera}
\graphicspath{
{./}
{volumes/}
{arclengths/}
{centroids/}
{techniques/}
{applications/}
{series/}
{powerseries/}
{odes/}
{lessons/}
}
\usepackage{booktabs}

\newcommand{\bigmath}[1]{$\displaystyle #1$}
\newcommand{\choicebreak}{}
\newenvironment{type}{}{}
\newenvironment{notes}{}{}
\newenvironment{keywords}{}{}
\newcommand{\offline}{}
\newenvironment{comments}{\begin{feedback}}{\end{feedback}}
\newenvironment{multiplechoice}{\begin{multipleChoice}}{\end{multipleChoice}}
\title{Applications of ODEs}
\author{Philip T. Gressman}

\begin{document}
\begin{abstract}
We study some sample applications of ODEs.
\end{abstract}
\maketitle

\begin{example}
The base of a solid region is bounded by the curves $x = 0$, $y= 0$, and $y = \sqrt{1-x^2}$. The cross sections perpendicular to the $x$ axis are squares. Compute the volume of the region.
\begin{solution}
Lines in the $xy$-plane which are perpendicular to the $x$-axis are vertical, so the base of a typical $x$ cross section will extend from $y=0$ to $y = \sqrt{1-x^2}$. Since each cross section will have area
\[ A(x) = \left( \sqrt{1-x^2} - 0 \right)^2 = 1-x^2. \]
To compute volume, we integrate $dV = A(x) dx$ between $x=0$ and $x=1$, since these are the most extreme values of $x$ found in our region. Therefore
\[ V = \int_0^1 ( 1 -x^2) dx = \left. x - \frac{x^3}{3} \right|_0^1 = 1 - \frac{1}{3} = \frac{2}{3}. \]
\end{solution}
\end{example}

\begin{example}
The base of a solid region is bounded by the curves $y = 0$, $x = \sqrt{y}$, and $x = 1$. The cross sections perpendicular to the $y$-axis are squares. Compute the volume of the region.
\begin{solution}
Lines in the $xy$-plane which are perpendicular to the $y$-axis are \wordChoice{\choice[correct]{horizontal}\choice{vertical}}, so the base of a typical $y$ cross section will extend from the graph $x=\answer{\sqrt{y}}$ to the graph $x = \answer{1}$. The length of the base is the difference of $x$-coordinates (since all points on a slice have the same $y$-coordinate), so the length of the base is $\answer{1 - \sqrt{y}}$, giving the square an area of 
\[ A(y) = \answer{(1-\sqrt{y})^2} \]
(note that the answer is a function of $y$ because different $y$ cross sections will generally have different areas).
To compute volume, we integrate $dV = A(y) dy$ between $y=0$ and $y=1$, since these are the most extreme values of $y$ found in our region (note that we can find the upper value $y=1$ by solving for the intersection of the curves $x = \sqrt{y}$ and $x=1$). Therefore we integrate $A(y) dy$ to conclude
\[ V = \int_{\answer{0}}^{\answer{1}} \answer{(1-\sqrt{y})^2} dy = \answer{\frac{1}{6}}. \]
\end{solution}
\end{example}


\end{document}
