\documentclass{ximera}
\graphicspath{
{./}
{volumes/}
{arclengths/}
{centroids/}
{techniques/}
{applications/}
{series/}
{powerseries/}
{odes/}
{lessons/}
}
\usepackage{booktabs}

\newcommand{\bigmath}[1]{$\displaystyle #1$}
\newcommand{\choicebreak}{}
\newenvironment{type}{}{}
\newenvironment{notes}{}{}
\newenvironment{keywords}{}{}
\newcommand{\offline}{}
\newenvironment{comments}{\begin{feedback}}{\end{feedback}}
\newenvironment{multiplechoice}{\begin{multipleChoice}}{\end{multipleChoice}}
\title{Exercises: Linear and Separable ODEs}
\author{Philip T. Gressman}

\begin{document}
\begin{abstract}
Exercises related to solving linear and separable ODEs.
\end{abstract}
\maketitle

\section*{Sample Quiz Questions}

\begin{question}%%%%%[SepIVP001]

Let \(y(x)\) be the solution to the initial value problem \[  \frac{dy}{dx}  = -(1 + 3 x^2)y^2\] and \(y(0) = 1/2\).  What is the value of \(y(1)\)?
\begin{multiplechoice}
\choice{\( \displaystyle \frac{1}{6}\)}
\choice[correct]{\( \displaystyle \frac{1}{4}\)}
\choice{\( \displaystyle \frac{1}{3}\)}
\choice{\( \displaystyle \frac{1}{2}\)}
\choice{\( \displaystyle \frac{\pi}{4}\)}
\choice{\( \displaystyle 1\)}
\end{multiplechoice}
\begin{feedback}
This is a separable ODE. Moving all functions of \(y\) to the left-hand side and all functions of \(x\) to the right-hand side and integrating gives \[\int  \frac{ -1  }{ y^2} ~ dy = \int (1 + 3 x^2) ~ dx, \] which yields \[\frac{1}{y} = x^3 + x + C.\] Evaluating at \(x = 0\) and \(y = 1/2\) gives \(2 = 0 + C\), so \[\frac{1}{y} = x^3 + x + 2,\] i.e., \[y = \frac{1}{x^3 + x + 2}.\] Plugging in \(x = 1\) gives \(y = 1/4\).
\end{feedback}

\end{question}

\section*{Sample Exam Questions}

\begin{question}%%%%%[2015C.06]

The solution of the initial value problem \(\displaystyle x \frac{dy}{dx} + 3y = 7 x^4\),  \(y(1) = 1\),
satisfies \(y(2) = \)
\begin{multiplechoice}
\choice{\(0\)}
\choice{\(1\)}
\choice{\(2\)}
\choice{\(4\)}
\choice{\(8\)}
\choice[correct]{\(16\)}
\end{multiplechoice}

\end{question}

\begin{question}%%%%%[2015C.07]

The solution of the initial value problem \(\displaystyle \frac{dy}{dx} - 20 x^4 e^{-y} = 0\),  \(y(0) = 0\),
satisfies \(y(1) = \)
\begin{multiplechoice}
\choice[correct]{\(\ln 5\)}
\choice{\(\ln 4\)}
\choice{\(\ln 3\)}
\choice{\(\ln 2\)}
\choice{\(1\)}
\choice{\(0\)}
\end{multiplechoice}

\end{question}

\begin{question}%%%%%[2016C.13]

Let \(y(x)\) be the solution of the initial value problem
\[ x \frac{dy}{dx} = e^x - y \ \text{ with } \ y(\ln 2) = 0. \]
Find \(y(1)\).
\begin{multiplechoice}
\choice{\(\displaystyle \frac{e^2}{2}\)}
\choice{\(\displaystyle 2e^2\)}
\choice{\(\displaystyle \frac{e}{2}\)}
\choice{\(0\)}
\choice[correct]{\(e-2\)}
\choice{\(1\)}
\end{multiplechoice}

\end{question}

\begin{question}%%%%%[2017C.07]

Let \(y(x)\) be the solution of the initial value problem
\[ x \frac{dy}{dx} = y + x^2 \sin x \ \text{ with } \ y (\pi) = 0. \]
What is \(y(2 \pi)\)?
\begin{multiplechoice}
\choice{\(-\pi\)}
\choice{\(-2\pi\)}
\choice[correct]{\(-4\pi\)}
\choice{\(0\)}
\choice{\(2\pi\)}
\choice{\(4\pi\)}
\end{multiplechoice}

\end{question}

\begin{question}%%%%%[2017C.08]

Consider the initial value problem
\[ (1 + x^2) \frac{dy}{dx} = 2y \ \text{ with } \ y(0) = 2. \]
What is \(\displaystyle \lim_{x \rightarrow \infty} y(x)\)?
\begin{multiplechoice}
\choice[correct]{\(2e^{\pi}\)}
\choice{\(2e^{\pi/2}\)}
\choice{\(2e^{\pi/4}\)}
\choice{\(1\)}
\choice{\(0\)}
\choice{\(e^{\pi}\)}
\end{multiplechoice}

\end{question}


\end{document}
