\documentclass{ximera}
\graphicspath{
{./}
{volumes/}
{arclengths/}
{centroids/}
{techniques/}
{applications/}
{series/}
{powerseries/}
{odes/}
{lessons/}
}
\usepackage{booktabs}

\newcommand{\bigmath}[1]{$\displaystyle #1$}
\newcommand{\choicebreak}{}
\newenvironment{type}{}{}
\newenvironment{notes}{}{}
\newenvironment{keywords}{}{}
\newcommand{\offline}{}
\newenvironment{comments}{\begin{feedback}}{\end{feedback}}
\newenvironment{multiplechoice}{\begin{multipleChoice}}{\end{multipleChoice}}
\title{Exercises: ODEs}
\author{Philip T. Gressman}

\begin{document}
\begin{abstract}
Exercises relating to fundamental properties of ODEs.
\end{abstract}
\maketitle

\begin{exercise}
The base of a solid region is bounded by the curves $x = 0$, $y = x^2$ and $y = x$. The cross sections perpendicular to the $x$-axis are squares. Compute the volume of the region.
\begin{itemize}
\item A typical square cross section has side length $L = \answer{x - x^2}$ and area $A = \answer{(x-x^2)^2}$. 
\item Possible numerical values of the $x$-coordinates of points in the base range from a minimum value of $x = \answer{0}$ up to a maximum of $x = \answer{1}$.
\item To compute volume, integrate:
\[ V = \int_{\answer{0}}^{\answer{1}} \answer{(x-x^2)^2} d\answer{x} = \answer{\frac{1}{30}}. \]
\end{itemize}
\end{exercise}

\begin{exercise}
Find the volume of the region in three-dimensional space defined by the inequalities
\begin{align*}
0 & \leq x \leq 1, \\
0 & \leq y \leq z^2, \\
0 & \leq z \leq 3.
\end{align*}
\begin{itemize}
\item Cross sections perpendicular to the $z$-axis are \wordChoice{\choice{square}\choice[correct]{rectangular}\choice{triangluar}} with length $\answer{1}$ in the $x$-direction and width $\answer{z^2}$ in the $y$-direction.
\item The area of a $z$ cross section is $A(z) = \answer{z^2}$.
\item To compute volume, integrate:
\[ V = \int_{\answer{0}}^{\answer{3}} \answer{z^2} dz = \answer{9}. \]
\end{itemize}
\end{exercise}


\end{document}
