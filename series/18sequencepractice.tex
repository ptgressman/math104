\documentclass{ximera}
\graphicspath{
{./}
{volumes/}
{arclengths/}
{centroids/}
{techniques/}
{applications/}
{series/}
{powerseries/}
{odes/}
{lessons/}
}
\usepackage{booktabs}

\newcommand{\bigmath}[1]{$\displaystyle #1$}
\newcommand{\choicebreak}{}
\newenvironment{type}{}{}
\newenvironment{notes}{}{}
\newenvironment{keywords}{}{}
\newcommand{\offline}{}
\newenvironment{comments}{\begin{feedback}}{\end{feedback}}
\newenvironment{multiplechoice}{\begin{multipleChoice}}{\end{multipleChoice}}
\title{Exercises: Sequences}
\author{Philip T. Gressman}

\begin{document}
\begin{abstract}
Exercises relating to sequences.
\end{abstract}
\maketitle

\begin{exercise}
The sequence $a_n = 1/n^3$ has limit $L = 0$. Suppose $\epsilon = 1/64$; find a threshold $N$ such that
\[ |a_n -L| < \epsilon \]
is guaranteed to hold for all $n > N$ (take your value of $N$ as small as possible).
\[ N = \answer{4}. \]
\end{exercise}

\begin{exercise}
The sequence $a_n = (2n^2 + (-1)^n)/n^2$ has limit $L = 2$. Suppose $\epsilon = 1/100$; find a threshold $N$ such that
\[ |a_n -L| < \epsilon \]
is guaranteed to hold for all $n > N$ (take your value of $N$ as small as possible).
\[ N = \answer{10}. \]
\end{exercise}

\begin{exercise}%[APEX0801SEQS09]
Determine the \(n^\text{th}\) term of the given sequence. $a_1 = 4$, $a_2 = 7$, $a_3 = 10$, $a_4 = 13$, $a_5 = 16$, \(\ldots\)
\[ a_n = \answer{3n+1}. \]
\end{exercise}
\begin{exercise}%[APEX0801SEQS09]
Determine the \(n^\text{th}\) term of the given sequence. $a_1 = 3$, $a_2 = -5/2$, $a_3 = 7/4$, $a_4 = -9/8$, $a_5 = 11/16$, \(\ldots\)
\[ a_n = \answer{\frac{(-1)^{n-1} (2n+1)}{2^{n-1}}}. \]
\end{exercise}

\begin{exercise}%[APEX0801SEQS17]
Determine whether the sequence converges or diverges. If convergent, give the limit of the sequence. 
\[ a_n = \left\{(-1)^n\frac{n}{n+1}\right\} \]
\begin{multipleChoice}
\choice{converges to $\answer{u}$}
\choice[correct]{diverges}
\end{multipleChoice}
\end{exercise}

\begin{exercise}%[APEX0801SEQS18]
Determine whether the sequence converges or diverges. If convergent, give the limit of the sequence. 
\[ a_n = \frac{4n^2-n+5}{3n^2+1} \]
\begin{multipleChoice}
\choice[correct]{converges to $\answer{\frac{4}{3}}$}
\choice{diverges}
\end{multipleChoice}
\end{exercise}

\begin{exercise}%[APEX0801SEQS19]
Determine whether the sequence converges or diverges. If convergent, give the limit of the sequence. 
\[ a_n = \frac{4^n}{5^n} \]
\begin{multipleChoice}
\choice[correct]{converges to $\answer{0}$}
\choice{diverges}
\end{multipleChoice}
\end{exercise}

\begin{exercise}
Determine whether the sequence converges or diverges. If convergent, give the limit of the sequence. 
\[ a_n = \left(1 - \frac{3}{n} \right)^{-n} \]
\begin{multipleChoice}
\choice[correct]{converges to $\answer{e^{3}}$}
\choice{diverges}
\end{multipleChoice}
\begin{hint}
Take the reciprocal and compare to your reference list of commonly-occurring limits.
\end{hint}
\end{exercise}

\begin{exercise}
Determine whether the sequence converges or diverges. If convergent, give the limit of the sequence. 
\[ a_n = \left(1 - \frac{3}{n}\right)^{-n^2} \]
\begin{multipleChoice}
\choice{converges to $\answer{u}$}
\choice[correct]{diverges}
\end{multipleChoice}
\begin{hint}
If a sequence $b_n$ tends to $e^3$, what will the sequence $(b_n)^n$ do?
\end{hint}
\end{exercise}

\begin{exercise}
Determine whether the sequence converges or diverges. If convergent, give the limit of the sequence. 
\[ a_n = \frac{(1.1)^n}{n} \]
\begin{multipleChoice}
\choice{converges to $\answer{u}$}
\choice[correct]{diverges}
\end{multipleChoice}
\begin{hint}
What are the relative orders of growth of numerator versus denominator?
\end{hint}
\end{exercise}

\begin{exercise}
Determine whether the sequence converges or diverges. If convergent, give the limit of the sequence. 
\[ a_n = \frac{(0.9)^n}{n} \]
\begin{multipleChoice}
\choice[correct]{converges to $\answer{0}$}
\choice{diverges}
\end{multipleChoice}
\end{exercise}

\begin{exercise}
Determine whether the sequence converges or diverges. If convergent, give the limit of the sequence. 
\[ a_n = n^{1000000} (0.9)^n \]
\begin{multipleChoice}
\choice[correct]{converges to $\answer{0}$}
\choice{diverges}
\end{multipleChoice}
\end{exercise}

\begin{exercise}
Determine whether the sequence converges or diverges. If convergent, give the limit of the sequence. 
\[ a_n = \frac{\ln n}{n} \]
\begin{multipleChoice}
\choice[correct]{converges to $\answer{0}$}
\choice{diverges}
\end{multipleChoice}
\end{exercise}

\begin{exercise}
Determine whether the sequence converges or diverges. If convergent, give the limit of the sequence. 
\[ a_n = \frac{\ln n}{n^{0.00001}} \]
\begin{multipleChoice}
\choice[correct]{converges to $\answer{0}$}
\choice{diverges}
\end{multipleChoice}
\end{exercise}


\begin{exercise}
Let $b_n$ be the sequence given by
\[ b_1 = 0 \ \text{ and } \ b_{n+1} = \frac{2 + b_n}{3} \ \text{ for } n \geq 1 \]
converges. Compute its limit.
\[ \lim_{n \rightarrow \infty} b_n = \answer{1}. \]
\end{exercise}

\begin{exercise}
Determine whether the sequence converges or diverges. If convergent, give the limit of the sequence. 
\[ a_1 = 1 \ \text{ and } \ a_{n+1} = a_n + \frac{1}{a_n} \ \text{ for } n \geq 1 \]
\begin{multipleChoice}
\choice{converges to $\answer{u}$}
\choice[correct]{diverges}
\end{multipleChoice}
\end{exercise}

\begin{exercise}
Determine whether the sequence converges or diverges. If convergent, give the limit of the sequence. 
\[ a_1 = 1 \ \text{ and } \ a_{n+1} = a_n + \frac{1}{4} (4 - a_n^2) \ \text{ for } n \geq 1 \]
\begin{multipleChoice}
\choice[correct]{converges to $\answer{2}$}
\choice{diverges}
\end{multipleChoice}
\end{exercise}

\begin{exercise}
Let $a_n$ be the sequence given by
\[ a_1 = \frac{1}{4}, \ \text{ and }  \ a_{n+1} = 2 a_n(1-a_n) \ \text{ for } n \geq 1 \]
converges. Compute its limit.
\[ \lim_{n \rightarrow \infty} a_n = \answer{\frac{1}{2}}. \]
\begin{hint}
If $a_n$ happens to be positive and less than $1/2$, then $2 (1-a_n) > 1$, so this forces $2 a_n(1-a_n) > a_n$ (meaning that the term after $a_n$ will be larger than $a_n$.
\begin{hint}
The function $2 x(1-x)$ is nonnegative on the interval $[0,1]$ and has a maximum value of $1/2$ attained at $x = 1/2$. This means that if $a_n$ is anything between $0$ and $1$, the next term of the sequence will always be between $0$ and $1/2$.
\end{hint}
\end{hint}
\end{exercise}

\section*{Sample Quiz Questions}

\begin{question}%%%%%[SeqConvSQRT01]

Find the limit of the sequence
\[\lim_{n \rightarrow \infty} \sqrt{\frac{2n-2}{2n^2-4n+3}}.\] \offline{Justify your response.}
(Hints will not be revealed until after you choose a response.)
\begin{multiplechoice}
\choice[correct]{\(\displaystyle 0\)}
\choice{\(\displaystyle \frac{1}{3}\)}
\choice{\(\displaystyle \frac{1}{2}\)}
\choice{\(\displaystyle 1\)}
\choice{\(\displaystyle 2\)}
\choice{\(\displaystyle 3\)}
\end{multiplechoice}
\begin{feedback}
Because the square root function is continuous, you can pass the limit through it and compute
\[ \sqrt{\lim_{n \rightarrow \infty} \frac{2n-2}{2n^2-4n+3}}. \]
\begin{hint}
Reduce numerator and denominator to the dominant terms (in the regime $n \rightarrow \infty$).
\begin{hint}
\[\begin{aligned}
\lim_{n \rightarrow \infty} \sqrt{\frac{2n-2}{2n^2-4n+3}} & = \lim_{n \rightarrow \infty} \sqrt{\frac{2n^{-1}-2n^{-2}}{2-4n^{-1}+3n^{-2}}} \\
 & = \sqrt{\frac{\lim_{n \rightarrow \infty} 2n^{-1}-2n^{-2}}{\lim_{n \rightarrow \infty} 2-4n^{-1}+3n^{-2}}} \\
 & = \sqrt{\frac{0}{2}} = 0. 
\end{aligned}\]
\end{hint}
\end{hint}
\end{feedback}

\end{question}

\begin{question}%%%%%[SeqConvORDER19]

Determine whether the limit below exists. If it exists, find its value.
\[\lim_{n \rightarrow \infty} \frac{(-1)^{n+1}n^2 - 2^{-n-1}}{(-1)^{n}n^2 + 4^{-n}}.\] \offline{Justify your response.}
(Hints will not be revealed until after you choose a response.)
\begin{multiplechoice}
\choice[correct]{\(\displaystyle -1\)}
\choice{\(\displaystyle 0\)}
\choice{\(\displaystyle \frac{1}{2}\)}
\choice{\(\displaystyle 2\)}
\choice{\(\displaystyle 3\)}
\choice{limit does not exist}
\end{multiplechoice}
\begin{feedback}
Comparing the orders of growth of the terms in the numerator, the first term dominates because \(|-1|>|1/2|\). 
\begin{hint} Likewise the first term dominates in the denominator because  \(|-1|>|1/4|\). \begin{hint} Neglecting non-dominant terms leads to the limit \[\lim_{n \rightarrow \infty} \frac{(-1)^{n+1}n^2}{(-1)^{n}n^2}\] which simply equals \(-1\). \end{hint} \end{hint}
\end{feedback}

\end{question}

\begin{question}%%%%%[SeqConvEXPN43]

Determine whether the limit below exists. If it exists, find its value.
\[\lim_{n \rightarrow \infty} \left(\frac{4n - 3}{4n + 1}\right)^{n}.\] \offline{Justify your response.}
(Hints will not be revealed until after you choose a response.)
\begin{multiplechoice}
\choice{\(\displaystyle 0\)}
\choice{\(\displaystyle 1\)}
\choice[correct]{\(\displaystyle e^{-1}\)}
\choice{\(\displaystyle e\)}
\choice{\(\displaystyle e^2\)}
\choice{limit does not exist}
\end{multiplechoice}
\begin{feedback}
First observe that 
\[\frac{4n - 3}{4n + 1} = 1 - \frac{4}{4n + 1} \rightarrow 1\]
as \(n \rightarrow \infty\). \begin{hint} Next, in light of the known limit \((1+x/k)^{k} \rightarrow e^x\) as \(k \rightarrow \infty\), manipulate exponents to see that  \[\left(1 - \frac{4}{4n + 1}\right)^n = \left( \left(1 - \frac{4}{4n + 1}\right)^{4n + 1} \right)^{ 1/4} \left(1 - \frac{4}{4n + 1}\right)^{-1/4}. \] \begin{hint} As \(n \rightarrow \infty\), the first term on the right-hand side tends to \(e^{-1}\) and the second term tends to \(1\). Thus the original sequence tends to \(e^{-1}\) as well. \end{hint} \end{hint}
\end{feedback}

\end{question}

\begin{question}%%%%%[SeqConvEXPN39]

Determine whether the limit below exists. If it exists, find its value.
\[\lim_{n \rightarrow \infty} \left(\frac{n + 3}{2n - 2}\right)^{n^2}.\] \offline{Justify your response.}
(Hints will not be revealed until after you choose a response.)
\begin{multiplechoice}
\choice[correct]{\(\displaystyle 0\)}
\choice{\(\displaystyle 1\)}
\choice{\(\displaystyle e^{-1}\)}
\choice{\(\displaystyle e\)}
\choice{\(\displaystyle e^2\)}
\choice{limit does not exist}
\end{multiplechoice}
\begin{feedback}
First observe that 
\[\frac{n + 3}{2n - 2} = \frac{1}{2} + \frac{2}{n - 1} \rightarrow \frac{1}{2}\]
as \(n \rightarrow \infty\). \begin{hint} Since the limit is positive and less than one, raising this expression to increasingly large powers generates a sequence which converges rapidly to zero. \end{hint}
\end{feedback}

\end{question}

\section*{Sample Exam Questions}

\begin{question}%%%%%[2016C.08]

Determine whether the sequence \(\displaystyle a_n = (-1)^{n-1} \frac{n^2}{1 + n^2 + n^3}\) converges or diverges. If it converges, find its limit.
\begin{multiplechoice}
\choice{divergent, \(\displaystyle \lim_{n \rightarrow \infty} a_n = 0\)}
\choice{convergent, \(\displaystyle \lim_{n \rightarrow \infty} a_n = 1\)}
\choice[correct]{convergent, \(\displaystyle \lim_{n \rightarrow \infty} a_n = 0\)}
\choice{convergent, \(\displaystyle \lim_{n \rightarrow \infty} a_n = -1\)}
\choice{divergent, \(\displaystyle \lim_{n \rightarrow \infty} a_n = \infty\)}
\choice{divergent, limit doesn't exist}
\end{multiplechoice}

\end{question}


\end{document}
