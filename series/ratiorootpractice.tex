\documentclass{ximera}
\graphicspath{
{./}
{volumes/}
{arclengths/}
{centroids/}
{techniques/}
{applications/}
{series/}
{powerseries/}
{odes/}
{lessons/}
}
\usepackage{booktabs}

\newcommand{\bigmath}[1]{$\displaystyle #1$}
\newcommand{\choicebreak}{}
\newenvironment{type}{}{}
\newenvironment{notes}{}{}
\newenvironment{keywords}{}{}
\newcommand{\offline}{}
\newenvironment{comments}{\begin{feedback}}{\end{feedback}}
\newenvironment{multiplechoice}{\begin{multipleChoice}}{\end{multipleChoice}}
\title{Exercises: Ratio and Root Tests}
\author{Philip T. Gressman}

\begin{document}
\begin{abstract}
Exercises relating to the ratio and root tests.
\end{abstract}
\maketitle


\section*{Sample Quiz Questions}

\begin{question}[2019RatioInconclusive1]
\begin{type}
multiplechoice
\end{type}
Determine which of the following three infinite series will lead to inconclusive results for the ratio test and then determine whether that series is convergent or divergent. 
\[ 
	\text{I: }   \sum_{k=1}^\infty \frac{1}{k - e^{-k}} \ \ \
	\text{II: }  \sum_{m=1}^\infty \frac{1}{m^2 - e^{m}} \ \ \
	\text{III: } \sum_{l=1}^\infty \frac{e^{-l}}{l^2+1}
\]
\begin{multiplechoice}
\choice{I inconclusive, converges}
\choice[correct]{I inconclusive, diverges}
\choice{II inconclusive, converges}
\choice{II inconclusive, diverges}
\choice{III inconclusive, converges}
\choice{III inconclusive, diverges}
\end{multiplechoice}
\begin{feedback}
The first series will give an inconclusive result for the ratio test because
\[ 
\lim_{k \rightarrow \infty} \frac{k - e^{-k}}{k+1 - e^{-k-1}} = \lim_{k \rightarrow \infty} \frac{1 - k^{-1} e^{-k}}{\frac{k+1}{k} - k^{-1} e^{-k-1}} = \frac{1 - 0}{1 - 0} = 1.
\]
However, we know that the harmonic series diverges and that
\[ \frac{1}{k - e^{-k}} > \frac{1}{k}, \]
so by direct comparison to the harmonic series, series I must diverge.
\end{feedback}
\begin{keywords}
handwritten,ratio test,series comparison test,harmonic series,conceptual
\end{keywords}
\end{question}


\end{document}
