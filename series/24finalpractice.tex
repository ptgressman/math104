\documentclass{ximera}
\graphicspath{
{./}
{volumes/}
{arclengths/}
{centroids/}
{techniques/}
{applications/}
{series/}
{powerseries/}
{odes/}
}

\newcommand{\bigmath}[1]{$\displaystyle #1$}
\newcommand{\choicebreak}{}
\newenvironment{type}{}{}
\newenvironment{notes}{}{}
\newenvironment{keywords}{}{}
\newcommand{\offline}{}
\newenvironment{comments}{\begin{feedback}}{\end{feedback}}
\newenvironment{multiplechoice}{\begin{multipleChoice}}{\end{multipleChoice}}
\title{Exercises: Cumulative}
%%%%%\author{Philip T. Gressman}

\begin{document}
\begin{abstract}
Exercises relating to various topics we have studied.
\end{abstract}
\maketitle

\begin{exercise}
Determine whether the series below converges absolutely, conditionally, or diverges.
\[ \sum_{n=1}^\infty \frac{(-1)^n}{n - e^{-n}} \]
\begin{multipleChoice}
\choice{Absolute}
\choice[correct]{Conditional}
\choice{Diverge}
\end{multipleChoice}
\end{exercise}

\begin{exercise}
Determine whether the series below converges absolutely, conditionally, or diverges.
\[ \sum_{n=1}^\infty \frac{(-1)^n e^{-n}}{n - e^{-n}} \]
\begin{multipleChoice}
\choice[correct]{Absolute}
\choice{Conditional}
\choice{Diverge}
\end{multipleChoice}
\begin{hint}
Take absolute values and try direct comparison test.
\end{hint}
\begin{hint}
Try direct comparison to $e^{-n}$.
\end{hint}
\end{exercise}

\begin{exercise}
Determine whether the series below converges absolutely, conditionally, or diverges.
\[ \sum_{n=1}^\infty \frac{(-1)^n n}{n - e^{-n}} \]
\begin{multipleChoice}
\choice{Absolute}
\choice{Conditional}
\choice[correct]{Diverge}
\end{multipleChoice}
\begin{hint}
What is the dominant term in the denominator as $n \rightarrow \infty$?
\end{hint}
\begin{hint}
The dominant term in the denominator is $n$.
\end{hint}
\begin{hint}
This means that the terms do not go to zero.
\end{hint}
\end{exercise}

\begin{exercise}
Determine whether the series below converges absolutely, conditionally, or diverges.
\[ \sum_{n=1}^\infty \frac{1}{n^2 + (-1)^n n} \]
\begin{multipleChoice}
\choice[correct]{Absolute}
\choice{Conditional}
\choice{Diverge}
\end{multipleChoice}
\begin{hint}
What is the dominant term in the denominator as $n \rightarrow \infty$?
\end{hint}
\begin{hint}
The dominant term in the denominator is $n^2$.
\end{hint}
\end{exercise}

\begin{exercise}
Determine whether the series below converges absolutely, conditionally, or diverges.
\[ \sum_{n=1}^\infty \frac{(-1)^n}{n - e^{-n}} \]
\begin{multipleChoice}
\choice{Absolute}
\choice[correct]{Conditional}
\choice{Diverge}
\end{multipleChoice}
\end{exercise}

\begin{exercise}
Determine whether the series below converges absolutely, conditionally, or diverges.
\[ \sum_{n=1}^\infty \frac{(-1)^n \sqrt{n}}{\ln (n+1)} \]
\begin{multipleChoice}
\choice{Absolute}
\choice{Conditional}
\choice[correct]{Diverge}
\end{multipleChoice}
\begin{hint}
Which term dominates as $n \rightarrow \infty$: $\sqrt{n}$ or $\ln (n+1)$?
\end{hint}
\begin{hint}
Answer: $\sqrt{n}$ dominates $\ln (n+1)$ as $n \rightarrow \infty$.
\end{hint}
\end{exercise}

\begin{exercise}
Determine whether the series below converges absolutely, conditionally, or diverges.
\[ \sum_{n=1}^\infty \frac{(-1)^n}{\sqrt{n} + \ln (n+1)} \]
\begin{multipleChoice}
\choice{Absolute}
\choice[correct]{Conditional}
\choice{Diverge}
\end{multipleChoice}
\begin{hint}
To show that convergence is not absolute, try limit comparison.
\end{hint}
\begin{hint}
The comparison series can be taken to be $1/\sqrt{n}$ in this case.
\end{hint}
\end{exercise}

\begin{exercise}
Determine whether the series below converges absolutely, conditionally, or diverges.
\[ \sum_{n=1}^\infty \frac{1}{n} \ln \left(2 + \frac{1}{n} \right) \]
\begin{multipleChoice}
\choice{Absolute}
\choice{Conditional}
\choice[correct]{Diverge}
\end{multipleChoice}
\begin{hint}
Try limit comparison with the harmonic series.
\end{hint}
\end{exercise}

\begin{exercise}
Determine whether the series below converges absolutely, conditionally, or diverges.
\[ \sum_{n=1}^\infty \frac{n \cos n \pi}{n^2 - 1} \]
\begin{multipleChoice}
\choice{Absolute}
\choice[correct]{Conditional}
\choice{Diverge}
\end{multipleChoice}
\begin{hint}
With absolute values, compare to a harmonic series.
\end{hint}
\begin{hint}
How do we know that the alternating series test applies? 
\end{hint}
\end{exercise}


\section*{Sample Exam Questions}

\begin{question}%%%%%[2015C.04.alt]

Determine whether the following series converge or diverge.
\[ \text{I: } \sum_{n=1}^\infty \frac{n^3}{n^4+4} \ \ \
\text{II: } \sum_{n=1}^\infty \frac{3^n}{n!} \ \ \ 
\text{III: } \sum_{n=2}^\infty \frac{ \ln \ln n}{\ln n} \ \ \
\text{IV: } \sum_{n=1}^\infty \frac{3n^2}{(n!)^2} \]
\begin{multiplechoice}
\choice{I \& II converge; III \& IV diverge}
\choice{I \& III converge; II \& IV diverge}
\choice{I \& IV converge; II \& III diverge}
\choice{II \& III converge; I \& IV diverge}
\choice[correct]{II \& IV converge; I \& III diverge}
\choice{III \& IV converge; I \& II diverge}
\end{multiplechoice}

\end{question}

\begin{question}%%%%%[2016C.09]

Determine whether the following series are convergent or divergent. Justify your answers.
\[ \text{I: } \sum_{n=1}^\infty \frac{n^2-3n}{\sqrt[3]{n^{10}-4n^2}} \ \ \ \ \text{II: } \sum_{n=1}^\infty \frac{(-n)^n}{5^{2n+3}} \]
\begin{multiplechoice}
\choice{I \& II divergent}
\choice[correct]{I convergent, II divergent} 
\choice{I divergent, II convergent}
\choice{I \& II convergent}
\end{multiplechoice}

\end{question}

\begin{question}%%%%%[2016C.10]

Determine whether the following series are convergent or divergent. Justify your answers.
\[ \text{I: } \sum_{n=1}^\infty \frac{\arctan n}{n^4} \ \ \ \ \text{II: } \sum_{n=1}^\infty \frac{\sin \frac{1}{n}}{n^2} \]
\begin{multiplechoice}
\choice{I \& II divergent}
\choice{I convergent, II divergent} 
\choice{I divergent, II convergent}
\choice[correct]{I \& II convergent}
\end{multiplechoice}

\end{question}

\begin{question}%%%%%[2017C.11]

Determine which of the following series are convergent. \offline{For full credit, be sure to explain your reasoning and specify which tests were used.}
\[ \text{I: } \sum_{n=2}^\infty 2ne^{-n^2} \ \ \ \ \text{II: } \sum_{n=2}^\infty \frac{n + 2 \ln n}{2 n^4} \ \ \ \ \text{III: } \sum_{n=2}^\infty \frac{n^n}{n!} \]
\begin{multiplechoice}
\choice{only I}
\choice[correct]{only I and II}
\choice{only I and III}
\choice{only II}
\choice{only II and III}
\choice{only III}
\end{multiplechoice}

\end{question}


\end{document}
