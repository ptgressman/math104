\documentclass{ximera}
\graphicspath{
{./}
{volumes/}
{arclengths/}
{centroids/}
{techniques/}
{applications/}
{series/}
{powerseries/}
{odes/}
{lessons/}
}
\usepackage{booktabs}

\newcommand{\bigmath}[1]{$\displaystyle #1$}
\newcommand{\choicebreak}{}
\newenvironment{type}{}{}
\newenvironment{notes}{}{}
\newenvironment{keywords}{}{}
\newcommand{\offline}{}
\newenvironment{comments}{\begin{feedback}}{\end{feedback}}
\newenvironment{multiplechoice}{\begin{multipleChoice}}{\end{multipleChoice}}
\title{Exercises: The Integral Test}
%%%%%\author{Philip T. Gressman}

\begin{document}
\begin{abstract}
Exercises relating to the integral test.
\end{abstract}
\maketitle



\begin{exercise}
The function
\[ \frac{\cos \pi x}{x} \]
is decreasing for $x \geq \answer{N/A}$ (if the function is not ultimately decreasing, enter N/A). The integral test \wordChoice{\choice{does}\choice[correct]{does not}} apply to the series
\[ \sum_{n=1}^\infty \frac{\cos \pi n}{n}. \]
The series \wordChoice{\choice{converges}\choice{diverges}\choice[correct]{can't be determined using this test}}.
\end{exercise}

\begin{exercise}
The function \[ \frac{\ln x}{x} \]
is decreasing for $x \geq \answer{e}$ (if the function is not ultimately decreasing, enter N/A). The integral test \wordChoice{\choice[correct]{does}\choice{does not}} apply to the series
\[ \sum_{n=3}^\infty \frac{\ln n}{n}. \]
We have
\[ \int_3^\infty \frac{\ln x}{x} dx = \answer{\infty}, \]
so the series \wordChoice{\choice{converges}\choice[correct]{diverges}\choice{can't be determined using this test}}.
\end{exercise}

\begin{exercise}
The function \[ \frac{1}{x^2+1} \]
is decreasing for $x \geq \answer{0}$ (if the function is not ultimately decreasing, enter N/A). The integral test \wordChoice{\choice[correct]{does}\choice{does not}} apply to the series
\[ \sum_{n=1}^\infty \frac{1}{n^2+1}. \]
We have that
\[ \int_1^\infty \frac{dx}{x^2 + 1} = \answer{\frac{\pi}{4}}, \]
so the series \wordChoice{\choice[correct]{converges}\choice{diverges}\choice{can't be determined using this test}}.
\end{exercise}

\begin{exercise}
The integral test 
\wordChoice{\choice[correct]{does}\choice{does not}} apply to the series
\[ \sum_{n=2}^\infty \frac{1}{n (\ln n)^2}. \]
The series \wordChoice{\choice[correct]{converges}\choice{diverges}\choice{can't be determined using this test}}.
\begin{hint}
Make a substitution $u = \ln x$.
\end{hint}
\end{exercise}

\begin{exercise}
The integral test 
\wordChoice{\choice[correct]{does}\choice{does not}} apply to the series
\[ \sum_{n=2}^\infty \frac{1}{n (\ln n)}. \]
The series \wordChoice{\choice{converges}\choice[correct]{diverges}\choice{can't be determined using this test}}.
\begin{hint}
Make a substitution $u = \ln x$.
\end{hint}
\end{exercise}

\begin{exercise}
The integral test 
\wordChoice{\choice{does}\choice[correct]{does not}} apply to the series
\[ \sum_{n=2}^\infty e^{-n^2} \cos n \pi . \]
The series \wordChoice{\choice{converges}\choice{diverges}\choice[correct]{can't be determined using this test}}.
\end{exercise}

\begin{exercise}
The integral test 
\wordChoice{\choice[correct]{does}\choice{does not}} apply to the series
\[ \sum_{n=2}^\infty n^3 e^{-n^2}. \]
The series \wordChoice{\choice[correct]{converges}\choice{diverges}\choice{can't be determined using this test}}.
\begin{hint}
Make a substitution $u = x^2$ and then integrate by parts.
\end{hint}
\end{exercise}


\begin{exercise}
The function 
\[ \frac{x}{(x^2+3)^2} \]
is decreasing for $x \geq \answer{1}$.
By the integral test,  
\[ \int_{\answer{8}}^\infty \frac{x}{(x^2+3)^2} dx \leq \sum_{n=8}^\infty \frac{n}{(n^2+3)^2} \leq \int_{\answer{7}}^\infty \frac{x}{(x^2+3)^2} dx. \]
We can approximate the infinite series by the sum of the first seven terms with what bounds on the error?
\[ \frac{1}{\answer{134}} + \sum_{n=1}^7 \frac{n}{(n^2+3)^2} \leq \sum_{n=1}^\infty \frac{n}{(n^2+3)^2} \leq \frac{1}{\answer{104}} + \sum_{n=1}^7 \frac{n}{(n^2+3)^2}. \]
\begin{hint}
Write
\[ \sum_{n=1}^\infty \frac{n}{(n^2+3)^2} = \sum_{n=1}^7 + \sum_{n=8}^\infty \frac{n}{(n^2+3)^2}  \]
and then use the bounds we know for the ``tail'' (i.e., the sum over $n \geq 8$).
\end{hint}
\end{exercise}

\begin{exercise}
Using the integral test, we can determine that the sum of the series
\[ \sum_{n=1}^\infty \frac{1}{n^2} \]
is \wordChoice{\choice{equal to}\choice{greater than}\choice[correct]{less than}} $2$.
\begin{hint}
Compare the series to the partial sum of the first three terms. Don't forget to include an estimate of the remainder.
\begin{hint}
\[ \sum_{n=1}^\infty \frac{1}{n^2} \leq \sum_{n=1}^3 \frac{1}{n^2} + \int_{3}^\infty \frac{dx}{x^2}. \]
\end{hint}
\end{hint}
\end{exercise}

\section*{Sample Quiz Questions}

\begin{question}%%%%%[2019IntegTestErr1]

When approximating the sum of the infinite series
\[ \sum_{n=1}^\infty \frac{1}{n^3} \]
by the sum of the first \(N\) terms, how large must \(N\) be to ensure that the approximation error is less than \(1/200\)? Choose the smallest correct bound among those listed.
\begin{multiplechoice}
\choice{\(N > 5\)}
\choice[correct]{\(N > 10\)}
\choice{\(N > 20\)}
\choice{\(N > 400\)}
\choice{\(N > 8000\)}
\choice{\(N > 160000\)}
\end{multiplechoice}
\begin{feedback}
Because the terms \(n^{-3}\) are positive and decreasing, we know that the partial sums are always less than or equal to the sum of the series. By the Integral Test, we can further say that
\[ \sum_{n=1}^\infty \frac{1}{n^3} - \sum_{n=1}^N \frac{1}{n^3} \leq \int_N^\infty \frac{1}{x^3} dx  = \frac{1}{2N^2}.\]
To be certain that the error is less than \(1/200\), we set \((2 N^2)^{-1} < 1/200\), which gives \(N > 10\).
\end{feedback}

\end{question}

\end{document}
