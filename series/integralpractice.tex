\documentclass{ximera}
\graphicspath{
{./}
{volumes/}
{arclengths/}
{centroids/}
{techniques/}
{applications/}
{series/}
{powerseries/}
{odes/}
{lessons/}
}
\usepackage{booktabs}

\newcommand{\bigmath}[1]{$\displaystyle #1$}
\newcommand{\choicebreak}{}
\newenvironment{type}{}{}
\newenvironment{notes}{}{}
\newenvironment{keywords}{}{}
\newcommand{\offline}{}
\newenvironment{comments}{\begin{feedback}}{\end{feedback}}
\newenvironment{multiplechoice}{\begin{multipleChoice}}{\end{multipleChoice}}
\title{Exercises: The Integral Test}
\author{Philip T. Gressman}

\begin{document}
\begin{abstract}
Exercises relating to the integral test.
\end{abstract}
\maketitle

\section*{Sample Quiz Questions}

\begin{question}[2019IntegTestErr1]
\begin{type}
multiplechoice
\end{type}
When approximating the sum of the infinite series
\[ \sum_{n=1}^\infty \frac{1}{n^3} \]
by the sum of the first \(N\) terms, how large must \(N\) be to ensure that the approximation error is less than \(1/200\)? Choose the smallest correct bound among those listed.
\begin{multiplechoice}
\choice{\(N > 5\)}
\choice[correct]{\(N > 10\)}
\choice{\(N > 20\)}
\choice{\(N > 400\)}
\choice{\(N > 8000\)}
\choice{\(N > 160000\)}
\end{multiplechoice}
\begin{feedback}
Because the terms \(n^{-3}\) are positive and decreasing, we know that the partial sums are always less than or equal to the sum of the series. By the Integral Test, we can further say that
\[ \sum_{n=1}^\infty \frac{1}{n^3} - \sum_{n=1}^N \frac{1}{n^3} \leq \int_N^\infty \frac{1}{x^3} dx  = \frac{1}{2N^2}.\]
To be certain that the error is less than \(1/200\), we set \((2 N^2)^{-1} < 1/200\), which gives \(N > 10\).
\end{feedback}
\begin{keywords}
handwritten,integral test,series estimation
\end{keywords}
\end{question}

\end{document}
