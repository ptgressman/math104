\documentclass{ximera}
\graphicspath{
{./}
{volumes/}
{arclengths/}
{centroids/}
{techniques/}
{applications/}
{series/}
{powerseries/}
{odes/}
{lessons/}
}
\usepackage{booktabs}

\newcommand{\bigmath}[1]{$\displaystyle #1$}
\newcommand{\choicebreak}{}
\newenvironment{type}{}{}
\newenvironment{notes}{}{}
\newenvironment{keywords}{}{}
\newcommand{\offline}{}
\newenvironment{comments}{\begin{feedback}}{\end{feedback}}
\newenvironment{multiplechoice}{\begin{multipleChoice}}{\end{multipleChoice}}
\title{Exercises: Series}
\author{Philip T. Gressman}

\begin{document}
\begin{abstract}
Exercises relating to fundamental properties of series.
\end{abstract}
\maketitle

\begin{exercise}
Compute the sum of the infinite series below.
\[ \sum_{n=1}^\infty \frac{2}{n(n+2)} = \answer{\frac{3}{2}} \]
\begin{hint}
Expand the terms using partial fractions and compute several partial sums by hand. What you get is something like a telescoping series, but cancellations occur in a slightly different way than usual.
\begin{hint}
\[ \frac{2}{n(n+2)} = \frac{1}{n} - \frac{1}{n+2}. \]
\begin{hint}
The formula for a general partial sum is
\[ \frac{2}{1 \cdot 3} + \cdots + \frac{2}{N \cdot (N+2)} = 1 + \frac{1}{2} - \frac{1}{N+1} - \frac{1}{N+2} \]
\end{hint}
\end{hint}
\end{hint}
\end{exercise}

\section*{Sample Quiz Questions}

\begin{question}%%%%%[2019Telescope1]

Compute the exact value of the infinite series
\[ \sum_{n=1}^\infty \ln \left( \frac{1 + n^{-1}}{1 + (n+1)^{-1}} \right). \]
\begin{multiplechoice}
\choice[correct]{\(\ln 2\)}
\choice{\(\ln 3\)}
\choice{\(\ln 4\)}
\choice{\(\ln 5\)}
\choice{\(\ln 6\)}
\choice{\(\ln 7\)}
\end{multiplechoice}
\begin{feedback}
The series is not a geometric series or Taylor series, we compute the first few partial sums:
\[ 
\begin{aligned}
S_1 & = \ln \left( \frac{1 + 1}{1 + 2^{-1}} \right)  = \ln \left( \frac{2}{\frac{3}{2}} \right) \\
S_2 & = \ln \left( \frac{1 + 1}{1 + 2^{-1}} \right)  + \ln \left( \frac{1 + 2^{-1}}{1 + 3^{-1}} \right)  = \ln \left( \frac{1 + 1}{1 + 3^{-1}} \right) = \ln \left( \frac{2}{\frac{4}{3}} \right) \\
S_3 & = \ln \left( \frac{2}{1 + 3^{-1}} \right)  + \ln \left( \frac{1 + 3^{-1}}{1 + 4^{-1}} \right)  = \ln \left( \frac{2}{1 + 4^{-1}} \right) = \ln \left( \frac{2}{\frac{5}{4}} \right) \\
& \ \vdots  \\
S_n & = \ln \left( \frac{2}{1 + (n+1)^{-1}} \right).
 \end{aligned}
\]
In particular, writing the sum of logarithms as a logarithm of a product leads to substantial cancellation.  By letting \(n \rightarrow \infty\), we get \(S_n \rightarrow \ln 2\).
\end{feedback}

\end{question}




\end{document}
