\documentclass{ximera}
\graphicspath{
{./}
{volumes/}
{arclengths/}
{centroids/}
{techniques/}
{applications/}
{series/}
{powerseries/}
{odes/}
{lessons/}
}
\usepackage{booktabs}

\newcommand{\bigmath}[1]{$\displaystyle #1$}
\newcommand{\choicebreak}{}
\newenvironment{type}{}{}
\newenvironment{notes}{}{}
\newenvironment{keywords}{}{}
\newcommand{\offline}{}
\newenvironment{comments}{\begin{feedback}}{\end{feedback}}
\newenvironment{multiplechoice}{\begin{multipleChoice}}{\end{multipleChoice}}
\title{Exercises: Alternating Series}
%%%%%\author{Philip T. Gressman}

\begin{document}
\begin{abstract}
Exercises relating to alternating series and absolute or conditional convergence.
\end{abstract}
\maketitle


\begin{exercise}
For the infinite series
\[ \sum_{n=2}^\infty \frac{(-1)^n}{ \ln n}, \]
the function $1/\ln n$ \wordChoice{\choice[correct]{is}\choice{is not}} positive and \wordChoice{\choice[correct]{is}\choice{is not}}  decreasing when $n \geq 2$. Furthermore it \wordChoice{\choice[correct]{does}\choice{does not}} tend to zero as $n \rightarrow \infty$. The alternating series test \wordChoice{\choice[correct]{does}\choice{does not}} apply and \wordChoice{\choice[correct]{implies convergence}\choice{implies divergence}\choice{says nothing about the series}}.
\end{exercise}

\begin{exercise}
For the infinite series
\[ \sum_{n=2}^\infty \frac{(-1)^n}{ \ln n + (-1)^n \sqrt{\ln n}}, \]
the function $1/(\ln n+(-1)^n \sqrt{\ln n})$ \wordChoice{\choice[correct]{is}\choice{is not}} positive and \wordChoice{\choice{is}\choice[correct]{is not}} decreasing when $n \geq 2$. Furthermore it \wordChoice{\choice[correct]{does}\choice{does not}} tend to zero as $n \rightarrow \infty$. The alternating series test \wordChoice{\choice{does}\choice[correct]{does not}} apply and \wordChoice{\choice{implies convergence}\choice{implies divergence}\choice[correct]{says nothing about the series}}.
\end{exercise}

\begin{exercise}
For the infinite series
\[ \sum_{n=3}^\infty \frac{(-1)^n}{ \ln n - (\ln n)^{-1/2}}, \]
the function $1/(\ln n - (\ln n)^{-1/2})$ \wordChoice{\choice[correct]{is}\choice{is not}} positive and \wordChoice{\choice[correct]{is}\choice{is not}} decreasing when $n \geq 3$. Furthermore it \wordChoice{\choice[correct]{does}\choice{does not}} tend to zero as $n \rightarrow \infty$. The alternating series test \wordChoice{\choice[correct]{does}\choice{does not}} apply and \wordChoice{\choice[correct]{implies convergence}\choice{implies divergence}\choice{says nothing about the series}}.
\end{exercise}


\begin{exercise}
For the infinite series
\[ \sum_{n=2}^\infty \frac{(-1)^n (n-1)^2}{n^2}, \]
the function $1/(\ln n+(-1)^n \sqrt{\ln n})$ \wordChoice{\choice[correct]{is}\choice{is not}} positive and \wordChoice{\choice[correct]{is}\choice{is not}} decreasing when $n \geq 2$. Furthermore it \wordChoice{\choice{does}\choice[correct]{does not}} tend to zero as $n \rightarrow \infty$. The alternating series test \wordChoice{\choice{does}\choice[correct]{does not}} apply and \wordChoice{\choice{implies convergence}\choice{implies divergence}\choice[correct]{says nothing about the series}}.
\end{exercise}

\begin{exercise}
For the infinite series
\[ \sum_{n=1}^\infty \frac{\cos \ln n}{ e^n}, \]
the function $1/(e^n)$ \wordChoice{\choice[correct]{is}\choice{is not}} positive and \wordChoice{\choice[correct]{is}\choice{is not}} decreasing when $n \geq 1$. Furthermore it \wordChoice{\choice[correct]{does}\choice{does not}} tend to zero as $n \rightarrow \infty$. The alternating series test \wordChoice{\choice{does}\choice[correct]{does not}} apply and \wordChoice{\choice{implies convergence}\choice{implies divergence}\choice[correct]{says nothing about the series}}.
\begin{hint}
First determine whether the series is alternating.
\end{hint}
\end{exercise}


\begin{exercise}
Does the alternating series test apply to the series?
\[ \sum_{n=5}^\infty \frac{\cos n \pi}{ \sqrt{n}} \]
\begin{multipleChoice}
\choice[correct]{Yes}
\choice{No, it's not alternating}
\choice{No, the terms are not decreasing}
\choice{No, the terms do not go to zero}
\end{multipleChoice}
If yes, for what minimum value of $N$ can you be certain that
\[ \sum_{n=5}^{N-1} \frac{\cos n \pi}{\sqrt{n}} \]
differs from the sum of the series by at most $10^{-3}$? If no such $N$ exists, write N/A.
\[ N \geq \answer{10^{6}}. \]
\begin{hint}
If the alternating series test applies, we would need the magnitude (i.e., absolute value) of the first term \textit{not} included in the partial sum to be no greater than $10^{-3}$.
\end{hint}
\end{exercise}

\begin{exercise}
Find an interval of length $\frac{1}{4}$ which contains the sum of the infinite series
\[ \sum_{n=1}^\infty \frac{(-1)^{n-1}}{n^2} \]
\[ \text{Interval }  = \left[ \answer{\frac{3}{4}}, \answer{1} \right] \]
\begin{hint}
Partial sums of an alternating series also alternate above and below the sum of the series itself.
\end{hint}
\end{exercise}


\section*{Sample Quiz Questions}

\begin{question}%%%%%[seriesacd001]

For each series below, determine whether it converges absolutely (A), converges conditionally (C), or diverges (D). \offline{Show how you used convergence tests to arrive at your answer.} \[\text{I: } \sum_{n = 1}^\infty\frac{\cos n \pi}{\sqrt[3]{n+3}} \qquad \text{II: } \sum_{n = 1}^\infty\frac{(-1)^n}{n^2+3} \qquad \text{III: } \sum_{n = 2}^\infty\frac{\cos n \pi}{\ln (n^2 + 1)}\]
\begin{multiplechoice}
\choice{I: C, II: D, III: D}
\choice[correct]{I: C, II: A, III: C}
\choice{I: A, II: C, III: A} 
\choice{I: D, II: C, III: D}
\choice{I: C, II: D, III: C}
\choice{I: C, II: A, III: A}
\end{multiplechoice}
\begin{feedback}
I: converges conditionally. The value of \(\cos n \pi\) alternates \(\pm 1\). The terms \((n+3)^{-1/3}\) decrease to zero, so the series converges by the alternating series test. The series is not absolutely convergent because the \(p\)-series with \(p = -1/3\) is divergent.

II: converges absolutely. The series converges absolutely by direct comparison to a \(p\)-series with \(p=2\).

III: converges conditionally. The series converges by the alternating series test because \(1/\ln (n^2+1)\) decreases to \(0\) as \(n \rightarrow \infty\) and \(\cos n \pi\) alternates in value between \(+1\) and \(-1\). However, \(1/\ln (n^2+1) \geq 1/n\) for all large \(n\), so by direct comparison to the harmonic series, the series is not absolutely convergent. Therefore the convergence is conditional.
\end{feedback}

\end{question}

\begin{question}%%%%%[seriesacd004]

For each series below, determine whether it converges absolutely (A), converges conditionally (C), or diverges (D). \offline{Show how you used convergence tests to arrive at your answer.} \[\text{I: } \sum_{n = 1}^\infty\frac{(-1)^n n^2}{2 n^2+1} \qquad \text{II: } \sum_{n = 1}^\infty\frac{1}{1 + n^3 e^{-n}} \qquad \text{III: } \sum_{n = 1}^\infty\frac{(-1)^n n + 2}{n^2}\]
\begin{multiplechoice}
\choice{I: D, II: D, III: D}
\choice{I: D, II: A, III: C}
\choice{I: C, II: C, III: A} 
\choice{I: A, II: C, III: D}
\choice[correct]{I: D, II: D, III: C}
\choice{I: D, II: A, III: A}
\end{multiplechoice}
\begin{feedback}
I: diverges. The series diverges because \(n^2/(n^2+1) \rightarrow 1\), meaning that the terms do not go to zero. The \(n\)-th term divergence test implies divergence.

II: diverges. The series diverges because \(n / (n + n^3 e^{-n}) \rightarrow 1\) (because \(n^3 e^{-n} \rightarrow 0\)). By the limit comparison theorem, this means the series has the same behavior as a \(p\)-series with \(p=1\), which means it diverges.

III: converges conditionally. The series converges because it is the sum of two convergent series: one with terms \((-1)^n / n\) (which is a convergent series by the alternating series test because \(1/n\) decreases to zero) and a second with terms \(2/n^2\) (which is a convergent \(p\)-series). However, the series is not absolutely convergent, because \[ \left| \frac{(-1)^n n + 2}{n^2} \right| = \frac{n + (-1)^n2}{n^2}\] for \(n \geq 2\), which is a sum of a {\it divergent} \(p\)-series with \(p=1\) and an absolutely convergent alternating \(p\)-series with \(p=2\). Thus the series is conditionally convergent.
\end{feedback}

\end{question}

\begin{question}%%%%%[2019AltSerError]

Which of the following intervals contains the value of the infinite series
\[ \sum_{n=0}^\infty \frac{(-1)^n}{n+1}? \]
\begin{multiplechoice}
\choice{\(\displaystyle \left[ \frac{1}{4}, \frac{1}{3} \right]\)}
\choice{\(\displaystyle \left[ \frac{1}{3}, \frac{1}{2} \right]\)}
\choice{\(\displaystyle \left[ \frac{1}{2}, \frac{7}{12} \right]\)}
\choice[correct]{\(\displaystyle \left[ \frac{7}{12}, \frac{5}{6} \right]\)}
\choice{\(\displaystyle \left[ \frac{5}{6}, \frac{11}{12} \right]\)}
\choice{\(\displaystyle \left[ \frac{11}{12}, \frac{7}{6} \right]\)}
\end{multiplechoice}
\begin{feedback}
The function \(1/(n+1)\) is positive and decreases to zero, so by the Alternating Series Test, we know that partial sums alternate above and below the actual value of the sum.  In particular, if we call the value of the sum \(L\), then
\[
\begin{aligned}
1 & \geq  L \\
1 - \frac{1}{2} & \leq  L \\
1 - \frac{1}{2} + \frac{1}{3} & \geq L \\
1 - \frac{1}{2} + \frac{1}{3} - \frac{1}{4} & \leq L
\end{aligned}
\]
and so on. The last two inequalities together imply that \(L\) belongs to the interval \([\frac{7}{12},\frac{5}{6}]\).
\end{feedback}

\end{question}

\section*{Sample Exam Questions}

\begin{question}%%%%%[2017C.12]

Determine whether the following series converge absolutely (A), converge conditionally (C), or diverge (D). \offline{For full credit be sure to explain your reasoning and specify which tests were used.}
\[ \sum_{n=2}^\infty \frac{(-1)^n 2^{2n}}{3^n} \ \ \ \ \sum_{n=2}^\infty \frac{(-1)^n}{\sqrt{n}} \]
\begin{multiplechoice}
\choice{both A}
\choice{one A, the other C}
\choice{one A, the other D} 
\choice{both C}
\choice[correct]{one C, the other D}
\choice{both D}
\end{multiplechoice}

\end{question}


\end{document}
