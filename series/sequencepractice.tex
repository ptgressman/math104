\documentclass{ximera}
\graphicspath{
{./}
{volumes/}
{arclengths/}
{centroids/}
{techniques/}
{applications/}
{series/}
{powerseries/}
{odes/}
}

\newcommand{\bigmath}[1]{$\displaystyle #1$}
\newcommand{\choicebreak}{}
\newenvironment{type}{}{}
\newenvironment{notes}{}{}
\newenvironment{keywords}{}{}
\newcommand{\offline}{}
\newenvironment{comments}{\begin{feedback}}{\end{feedback}}
\newenvironment{multiplechoice}{\begin{multipleChoice}}{\end{multipleChoice}}
\title{Exercises: Sequences}
\author{Philip T. Gressman}

\begin{document}
\begin{abstract}
Exercises relating to sequences.
\end{abstract}
\maketitle


\section*{Sample Quiz Questions}

\begin{question}%%%%%[SeqConvSQRT01]

Find the limit of the sequence
\[\lim_{n \rightarrow \infty} \sqrt{\frac{2n-2}{2n^2-4n+3}}.\] \offline{Justify your response.}
\begin{multiplechoice}
\choice[correct]{\(\displaystyle 0\)}
\choice{\(\displaystyle \frac{1}{3}\)}
\choice{\(\displaystyle \frac{1}{2}\)}
\choice{\(\displaystyle 1\)}
\choice{\(\displaystyle 2\)}
\choice{\(\displaystyle 3\)}
\end{multiplechoice}
\begin{feedback}
\[\begin{aligned}
\lim_{n \rightarrow \infty} \sqrt{\frac{2n-2}{2n^2-4n+3}} & = \lim_{n \rightarrow \infty} \sqrt{\frac{2n^{-1}-2n^{-2}}{2-4n^{-1}+3n^{-2}}} \\
 & = \sqrt{\frac{\lim_{n \rightarrow \infty} 2n^{-1}-2n^{-2}}{\lim_{n \rightarrow \infty} 2-4n^{-1}+3n^{-2}}} \\
 & = \sqrt{\frac{0}{2}} = 0. 
\end{aligned}\]
\end{feedback}

\end{question}

\begin{question}%%%%%[SeqConvORDER19]

Determine whether the limit below exists. If it exists, find its value.
\[\lim_{n \rightarrow \infty} \frac{(-1)^{n+1}n^2 - 2^{-n-1}}{(-1)^{n}n^2 + 4^{-n}}.\] \offline{Justify your response.}
\begin{multiplechoice}
\choice[correct]{\(\displaystyle -1\)}
\choice{\(\displaystyle 0\)}
\choice{\(\displaystyle \frac{1}{2}\)}
\choice{\(\displaystyle 2\)}
\choice{\(\displaystyle 3\)}
\choice{limit does not exist}
\end{multiplechoice}
\begin{feedback}
Comparing the orders of growth of the terms in the numerator, the first term dominates because \(|-1|>|1/2|\). Likewise the first term dominates in the denominator because  \(|-1|>|1/4|\). Neglecting non-dominant terms leads to the limit \[\lim_{n \rightarrow \infty} \frac{(-1)^{n+1}n^2}{(-1)^{n}n^2}\] which simply equals \(-1\).
\end{feedback}

\end{question}

\begin{question}%%%%%[SeqConvEXPN43]

Determine whether the limit below exists. If it exists, find its value.
\[\lim_{n \rightarrow \infty} \left(\frac{4n - 3}{4n + 1}\right)^{n}.\] \offline{Justify your response.}
\begin{multiplechoice}
\choice{\(\displaystyle 0\)}
\choice{\(\displaystyle 1\)}
\choice[correct]{\(\displaystyle e^{-1}\)}
\choice{\(\displaystyle e\)}
\choice{\(\displaystyle e^2\)}
\choice{limit does not exist}
\end{multiplechoice}
\begin{feedback}
First observe that 
\[\frac{4n - 3}{4n + 1} = 1 - \frac{4}{4n + 1} \rightarrow 1\]
as \(n \rightarrow \infty\). Next, in light of the known limit \((1+x/k)^{k} \rightarrow e^x\) as \(k \rightarrow \infty\), manipulate exponents to see that  \[\left(1 - \frac{4}{4n + 1}\right)^n = \left( \left(1 - \frac{4}{4n + 1}\right)^{4n + 1} \right)^{ 1/4} \left(1 - \frac{4}{4n + 1}\right)^{-1/4}. \] As \(n \rightarrow \infty\), the first term on the right-hand side tends to \(e^{-1}\) and the second term tends to \(1\). Thus the original sequence tends to \(e^{-1}\) as well.
\end{feedback}

\end{question}

\begin{question}%%%%%[SeqConvEXPN39]

Determine whether the limit below exists. If it exists, find its value.
\[\lim_{n \rightarrow \infty} \left(\frac{n + 3}{2n - 2}\right)^{n^2}.\] \offline{Justify your response.}
\begin{multiplechoice}
\choice[correct]{\(\displaystyle 0\)}
\choice{\(\displaystyle 1\)}
\choice{\(\displaystyle e^{-1}\)}
\choice{\(\displaystyle e\)}
\choice{\(\displaystyle e^2\)}
\choice{limit does not exist}
\end{multiplechoice}
\begin{feedback}
First observe that 
\[\frac{n + 3}{2n - 2} = \frac{1}{2} + \frac{2}{n - 1} \rightarrow \frac{1}{2}\]
as \(n \rightarrow \infty\). Since the limit is positive and less than one, raising this expression to increasingly large powers generates a sequence which converges rapidly to zero.
\end{feedback}

\end{question}

\section*{Sample Exam Questions}

\begin{question}%%%%%[2016C.08]

Determine whether the sequence \(\displaystyle a_n = (-1)^{n-1} \frac{n^2}{1 + n^2 + n^3}\) converges or diverges. If it converges, find its limit.
\begin{multiplechoice}
\choice{divergent, \(\displaystyle \lim_{n \rightarrow \infty} a_n = 0\)}
\choice{convergent, \(\displaystyle \lim_{n \rightarrow \infty} a_n = 1\)}
\choice[correct]{convergent, \(\displaystyle \lim_{n \rightarrow \infty} a_n = 0\)}
\choice{convergent, \(\displaystyle \lim_{n \rightarrow \infty} a_n = -1\)}
\choice{divergent, \(\displaystyle \lim_{n \rightarrow \infty} a_n = \infty\)}
\choice{divergent, limit doesn't exist}
\end{multiplechoice}

\end{question}


\end{document}
