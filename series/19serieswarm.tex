\documentclass{ximera}
\graphicspath{
{./}
{volumes/}
{arclengths/}
{centroids/}
{techniques/}
{applications/}
{series/}
{powerseries/}
{odes/}
{lessons/}
}
\usepackage{booktabs}

\newcommand{\bigmath}[1]{$\displaystyle #1$}
\newcommand{\choicebreak}{}
\newenvironment{type}{}{}
\newenvironment{notes}{}{}
\newenvironment{keywords}{}{}
\newcommand{\offline}{}
\newenvironment{comments}{\begin{feedback}}{\end{feedback}}
\newenvironment{multiplechoice}{\begin{multipleChoice}}{\end{multipleChoice}}
\title{Series}
%%%%%\author{Philip T. Gressman}

\begin{document}
\begin{abstract}
We introduce the concept of a series and study some fundamental properties.
\end{abstract}
\maketitle

\section*{Online Texts}
\begin{itemize}
\item \link[OpenStax II 5.2: Infinite Series]{https://openstax.org/books/calculus-volume-2/pages/5-2-infinite-series} and \link[OpenStax II 5.3: The Divergence Test]{https://openstax.org/books/calculus-volume-2/pages/5-3-the-divergence-and-integral-tests}
\item \link[Ximera OSU: Series]{https://ximera.osu.edu/mooculus/calculus2/sumsOfSequences/titlePage} and \link[Ximera OSU: The Divergence Test]{https://ximera.osu.edu/mooculus/calculus2/divergenceTest/titlePage}
\item \link[Community Calculus 11.2: Series]{https://www.whitman.edu/mathematics/calculus_online/section11.02.html}
\end{itemize}

\section*{Examples}

\begin{example}
Find a formula for the partial sums of the series
\[ \sum_{n=1}^\infty \left[ e^{\frac{1}{n}} - e^{\frac{1}{n+1}}\right]  \]
and then compute the sum of the series.
\begin{itemize}
\item We can start by computing the first few partial sums:
\[ \begin{aligned} 
S_1 & = (e - e^{1/2}) \\
S_2 & = (e^{1} - e^{1/2}) + (e^{1/2} - e^{1/3}) = e - e^{1/3} \\
S_3 & = (e^{1} - e^{1/2}) + (e^{1/2} - e^{1/3}) + (e^{1/3} - e^{1/4}) = e - e^{1/4} \\
S_4 & = (e^{1} - e^{1/2}) + (e^{1/2} - e^{1/3}) + (e^{1/3} - e^{1/4}) + (e^{1/4} - e^{1/5}) = e - e^{1/5} \end{aligned} \]
We observe a general pattern that is consistent with a telescoping series:
\[ S_n = e - \answer{e^{1/(n+1)}} \]
\item Now we let $n \rightarrow \infty$:
\[ \lim_{n \rightarrow \infty} S_n = \lim_{n \rightarrow \infty} \left( \answer{e - e^{1/(n+1)}} \right) = e - e^{\lim_{n \rightarrow \infty} \frac{1}{n+1}} = e - e^{\answer{0}} = e - 1. \]
Therefore
\[ \sum_{n=1}^\infty \left[ e^{\frac{1}{n}} - e^{\frac{1}{n+1}}\right] = \answer{e-1}. \]
\end{itemize}
\end{example}

\begin{example}
Show that each series below is a geometric series; determine which ones are convergent, and for those that are convergent, find their sum.
\[ \sum_{n=0}^\infty \frac{3^n}{2} \qquad \sum_{n=0}^\infty \frac{2}{3^n} \qquad \sum_{n=0}^\infty 2^{-n-2}  \qquad \sum_{n=0}^\infty \frac{(-1)^{n+1} 3^n}{4^n} \]
\begin{itemize}
\item First we rewrite each geometric series in standard form: i.e., write each term in the form $a r^n$ where $a$ and $r$ are fixed quantities independent of $n$:
\[ \sum_{n=0}^\infty \frac{3^n}{2} = \sum_{n=0}^\infty \answer{\frac{1}{2}} \left[ \answer{3} \right]^n \]
\[ \sum_{n=0}^\infty \frac{2}{3^n} = \sum_{n=0}^\infty \answer{2} \left[ \answer{\frac{1}{3}} \right]^n \]
\[ \sum_{n=0}^\infty 2^{-n-2} = \sum_{n=0}^\infty \answer{\frac{1}{4}} \left[ \answer{\frac{1}{2}} \right]^n \]
\[ \sum_{n=0}^\infty \frac{(-1)^{n+1} 3^n}{4^n} = \sum_{n=0}^\infty \answer{-1} \left[ \answer{\frac{-3}{4}} \right]^n \]
\item A geometric series is convergent exactly when $|r| < \answer{1}$. This means that the following series are convergent (select all that apply):
\begin{selectAll}
\choice{$\displaystyle \sum_{n=0}^\infty \frac{3^n}{2}$}
\choice[correct]{$\displaystyle \sum_{n=0}^\infty \frac{2}{3^n}$}
\choice[correct]{$\displaystyle  \sum_{n=0}^\infty 2^{-n-2}$}
\choice[correct]{$\displaystyle \sum_{n=0}^\infty \frac{(-1)^{n+1} 3^n}{4^n}$}
\end{selectAll}
\item This means that the given series have the sums (write N/A if the series diverges, otherwise give the sum)
\[  \sum_{n=0}^\infty \frac{3^n}{2} = \answer{N/A} \qquad \sum_{n=0}^\infty \frac{2}{3^n} = \answer{3} \qquad \sum_{n=0}^\infty 2^{-n-2}  = \answer{\frac{1}{2}} \qquad \sum_{n=0}^\infty \frac{(-1)^{n+1} 3^n}{4^n} = \answer{-\frac{4}{7}}. \]
\end{itemize}
\end{example}

\begin{example}
Reindex each series below to begin at the specified starting value:

Series 1: \begin{hint} Inside each term, replace every occurrence of $n$ with $n-1$. \end{hint}
\[ \sum_{n=0}^\infty \frac{1}{n+1} = \sum_{n=1}^\infty \answer{\frac{1}{n}} \]

Series 2: \begin{hint} Inside each term, replace every occurrence of $n$ with $n+1$. \end{hint}
\[ \sum_{n=2}^\infty e^{-2n-1} =  \sum_{n=0}^\infty e^{\answer{-2n-5}} \]

Series 3: \begin{hint} Inside each term, replace every occurrence of $n$ with $n+1$. \end{hint}
\[ \sum_{n=3}^\infty (-1)^n \cos \frac{n^2 + n}{3} = \sum_{n=2}^\infty \answer{(-1)^{n+1}} \cos \answer{\frac{(n+1)^2+(n+1)}{3}} \]
\end{example}

\begin{example}
Suppose the series $\displaystyle \sum_{n=1}^\infty a_n$ and $\displaystyle \sum_{n=1}^\infty b_n$ both converge and the series $\displaystyle \sum_{n=1}^\infty c_n$ and $\displaystyle \sum_{n=1}^\infty d_n$ both diverge. Then
\begin{itemize}
\item The series $\displaystyle \sum_{n=1}^\infty (a_n + b_n)$ \wordChoice{\choice[correct]{converges}\choice{diverges}\choice{may converge or diverge}}.
\item The series $\displaystyle \sum_{n=1}^\infty (a_n + c_n)$ \wordChoice{\choice{converges}\choice[correct]{diverges}\choice{may converge or diverge}}.
\item The series $\displaystyle \sum_{n=1}^\infty (c_n + d_n)$ \wordChoice{\choice{converges}\choice{diverges}\choice[correct]{may converge or diverge}}.
\end{itemize}
\end{example}



\end{document}
