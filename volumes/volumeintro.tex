\documentclass{ximera}
\graphicspath{
{./}
{volumes/}
{arclengths/}
{centroids/}
{techniques/}
{applications/}
{series/}
{powerseries/}
{odes/}
{lessons/}
}
\usepackage{booktabs}

\newcommand{\bigmath}[1]{$\displaystyle #1$}
\newcommand{\choicebreak}{}
\newenvironment{type}{}{}
\newenvironment{notes}{}{}
\newenvironment{keywords}{}{}
\newcommand{\offline}{}
\newenvironment{comments}{\begin{feedback}}{\end{feedback}}
\newenvironment{multiplechoice}{\begin{multipleChoice}}{\end{multipleChoice}}
\title{Applications of Integration}
%%%%%\author{Philip T. Gressman}

\begin{document}
\begin{abstract}
  We study some important application of integrations: computing volumes of a variety of complicated three-dimensional objects, computing arc length and surface area, and finding centers of mass.
\end{abstract}
\maketitle

Integration is the tool to use whenever a quantity can be conceived as an \textit{accumulation of infinitesimal parts}. 
Volume is one of the most basic and important of such quantities. In the activities that follow, we regard volume as the accumulated size of infinitely thin slices and use this perspective to derive and apply a number of formulas for computing volume. Following this, we will examine other applications of integration to the computation of arc length, surface area, and centers of mass.




\end{document}
