\documentclass{ximera}
\graphicspath{
{./}
{volumes/}
{arclengths/}
{centroids/}
{techniques/}
{applications/}
{series/}
{powerseries/}
{odes/}
{lessons/}
}
\usepackage{booktabs}

\newcommand{\bigmath}[1]{$\displaystyle #1$}
\newcommand{\choicebreak}{}
\newenvironment{type}{}{}
\newenvironment{notes}{}{}
\newenvironment{keywords}{}{}
\newcommand{\offline}{}
\newenvironment{comments}{\begin{feedback}}{\end{feedback}}
\newenvironment{multiplechoice}{\begin{multipleChoice}}{\end{multipleChoice}}
\title{Exercises: Choose Your Method}
\author{Philip T. Gressman}

\begin{document}
\begin{abstract}
Exercises choosing a method for computing volume.
\end{abstract}
\maketitle

\section*{Sample Exam Questions}

\begin{question}%%%%%[2016.29]
The region in the plane bounded by $y = e^{-x/2}$ and the $x$-axis for $0 \leq x \leq \ln 2$ is rotated about the $x$-axis. The volume of the resulting solid of revolution is
\begin{multiplechoice}
\choice{\(\displaystyle \frac{2 \pi}{3}\)}
\choice{\(\displaystyle \frac{\pi}{3}\)}
\choice{\(\displaystyle \frac{3}{2}\)}
\choice{\(\displaystyle 2 \pi\)}
\choice{\(\displaystyle \frac{5}{3}\)}
\choice[correct]{\(\displaystyle \frac{\pi}{2}\)}
\end{multiplechoice}
\begin{feedback}
If $x$ is used as the slicing variable, then slices are vertical and consequently perpendicular to the axis of rotation. \begin{hint}
Furthermore one side of the region lies along the axis, so the disk method is appropriate in this case. 
\begin{hint}
The distance from the axis to the upper edge of the region is $e^{-x/2}$, so
\[ \begin{aligned} V & = \int_0^{\ln 2} \pi \left( e^{-x/2} \right)^2 dx = \pi \int_0^{\ln 2} e^{-x} dx \\ & =  \left. - \pi e^{-x} \right|_{x=0}^{\ln 2} = \pi (-e^{-\ln 2} + e^0) = \pi \left( - \frac{1}{2} + 1 \right) = \frac{\pi}{2}. \end{aligned}\]
\end{hint}
\end{hint}
\end{feedback}

\end{question}

\begin{question}%%%%%[2018.S.1]
Compute the volume of the solid of revolution obtained by rotating the region between $x=0$, $y=0$, and $y=\sqrt{2+3x^2 - 5x^4}$ around the $x$-axis.
\begin{multiplechoice} 
\choice{\(\displaystyle - \pi\)}
\choice{\(\displaystyle 0\)}
\choice{\(\displaystyle \pi\)}
\choice[correct]{\(\displaystyle 2 \pi\)}
\choice{\(\displaystyle 3 \pi\)}
\choice{\(\displaystyle 4 \pi\)}
\end{multiplechoice}

\end{question}

\begin{question}
 The region in the plane bounded on the right by the curve $x = 2 - y^2$, on the left by the curve $x = y^2$, and on the bottom by $y = 0$ is revolved around the $y$-axis. Compute the volume of the resulting solid.
\begin{multipleChoice}
\choice[correct]{$\displaystyle \frac{8 \pi}{3}$}
\choice{$\displaystyle \frac{9 \pi}{4}$}
\choice{$\displaystyle \frac{10 \pi}{7}$}
\choice{$\displaystyle \frac{11 \pi}{5}$}
\choice{$\displaystyle \frac{12 \pi}{11}$}
\choice{none of these}
\end{multipleChoice}
\end{question}

\begin{question}%%%%%[2015C.14]

The region between the graph of \(y = 1-x^2\) and the \(x\)-axis is rotated around the line \(y=1\). What is the volume of the resulting solid?
\begin{multiplechoice}
\choice{\(\displaystyle \frac{2 \pi}{5}\)}
\choice{\(\displaystyle \frac{4 \pi}{5}\)}
\choice{\(\displaystyle \frac{6 \pi}{5}\)}
\choice[correct]{\(\displaystyle \frac{8 \pi}{5}\)}
\choice{\(2 \pi\)}
\choice{\(\displaystyle \frac{12 \pi}{5}\)}
\end{multiplechoice}

\end{question}


\begin{question}%%%%%[2015C.15]

Calculate the volume of the solid obtained by rotating the area between the graphs of \(\displaystyle y = \frac{1}{\sqrt{x^2-1}}\) and the \(x\)-axis for \(1 < x < \sqrt{5}\) around the \(y\)-axis.
\begin{multiplechoice}
\choice{\(\pi\)}
\choice[correct]{\(4 \pi\)}
\choice{\(6 \pi\)}
\choice{\(8 \pi\)}
\choice{\(3 \pi\)}
\choice{\(2 \pi\)}
\end{multiplechoice}

\end{question}

\begin{question}%%%%%[2016C.14]

Let \(f(x)\) be a continuous function that satisfies \(f(0) = 0\) and \(f(x) > 0\) for \(x > 0\). For every \(b > 0\), when the region between the graph of \(y = f(x)\), the \(x\)-axis, and the line \(x=b\) is rotated around the \(x\)-axis, the volume of the resulting solid is \(18 \pi b^2\). What is \(f(x)\)?
\begin{multiplechoice}
\choice{\(\displaystyle 9x\)}
\choice{\(\displaystyle 3x^2\)}
\choice[correct]{\(\displaystyle 6 \sqrt{x}\)}
\choice{\(\displaystyle 27 x^{3/2}\)}
\choice{\(\displaystyle 9 x^2\)}
\choice{\(\displaystyle \sqrt{3x}\)}
\end{multiplechoice}

\end{question}

\begin{question}%%%%%[2017C.02]

Find the volume of the solid generated by revolving the region bounded above by \(y = \sec x\) and bounded below by \(y=0\) for \(0 \leq x \leq \pi/3\) about the \(x\)-axis.
\begin{multiplechoice}
\choice{\(\pi\)}
\choice{\(2 \pi\)}
\choice[correct]{\(\pi \sqrt{3}\)}
\choice{\(3 \pi\)}
\choice{\(4 \pi\)}
\choice{none of these}
\end{multiplechoice}

\end{question}




\end{document}
