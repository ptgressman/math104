\documentclass{ximera}
\graphicspath{
{./}
{volumes/}
{arclengths/}
{centroids/}
{techniques/}
{applications/}
{series/}
{powerseries/}
{odes/}
}

\newcommand{\bigmath}[1]{$\displaystyle #1$}
\newcommand{\choicebreak}{}
\newenvironment{type}{}{}
\newenvironment{notes}{}{}
\newenvironment{keywords}{}{}
\newcommand{\offline}{}
\newenvironment{comments}{\begin{feedback}}{\end{feedback}}
\newenvironment{multiplechoice}{\begin{multipleChoice}}{\end{multipleChoice}}
\title{Exercises: Arc Length}
\author{Philip T. Gressman}

\begin{document}
\begin{abstract}
We practice computing arc length.
\end{abstract}
\maketitle

\begin{exercise}%[APEX0704ARCL04]
Find the arc length of the function on the given interval: \(\displaystyle f(x) = \sqrt{8}x\) on \([-1, 1]\)
\[ L = \answer{6}. \]
%%
%%
\end{exercise}

\begin{exercise}
Find the arc length of the function on the given interval: \(\displaystyle f(x) = \ln \left(\cos x\right)\) on \([0, \pi/4]\).
(You may use the fact that $\int \sec x ~ dx = \ln |\sec x + \tan x| + C$.)
\[ L = \answer{\ln (1 + \sqrt{2})}. \]
\end{exercise}

\begin{exercise}%[APEX0704ARCL13]
Set up the integral to compute the arc length of the function on the given interval: \(\displaystyle f(x) = x^2\) on \([0, 1]\).
\[ L = \int_{\answer{0}}^{\answer{1}} \answer{\sqrt{1+4x^2}} \ dx\]
%
%
\end{exercise}

\begin{question}%%%%%[2017C.03]
Let \(\displaystyle y = \frac{x^4}{16} + \frac{1}{2x^2}\). Find the arc length for \(1 \leq x \leq \sqrt{2}\).
\[ L = \answer{\frac{7}{16}}. \]
\end{question}

\section*{Sample Quiz Questions}

\begin{question}%%%%%[Arclen01]

Compute the arc length of the curve \[y = \frac{3}{4}x^{-2} + \frac{1}{24}x^{4} - 1\] between the endpoints \(x = \sqrt{3}\) and \(x = \sqrt{6}\).
(Hints won't be revealed until after you choose a response.)
\begin{multiplechoice}
\choice{\(\displaystyle \frac{5}{12}\)}
\choice{\(\displaystyle \frac{7}{12}\)}
\choice{\(\displaystyle \frac{3}{4}\)}
\choice{\(\displaystyle \frac{11}{12}\)}
\choice{\(\displaystyle \frac{13}{12}\)}
\choice[correct]{\(\displaystyle \frac{5}{4}\)}
\end{multiplechoice}
\begin{feedback}
Applying the formula for arc length gives that
\[
\begin{aligned}
L & = \int_{\sqrt{3}}^{\sqrt{6}} \sqrt{1 + \left( \frac{dy}{dx} \right)^2}~dx 
   = \int_{\sqrt{3}}^{\sqrt{6}} \sqrt{1 + \left( -\frac{3}{2}x^{-3} + \frac{1}{6}x^{3} \right)^2}~dx
\end{aligned}
\]
\begin{hint}
\[
\begin{aligned}
  & = \int_{\sqrt{3}}^{\sqrt{6}} \sqrt{1 + \frac{9}{4}x^{-6} - \frac{1}{2} + \frac{1}{36}x^{6}} ~ dx 
   = \int_{\sqrt{3}}^{\sqrt{6}} \sqrt{ \left( \frac{3}{2}x^{-3} + \frac{1}{6}x^{3} \right)^2} ~ dx \\
  & = \int_{\sqrt{3}}^{\sqrt{6}} \left( \frac{3}{2}x^{-3} + \frac{1}{6}x^{3} \right) ~ dx 
   = \left.  \left( -\frac{3}{4}x^{-2} + \frac{1}{24}x^{4} \right) \right|_{\sqrt{3}}^{\sqrt{6}} \\
  & =  \left( -\frac{1}{8} + \frac{3}{2} \right) - \left( -\frac{1}{4} + \frac{3}{8} \right) 
   = \frac{5}{4}.
\end{aligned}
\]
Note that you must always take the positive square root in going from line two to line three. In particular, if you get a negative answer, you have likely taken the negative square root.
\end{hint}
\end{feedback}

\end{question}

\section*{Sample Exam Questions}

\begin{question}(2017 Midterm 1)
Compute the length of the curve $x = \frac{1}{8} (y^2 + 2y) - \ln (y+1)$ between $y=0$ and $y = 2$.
\begin{multipleChoice}
\choice[correct]{$\displaystyle 1 + \ln 3$}
\choice{$\displaystyle 2 + \ln 6$}
\choice{$\displaystyle 3 + \ln 9$}
\choice{$\displaystyle 4 + \ln 12$}
\choice{$\displaystyle 5 + \ln 15$}
\choice{none of the above}
\end{multipleChoice}
\end{question}

\begin{question}%%%%%[2018.S.4]
 Find the arc length of the following curve between $x=-1$ and $x=1$:
\[ y = 3 \cosh \frac{x}{3}. \]
(Note: $\cosh x = (e^x + e^{-x})/2$.)
\begin{multiplechoice}
\choice{\(\displaystyle \frac{e}{3} - \frac{1}{3e}\)}
\choice{\(\displaystyle \frac{e}{2} - \frac{1}{2e}\)}
\choice{\(\displaystyle e - \frac{1}{e}\)}
\choice{\(\displaystyle 2 e - \frac{2}{e}\)}
\choice[correct]{\(\displaystyle 3 e - \frac{3}{e}\)}
\choice{none of the above}
\end{multiplechoice}

\end{question}

\begin{question}
 A certain curve $y = f(x)$ in the plane has the property that its length between the endpoints $x=0$ and $x=a$ is equal to
\[ \int_0^a \sqrt{1 + \sin^2 t} ~ dt \]
for every value of $a > 0$.  Assuming the curve passes through the points $(0,0)$ and $\left(\frac{\pi}{2},1\right)$, what is $f\left( \frac{\pi}{4} \right)$?
\begin{multipleChoice} 
\choice{$\displaystyle \frac{1}{2}$}
\choice{$\displaystyle \frac{1}{\sqrt{2}}$}
\choice[correct]{$\displaystyle 1 - \frac{1}{\sqrt{2}}$}
\choice{$\displaystyle 0$}
\choice{$\displaystyle - \frac{1}{\sqrt{2}}$}
\choice{none of these}
\end{multipleChoice}
\end{question}

\begin{question}%%%%%[2015C.13]

Find the length of the part of the curve \(\displaystyle y = \frac{3}{16} e^{2x} + \frac{1}{3} e^{-2x}\) for \(0 \leq x \leq \ln 2\).
\begin{multiplechoice}
\choice[correct]{\(\displaystyle \frac{13}{16}\)}
\choice{\(\displaystyle \frac{11}{16}\)}
\choice{\(\displaystyle \frac{3}{8}\)}
\choice{\(\displaystyle \frac{9}{8}\)}
\choice{\(\displaystyle \frac{29}{64}\)}
\choice{\(\displaystyle \frac{3}{4}\)}
\end{multiplechoice}

\end{question}

\begin{question}%%%%%[2016C.01]

Find the length of the part of the curve \(\displaystyle y = \frac{x^4}{4} + \frac{1}{8x^2}\) for \(1 \leq x \leq 2\).
\begin{multiplechoice}
\choice[correct]{\(\displaystyle \frac{13}{16}\)}
\choice{\(\displaystyle \frac{11}{16}\)}
\choice{\(\displaystyle \frac{7}{8}\)}
\choice{\(\displaystyle \frac{13\sqrt{2}}{16}\)}
\choice{\(\displaystyle \frac{11\sqrt{2}}{16}\)}
\choice{\(\displaystyle \frac{7\sqrt{2}}{8}\)}
\end{multiplechoice}

\end{question}




\end{document}
