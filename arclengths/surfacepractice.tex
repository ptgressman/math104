\documentclass{ximera}
\graphicspath{
{./}
{volumes/}
{arclengths/}
{centroids/}
{techniques/}
{applications/}
{series/}
{powerseries/}
{odes/}
{lessons/}
}
\usepackage{booktabs}

\newcommand{\bigmath}[1]{$\displaystyle #1$}
\newcommand{\choicebreak}{}
\newenvironment{type}{}{}
\newenvironment{notes}{}{}
\newenvironment{keywords}{}{}
\newcommand{\offline}{}
\newenvironment{comments}{\begin{feedback}}{\end{feedback}}
\newenvironment{multiplechoice}{\begin{multipleChoice}}{\end{multipleChoice}}
\title{Exercises: Surface Area}
\author{Philip T. Gressman}

\begin{document}
\begin{abstract}
Various exercises related to the computation of areas of surfaces of revolution.
\end{abstract}
\maketitle


\section*{Sample Exam Questions}

\begin{question}
Give an integral formula for the area of the surface generated by revolving the curve $y = \ln x$ between $x=1$ and $x=2$ about the $y$-axis. Explain your answer. You do not need to evaluate the integral. 
\begin{multipleChoice}
\choice[correct]{\(\displaystyle \int_1^2 \frac{2 \pi x}{\sqrt{1 + (\ln x)^2}} ~ dx\)}
\choice{\(\displaystyle \int_1^2 \frac{2 \pi x}{\sqrt{1 + (\ln x)^2}} ~ dx\)}
\choice{\(\displaystyle \int_1^2 \frac{2 \pi x}{\sqrt{1 + (\ln x)^2}} ~ dx\)} 
\choice{\(\displaystyle \int_1^2 \frac{2 \pi x}{\sqrt{1 + (\ln x)^2}} ~ dx\)}
\choice{\(\displaystyle \int_1^2 \frac{2 \pi x}{\sqrt{1 + (\ln x)^2}} ~ dx\)}
\choice{\(\displaystyle \int_1^2 \frac{2 \pi x}{\sqrt{1 + (\ln x)^2}} ~ dx\)}
\end{multipleChoice}
\end{question}

\begin{question}
 The curve $y = \frac{x^2}{8}$ between $x=0$ and $x = 3$ is revolved around the $y$-axis. Compute the surface area of the resulting surface.
\begin{multipleChoice}
\choice{$\displaystyle \frac{31 \pi}{6}$}
\choice{$\displaystyle \frac{41 \pi}{6}$}
\choice[correct]{$\displaystyle \frac{61 \pi}{6}$}
\choice{$\displaystyle \frac{71 \pi}{6}$}
\choice{$\displaystyle \frac{91 \pi}{6}$}
\choice{none of the above}
\end{multipleChoice}
\end{question}


\end{document}
