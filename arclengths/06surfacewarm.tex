\documentclass{ximera}
\graphicspath{
{./}
{volumes/}
{arclengths/}
{centroids/}
{techniques/}
{applications/}
{series/}
{powerseries/}
{odes/}
}

\newcommand{\bigmath}[1]{$\displaystyle #1$}
\newcommand{\choicebreak}{}
\newenvironment{type}{}{}
\newenvironment{notes}{}{}
\newenvironment{keywords}{}{}
\newcommand{\offline}{}
\newenvironment{comments}{\begin{feedback}}{\end{feedback}}
\newenvironment{multiplechoice}{\begin{multipleChoice}}{\end{multipleChoice}}
\title{Surface Area}
\author{Philip T. Gressman}

\begin{document}
\begin{abstract}
We practice setting up integrals for the surface area of surfaces of revolution.
\end{abstract}
\maketitle

\section*{Online Texts}
\begin{itemize}
\item \href{https://openstax.org/books/calculus-volume-2/pages/2-4-arc-length-of-a-curve-and-surface-area}{OpenStax II 2.4: Surface Area}
\item \href{https://ximera.osu.edu/mooculus/calculus2/surfaceArea/titlePage}{Ximera OSU: Surface Area}
\item \href{https://www.whitman.edu/mathematics/calculus_online/section09.10.html}{Community Calculus 9.10: Surface Area}
\end{itemize}

\section*{Examples}

\begin{example}
Suppose the graph $y = 2\sqrt{x}$ between $x = 0$ and $x = 1$ is revolved around the $x$-axis. Compute the area of the resulting surface.
\begin{itemize}
\item The distance from the $x$-axis to the point $(x,2\sqrt{x})$ is $\answer{2 \sqrt{x}}$.
\item For this curve, the arc length element satisfies $ds = \sqrt{1 + \left( \frac{dy}{dx} \right)^2} dx = \answer{\sqrt{1 + \frac{1}{x}}} dx$
\[ \begin{aligned} A & = \int_{\answer{0}}^{\answer{1}} \answer{2\sqrt{x}} ds \\
& = \int_{\answer{0}}^{\answer{1}} \answer{2 \sqrt{x+1}} dx \\
&= \answer{\frac{4}{3} \left( 2\sqrt{2} - 1 \right)}.
\end{aligned} \]
\end{itemize}
\end{example}



\end{document}
