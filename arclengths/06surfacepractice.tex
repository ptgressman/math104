\documentclass{ximera}
\graphicspath{
{./}
{volumes/}
{arclengths/}
{centroids/}
{techniques/}
{applications/}
{series/}
{powerseries/}
{odes/}
}

\newcommand{\bigmath}[1]{$\displaystyle #1$}
\newcommand{\choicebreak}{}
\newenvironment{type}{}{}
\newenvironment{notes}{}{}
\newenvironment{keywords}{}{}
\newcommand{\offline}{}
\newenvironment{comments}{\begin{feedback}}{\end{feedback}}
\newenvironment{multiplechoice}{\begin{multipleChoice}}{\end{multipleChoice}}
\title{Exercises: Surface Area}
\author{Philip T. Gressman}

\begin{document}
\begin{abstract}
Various exercises related to the computation of areas of surfaces of revolution.
\end{abstract}
\maketitle


\begin{exercise}%[APEX0704ARCL29]
Find the surface area of the solid formed by revolving \(y=2x\) on \([0,1]\) about the \(x\)-axis.
\[ A = \answer{2\pi\sqrt{5}}. \]
%
%
\end{exercise}

\begin{exercise}%[APEX0704ARCL30]
Find the surface area of the solid formed by revolving \(y=x^2\) on \([0,1]\) about the \(y\)-axis.
\begin{hint}
To compute the integral, you will need to make a substitution like $u = 1 + 4x^2$ or something similar.
\end{hint}
\[ A = \answer{\frac{(5 \sqrt{5} - 1) \pi}{6}}. \]
%\(2\pi\int_0^1 x\sqrt{1+4x^2}\ dx = \pi/6(5\sqrt{5}-1)\)
%
%
\end{exercise}

\begin{exercise}%[APEX0704ARCL31]
Find the surface area of the solid formed by revolving \(y=x^3\) on \([0,1]\) about the \(x\)-axis.
\begin{hint}
To compute the integral, you will need to make a substitution like $u = 1 + 9 x^4$ or something similar.
\end{hint}
\[ L = \answer{\frac{(10 \sqrt{10} - 1) \pi}{27}}. \]
%\(2\pi\int_0^1 x^3\sqrt{1+9x^4}\ dx = \pi/27(10\sqrt{10}-1)\)
%
%
\end{exercise}

\section*{Sample Exam Questions}

\begin{question}%%%%%[2016.36]
Give an integral formula for the area of the surface generated by revolving the curve $y = \ln x$ between $x=1$ and $x=2$ about the $y$-axis. Explain your answer. You do not need to evaluate the integral. 
\begin{multiplechoice}
\choice[correct]{\(\displaystyle \int_1^2 2 \pi \sqrt{x^2+1} ~ dx\)}
\choice{\(\displaystyle \int_1^2 2 \pi (\ln x) \frac{\sqrt{x^2+1}}{x} ~ dx\)}
\choice{\(\displaystyle \int_1^2 \frac{2 \pi}{x} \sqrt{1 + (\ln x)^2} ~ dx\)} 
\choice{\(\displaystyle \int_1^2 \frac{1}{2 \pi \sqrt{x^2+1}} ~ dx\)}
\choice{\(\displaystyle \int_1^2 2 \pi (\ln x) \frac{x}{\sqrt{x^2+1}} ~ dx\)}
\choice{\(\displaystyle \int_1^2 \frac{2 \pi x}{\sqrt{1 + (\ln x)^2}} ~ dx\)}
\end{multiplechoice}
\end{question}


\begin{question}
 The curve $y = \frac{x^2}{8}$ between $x=0$ and $x = 3$ is revolved around the $y$-axis. Compute the surface area of the resulting surface.
\begin{multipleChoice}
\choice{$\displaystyle \frac{31 \pi}{6}$}
\choice{$\displaystyle \frac{41 \pi}{6}$}
\choice[correct]{$\displaystyle \frac{61 \pi}{6}$}
\choice{$\displaystyle \frac{71 \pi}{6}$}
\choice{$\displaystyle \frac{91 \pi}{6}$}
\choice{none of the above}
\end{multipleChoice}
\end{question}


\end{document}
