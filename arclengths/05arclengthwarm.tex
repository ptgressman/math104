\documentclass{ximera}
\graphicspath{
{./}
{volumes/}
{arclengths/}
{centroids/}
{techniques/}
{applications/}
{series/}
{powerseries/}
{odes/}
{lessons/}
}
\usepackage{booktabs}

\newcommand{\bigmath}[1]{$\displaystyle #1$}
\newcommand{\choicebreak}{}
\newenvironment{type}{}{}
\newenvironment{notes}{}{}
\newenvironment{keywords}{}{}
\newcommand{\offline}{}
\newenvironment{comments}{\begin{feedback}}{\end{feedback}}
\newenvironment{multiplechoice}{\begin{multipleChoice}}{\end{multipleChoice}}
\title{Arc Length}
\author{Philip T. Gressman}

\begin{document}
\begin{abstract}
We practice setting up and executing arc length calculations.
\end{abstract}
\maketitle

\section*{(Video) Calculus: Single Variable}
\youtube{fT1LVQdF-70}

\section*{Online Texts}
\begin{itemize}
\item \href{https://openstax.org/books/calculus-volume-2/pages/2-4-arc-length-of-a-curve-and-surface-area}{OpenStax II 2.4 Arc Length}
\item \href{https://ximera.osu.edu/mooculus/calculus2/lengthOfCurves/titlePage}{Ximera OSU: Arc Length}
\item \href{https://www.whitman.edu/mathematics/calculus_online/section09.09.html}{Community Calculus: Arc Length}
\end{itemize}


\section*{Examples}

\begin{example}%[Arclen03]
Compute the arc length of the curve $$x = -\frac{1}{8}y^{-2} - \frac{1}{4}y^{4} - 2$$ between the endpoints $y = 1/\sqrt{2}$ and $y = 1$.
\begin{itemize}
\item First we compute the derivative: \[\frac{dx}{dy} = \answer{\frac{1}{4} y^{-3} - y^3}. \]
\item Next, we write the arc length element:
\[ ds = \sqrt{ 1 + \left(\frac{dx}{dy}\right)^2} dy = \sqrt{1 + \left( \answer{ \frac{1}{4} y^{-3} - y^3} \right)^2} dy. \]
\item The key to integrating is to first fully simplify the quantity inside the square root:
\[ \begin{aligned}
1 + \left( \answer{\frac{1}{4} y^{-3} - y^3} \right)^2 = & 1 + \frac{1}{16} y^{-6} + 2 \left( \frac{1}{4} y^{-3} \right) \left( - y^3 \right) + y^6 \\
& = 1 + \frac{1}{16} y^{-6} - \frac{1}{2} + y^6
\end{aligned} \]
\begin{observation}
When the middle term in a binomial (i.e., FOIL) expansion is $-1/2$ adding one \textbf{always} gives a perfect square. In this case, the algebraic fact is
\[  1 + \frac{1}{16} y^{-6} - \frac{1}{2} + y^6 =   \frac{1}{16} y^{-6} + \frac{1}{2} + y^6 = \left( \answer{\frac{1}{4} y^{-3} + y^3} \right)^2. \]
The general phenomenon is that what was formerly the square of a difference---in this case, $\left( \frac{1}{4} y^{-3} - y^3 \right)^2$---becomes the square of the corresponding sum.
\end{observation} 
Therefore
$$
\begin{aligned}
L & = \int_{1/\sqrt{2}}^{1} \sqrt{1 + \left( \frac{dx}{dy} \right)^2}~dy 
   = \int_{1/\sqrt{2}}^{1} \sqrt{1 + \left( \answer{\frac{1}{4}y^{-3} - y^{3}} \right)^2}~dy \\
  & = \int_{1/\sqrt{2}}^{1} \sqrt{1 + \frac{1}{16}y^{-6} - \frac{1}{2} + y^{6}} ~ dy 
   = \int_{1/\sqrt{2}}^{1} \sqrt{ \left( \answer{\frac{1}{4}y^{-3} + y^{3}} \right)^2} ~ dy \\
  & = \int_{1/\sqrt{2}}^{1} \answer{\frac{1}{4}y^{-3} + y^{3}} ~ dy 
   = \left.  \left( \answer{-\frac{1}{8}y^{-2} + \frac{1}{4}y^{4}} \right) \right|_{1/\sqrt{2}}^{1} \\
  & =  \left( -\frac{1}{8} + \frac{1}{4} \right) - \left( -\frac{1}{4} + \frac{1}{16} \right) 
   = \frac{5}{16}.
\end{aligned}
$$
\end{itemize}
\end{example}





\end{document}
