\documentclass{ximera}
\graphicspath{
{./}
{volumes/}
{arclengths/}
{centroids/}
{techniques/}
{applications/}
{series/}
{powerseries/}
{odes/}
}

\newcommand{\bigmath}[1]{$\displaystyle #1$}
\newcommand{\choicebreak}{}
\newenvironment{type}{}{}
\newenvironment{notes}{}{}
\newenvironment{keywords}{}{}
\newcommand{\offline}{}
\newenvironment{comments}{\begin{feedback}}{\end{feedback}}
\newenvironment{multiplechoice}{\begin{multipleChoice}}{\end{multipleChoice}}
\title{Exercises: Taylor Series Applications}
%%%%%\author{Philip T. Gressman}

\begin{document}
\begin{abstract}
Various exercises relating to the application of Taylor Series to other problems of interest.
\end{abstract}
\maketitle

\begin{exercise}
Compute the limit 
\[ \lim_{n \rightarrow \infty} n^3 \left[ \arctan \frac{1}{n} - \frac{1}{n} \right]. \]
\begin{itemize}
\item The first few terms of the Maclaurin series for $\arctan x$ are
\[ \cos x = \answer{x} - \answer{\frac{x^3}{3}} + \answer{\frac{x^5}{5}} + \cdots \]
\item Plug in $x = \frac{1}{n}$ and subtract $1/n$:
\[ \arctan \frac{1}{n} - \frac{1}{n} = - \frac{1}{3n^3} + \answer{\frac{1}{5n^5}} + \cdots \]
\item Multiply by $n^3$, neglect all but the dominant term, and conclude
\[ \lim_{n \rightarrow \infty} n^3 \left[ \arctan \frac{1}{n} - \frac{1}{n} \right] = \answer{-\frac{1}{3}} \]
\end{itemize}
\end{exercise}

\begin{exercise}
Compute the limit
\[ \lim_{n \rightarrow \infty} \frac{\ln (n+3) - \ln n}{\ln (n+1) - \ln n}. \]
\begin{itemize}
\item First combine logarithms to simplify a bit:
\[  \ln (n+3) - \ln n  =  \ln \answer{1 + \frac{3}{n}} \ \ \text{ and } \ \ \ln (n+1) - \ln n  =  \ln \answer{1 + \frac{1}{n}}. \]
\item Write out the first few terms of the Maclaurin series for $\ln (1+x)$ and substitute in the appropriate expression involving $n$:
\[ \ln (1 + x) = \answer{x} - \answer{\frac{x^2}{2}} + \answer{\frac{x^3}{3}} - \cdots \]
\[ \ln (n+3) - \ln n = \answer{ \frac{3}{n}} - \answer{\frac{9}{2n^2}} + \answer{\frac{9}{n^3}} - \cdots \]
\[ \ln (n+1) - \ln n = \answer{\frac{1}{n}} - \answer{\frac{1}{2n^2}} + \answer{\frac{1}{3n^3}} - \cdots \]
\item Multiply by $n$, neglect all but the dominant term (as $n \rightarrow \infty)$ to conclude
\[ \lim_{n \rightarrow \infty} \frac{\ln (n+3) - \ln n}{\ln (n+1) - \ln n} = \answer{3}. \]
\end{itemize}
\end{exercise}

\begin{exercise}
Compute the limit
\[ \lim_{x \rightarrow 1}  \frac{\ln x - x + 1}{\sin \pi x}. \]
\begin{itemize}
\item Compute the first few terms of the Taylor series of numerator and denominator centered at $x = 1$:
\[ \ln x - x + 1 = \answer{0} + \answer{0} (x-1) + \answer{-\frac{1}{2}} (x-1)^2 + \answer{\frac{1}{3}} (x-1)^3 + \cdots \]
\[ \sin \pi x = \answer{0} + \answer{-\pi}(x-1) + \answer{0} (x-1)^2 + \answer{\frac{\pi^3}{6}}(x-1)^3 + \cdots \]
\item Neglect all but the dominant terms in numerator and denominator. Note that dominant here means as $x \rightarrow 1$. Then take the limit $x \rightarrow 1$:
\[ \lim_{x \rightarrow 1}  \frac{\ln x - x + 1}{\sin \pi x} = \answer{0}. \]
\end{itemize}
\end{exercise}


\begin{exercise}
Use the remainder formula for Taylor series to determine which partial sums of the series 
\[ \sum_{n=0}^\infty \frac{1}{n!} \]
differ from $e$ by at most $1/100$?
\begin{itemize}
\item The remainder formula says that
\[ e^1 - \sum_{n=0}^{N} \frac{1}{n!} = \frac{e^{\xi}}{(N+1)!} (1-0)^{N+1} \]
for some point $\xi$ which depends on $N$ and is somewhere in the interval $(\answer{0},\answer{1})$. The upper bound for the magnitude of $e^\xi$ on this interval is $M = \answer{e}$.
\item For which values of $N$ do we have
\[  \frac{M}{(N+1)!} \leq \frac{1}{50}? \]
(Check manually for small values of $N$; use the fact that $e$ is between $2$ and $3$.)
\[ N \geq \answer{4}. \]
\end{itemize}
\end{exercise}



\begin{exercise}
In this exercise, we will investigate two different ways of numerically approximating the value of $\ln 2$ using infinite series.
\begin{itemize}
\item The Maclaurin series of the function $\ln 1 + x$ converges conditionally at $x = 1$ to $\ln 2$. Compute the Maclaurin series: 
\[ \ln (1 + x) = \sum_{n=0}^\infty \answer{\frac{(-1)^n}{n+1}} x^{n+1} \]
(Reindex your answer to match the template above if your answer doesn't work as-is.)
\item  What degree Taylor polynomial would you need to use to approximate the value of $\ln 2$ to an error strictly less than $10^{-3}$? Answer: Taylor polynomial used should have degree $N = \answer{1000}$ or greater.
\item A similar but distinct strategy would be to compute $\ln (1/2)$ instead because we know $- \ln (1/2)  = \ln 2$. Evaluating the Maclaurin series at $x=-1/2$ and doing a little simplification, we see from the above series that
\[ \ln 2 = \sum_{n=1}^\infty \answer{\frac{1}{n 2^n}}. \]
(Once again, reindex if your answer does not already start at $n=1$.)
\item The presence of a factor exponential in $n$ suggests that this second series converges to $\ln 2$ \wordChoice{\choice{much slower}\choice{slightly slower}\choice{slightly faster}\choice[correct]{much faster}} than the first series. Of the two series, then, the \wordChoice{\choice{first}\choice[correct]{second}} series presents a more efficient way to compute $\ln 2$ numerically.
\end{itemize}
\begin{hint}
For the first estimation, use the series estimation features of the Alternating Series Test.
\begin{hint}
For an alternating series, the error of approximating the series by a partial sum is never greater than the magnitude of the \textit{first term omitted}.
\end{hint}
\end{hint}
\end{exercise}

\begin{exercise}
Use the remainder formula for Taylor series to determine which partial sums of the series
\[ \sum_{n=1}^\infty \frac{1}{n 3^n} \]
differ from $\ln (3/2)$ by at most $10^{-2}$?
\begin{itemize}
\item Guided by the example above, we will use the Maclaurin series 
\[ - \ln (1-x) = \sum_{n=1}^\infty \frac{x^n}{n}. \]
First, compute the derivatives of $- \ln (1-x)$. Find a pattern which holds for all $N$:
\[ \frac{d^{N+1}}{dx^{N+1}} \left( - \ln (1 - x) \right)  = \answer{ \frac{N!}{(1-x)^{N+1}}}. \]
When $x$ is between $0$ and $1/3$, the largest value of the $(N+1)$-st derivative is what?
\[ \left| \frac{d^{N+1}}{dx^{N+1}} \left( - \ln (1 - x) \right) \right| \leq \answer{N! 3^{N+1}/2^{N+1}}. \]
\item Using this upper bound for the $(N+1)$-st derivative, we know that
\[ \left| \ln \frac{3}{2} - \sum_{n=0}^N \frac{1}{n 3^n} \right| \leq \answer{ \frac{1}{(N+1) 2^{N+1}}} \]
(use the upper bound and the remainder formula.)
\item For which values of $N$ is the expression you just found less than $1/100$? Check by hand for smallish values of $N$ to find the smallest one which works.
\[ N \geq \answer{4}. \]
\end{itemize}
\end{exercise}



\section*{Sample Quiz Questions}



\begin{question}%%%%%[TAp0]

Use Taylor series to estimate the value of
\[\sqrt[3]{\frac{11}{10}}\] to within an error of at most \(1/900\).
(Hints will not be revealed until you choose a response.)
\begin{multiplechoice}
\choice[correct]{\(\displaystyle \frac{31}{30}\)}
\choice{\(\displaystyle \frac{47}{45}\)}
\choice{\(\displaystyle \frac{19}{18}\)}
\choice{\(\displaystyle \frac{16}{15}\)}
\choice{\(\displaystyle \frac{97}{90}\)}
\choice{\(\displaystyle \frac{49}{45}\)}
\end{multiplechoice}
\begin{feedback}
We may use the remainder formula for Taylor series to approach this problem.
Suppose \(p_n(x)\) is the degree \(n\) Taylor polynomial of the function
\[ f(x) = \sqrt[3]{1+x}\]
with center \(a=0\). Then the error \(E_n(x)\), i.e., the difference between the polynomial and the function, does not exceed \(\frac{f^{(n+1)}(\xi)}{(n+1)!}x^{n+1}\), where \(\xi\) is some unknown point  in the range \(0 \leq \xi \leq x\). \begin{hint} In this case one should take \(x = 1/10\) and determine how many derivatives are required to make this error estimate less than the given threshold. \begin{hint} This means checking by hand for small numbers of derivatives. For the specific problem at hand, if we approximate \(f(x)\) by the Taylor polynomial of degree \( n = 1\), we have 
\[ \left| \frac{x^{n+1} }{(n+1)!} f^{(n+1)}(\xi) \right| = \left| \frac{x^{n+1}}{(n+1)!} \left( -\frac{2}{9} (\xi+1)^{-5/3} \right) \right| \leq \left| \frac{x^{n+1}}{(n+1)!} \left( \frac{2}{9} \right) \right| = \frac{1}{900} \] when \(x = 1/10\). \begin{hint}
We conclude that the correct Taylor approximation is
\[ p_n \left( \frac{1}{10}\right) =  \left(\frac{1}{10}\right)^{0} + \frac{1}{3}\left(\frac{1}{10}\right)^{1} =  1 + \frac{1}{30} = \frac{31}{30}.\]
\end{hint} \end{hint} \end{hint}
\end{feedback}

\end{question}

\begin{question}%%%%%[TAp1]

Use Taylor series to estimate the value of
\[e^{-\frac{1}{3}}\] to within an error of at most \(1/162\).
\begin{multiplechoice}
\choice{\(\displaystyle \frac{5}{9}\)}
\choice[correct]{\(\displaystyle \frac{13}{18}\)}
\choice{\(\displaystyle \frac{8}{9}\)}
\choice{\(\displaystyle \frac{19}{18}\)}
\choice{\(\displaystyle \frac{11}{9}\)}
\choice{\(\displaystyle \frac{25}{18}\)}
\end{multiplechoice}
\begin{feedback}
We may use the remainder formula for Taylor series to approach this problem.
Suppose \(p_n(x)\) is the degree \(n\) Taylor polynomial of the function
\[ f(x) = e^{-x}\]
with center \(a=0\). Then the error \(E_n(x)\), i.e., the difference between the polynomial and the function, does not exceed \(\frac{f^{(n+1)}(\xi)}{(n+1)!}x^{n+1}\), where \(\xi\) is some unknown point  in the range \(0 \leq \xi \leq x\). \begin{hint}  In this case one should take \(x = 1/3\) and determine how many derivatives are required to make this error estimate less than the given threshold. \begin{hint} This means checking by hand for small numbers of derivatives. For the specific problem at hand, if we approximate \(f(x)\) by the Taylor polynomial of degree \( n = 2\), we have 
\[ \left| \frac{x^{n+1} }{(n+1)!} f^{(n+1)}(\xi) \right| = \left| \frac{x^{n+1}}{(n+1)!} \left( e^{-\xi} \right) \right| \leq \left| \frac{x^{n+1}}{(n+1)!} \left( 1 \right) \right| = \frac{1}{162} \] when \(x = 1/3\). \begin{hint}
We conclude that the correct Taylor approximation is
\[ p_n \left( \frac{1}{3}\right) =  \left(\frac{1}{3}\right)^{0} -1\left(\frac{1}{3}\right)^{1} + \frac{1}{2}\left(\frac{1}{3}\right)^{2} =  1 -\frac{1}{3} + \frac{1}{18} = \frac{13}{18}.\] \end{hint} \end{hint} \end{hint}
\end{feedback}

\end{question}


\end{document}
