\documentclass{ximera}
\graphicspath{
{./}
{volumes/}
{arclengths/}
{centroids/}
{techniques/}
{applications/}
{series/}
{powerseries/}
{odes/}
}

\newcommand{\bigmath}[1]{$\displaystyle #1$}
\newcommand{\choicebreak}{}
\newenvironment{type}{}{}
\newenvironment{notes}{}{}
\newenvironment{keywords}{}{}
\newcommand{\offline}{}
\newenvironment{comments}{\begin{feedback}}{\end{feedback}}
\newenvironment{multiplechoice}{\begin{multipleChoice}}{\end{multipleChoice}}
\title{Exercises: Taylor Series Applications}
%%%%%\author{Philip T. Gressman}

\begin{document}
\begin{abstract}
Various exercises relating to the application of Taylor Series to other problems of interest.
\end{abstract}
\maketitle



\begin{exercise}
In this exercise, we will investigate two different ways of numerically approximating the value of $\ln 2$ using infinite series.

\begin{itemize}
\item The Maclaurin series of the function $\ln 1 + x$ converges conditionally at $x = 1$ to $\ln 2$. Compute the Maclaurin series: 
\[ \ln (1 + x) = \sum_{n=0}^\infty \answer{\frac{(-1)^n}{n+1}} x^{n+1} \]
(Reindex your answer to match the template above if your answer doesn't work as-is.)
\item  What degree Taylor polynomial would you need to use to approximate the value of $\ln 2$ to an error strictly less than $10^{-3}$? Answer: Taylor polynomial used should have degree $N = \answer{1000}$ or greater.
\item A similar but distinct strategy would be to compute $\ln (1/2)$ instead because we know $- \ln (1/2)  = \ln 2$. Evaluating the Maclaurin series at $x=-1/2$ and doing a little simplification, we see from the above series that
\[ \ln 2 = \sum_{n=1}^\infty \answer{\frac{1}{n 2^n}}. \]
(Once again, reindex if your answer does not already start at $n=1$.)
\item The presence of a factor exponential in $n$ suggests that this second series converges to $\ln 2$ \wordChoice{\choice{much slower}\choice{slightly slower}\choice{slightly faster}\choice[correct]{much faster}} than the first series. Of the two series, then, the \wordChoice{\choice{first}\choice[correct]{second}} series presents a more efficient way to compute $\ln 2$ numerically.
\end{itemize}
\begin{hint}
For the first estimation, use the series estimation features of the Alternating Series Test.
\begin{hint}
For an alternating series, the error of approximating the series by a partial sum is never greater than the magnitude of the \textit{first term omitted}.
\end{hint}
\end{hint}
\end{exercise}




\section*{Sample Quiz Questions}



\begin{question}%%%%%[TAp0]

Use Taylor series to estimate the value of
\[\sqrt[3]{\frac{11}{10}}\] to within an error of at most \(1/900\).
(Hints will not be revealed until you choose a response.)
\begin{multiplechoice}
\choice[correct]{\(\displaystyle \frac{31}{30}\)}
\choice{\(\displaystyle \frac{47}{45}\)}
\choice{\(\displaystyle \frac{19}{18}\)}
\choice{\(\displaystyle \frac{16}{15}\)}
\choice{\(\displaystyle \frac{97}{90}\)}
\choice{\(\displaystyle \frac{49}{45}\)}
\end{multiplechoice}
\begin{feedback}
We may use the remainder formula for Taylor series to approach this problem.
Suppose \(p_n(x)\) is the degree \(n\) Taylor polynomial of the function
\[ f(x) = \sqrt[3]{1+x}\]
with center \(a=0\). Then the error \(E_n(x)\), i.e., the difference between the polynomial and the function, does not exceed \(\frac{f^{(n+1)}(\xi)}{(n+1)!}x^{n+1}\), where \(\xi\) is some unknown point  in the range \(0 \leq \xi \leq x\). \begin{hint} In this case one should take \(x = 1/10\) and determine how many derivatives are required to make this error estimate less than the given threshold. \begin{hint} This means checking by hand for small numbers of derivatives. For the specific problem at hand, if we approximate \(f(x)\) by the Taylor polynomial of degree \( n = 1\), we have 
\[ \left| \frac{x^{n+1} }{(n+1)!} f^{(n+1)}(\xi) \right| = \left| \frac{x^{n+1}}{(n+1)!} \left( -\frac{2}{9} (\xi+1)^{-5/3} \right) \right| \leq \left| \frac{x^{n+1}}{(n+1)!} \left( \frac{2}{9} \right) \right| = \frac{1}{900} \] when \(x = 1/10\). \begin{hint}
We conclude that the correct Taylor approximation is
\[ p_n \left( \frac{1}{10}\right) =  \left(\frac{1}{10}\right)^{0} + \frac{1}{3}\left(\frac{1}{10}\right)^{1} =  1 + \frac{1}{30} = \frac{31}{30}.\]
\end{hint} \end{hint} \end{hint}
\end{feedback}

\end{question}

\begin{question}%%%%%[TAp1]

Use Taylor series to estimate the value of
\[e^{-\frac{1}{3}}\] to within an error of at most \(1/162\).
\begin{multiplechoice}
\choice{\(\displaystyle \frac{5}{9}\)}
\choice[correct]{\(\displaystyle \frac{13}{18}\)}
\choice{\(\displaystyle \frac{8}{9}\)}
\choice{\(\displaystyle \frac{19}{18}\)}
\choice{\(\displaystyle \frac{11}{9}\)}
\choice{\(\displaystyle \frac{25}{18}\)}
\end{multiplechoice}
\begin{feedback}
We may use the remainder formula for Taylor series to approach this problem.
Suppose \(p_n(x)\) is the degree \(n\) Taylor polynomial of the function
\[ f(x) = e^{-x}\]
with center \(a=0\). Then the error \(E_n(x)\), i.e., the difference between the polynomial and the function, does not exceed \(\frac{f^{(n+1)}(\xi)}{(n+1)!}x^{n+1}\), where \(\xi\) is some unknown point  in the range \(0 \leq \xi \leq x\). \begin{hint}  In this case one should take \(x = 1/3\) and determine how many derivatives are required to make this error estimate less than the given threshold. \begin{hint} This means checking by hand for small numbers of derivatives. For the specific problem at hand, if we approximate \(f(x)\) by the Taylor polynomial of degree \( n = 2\), we have 
\[ \left| \frac{x^{n+1} }{(n+1)!} f^{(n+1)}(\xi) \right| = \left| \frac{x^{n+1}}{(n+1)!} \left( e^{-\xi} \right) \right| \leq \left| \frac{x^{n+1}}{(n+1)!} \left( 1 \right) \right| = \frac{1}{162} \] when \(x = 1/3\). \begin{hint}
We conclude that the correct Taylor approximation is
\[ p_n \left( \frac{1}{3}\right) =  \left(\frac{1}{3}\right)^{0} -1\left(\frac{1}{3}\right)^{1} + \frac{1}{2}\left(\frac{1}{3}\right)^{2} =  1 -\frac{1}{3} + \frac{1}{18} = \frac{13}{18}.\] \end{hint} \end{hint} \end{hint}
\end{feedback}

\end{question}


\end{document}
