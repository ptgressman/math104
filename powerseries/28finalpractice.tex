\documentclass{ximera}
\graphicspath{
{./}
{volumes/}
{arclengths/}
{centroids/}
{techniques/}
{applications/}
{series/}
{powerseries/}
{odes/}
}

\newcommand{\bigmath}[1]{$\displaystyle #1$}
\newcommand{\choicebreak}{}
\newenvironment{type}{}{}
\newenvironment{notes}{}{}
\newenvironment{keywords}{}{}
\newcommand{\offline}{}
\newenvironment{comments}{\begin{feedback}}{\end{feedback}}
\newenvironment{multiplechoice}{\begin{multipleChoice}}{\end{multipleChoice}}
\title{Exercises: Cumulative}
%%%%%\author{Philip T. Gressman}

\begin{document}
\begin{abstract}
Exercises relating to various topics we have studied.
\end{abstract}
\maketitle


\begin{exercise}
The series
\[ \sum_{n=1}^\infty \left( e^{ \frac{1}{n}} - 1 \right) \]
is \wordChoice{\choice{convergent}\choice[correct]{divergent}} by limit comparison to the $p$-series
\[ \sum_{n=1}^\infty \frac{1}{n^p} \]
with $p = \answer{1}$.  Likewise, the
series
\[ \sum_{n=1}^\infty \left( e^{ \frac{1}{n}} - 1 - \frac{1}{n} \right) \]
is \wordChoice{\choice[correct]{convergent}\choice{divergent}} by limit comparison to the $p$-series
with $p = \answer{2}$.
\begin{hint}
Use a Maclaurin series to determine the dominant behavior of 
\[ e^{\frac{1}{n}} - 1 \]
as $n \rightarrow \infty$.
\end{hint}
\end{exercise}

\begin{exercise}
Fill in the blank below with an appropriate constant to make the series absolutely convergent:
\[ \sum_{n=1}^\infty \left[ \ln \left( 1 + \frac{1}{n} \right) - \frac{\answer{1}}{n} \right] \]
\begin{hint}
Use the Maclaurin series for $\ln (1 + x)$ and substitute $x = 1/n$.
\end{hint}
\end{exercise}

\begin{exercise}
Compute the sum of the series below.
\[ \sum_{n=0}^\infty \frac{(-1)^n}{2n+1} \frac{1}{3^n} \]
\begin{itemize}
\item We know the series
\[ \sum_{n=0}^\infty \frac{(-1)^n x^{2n+1}}{2n+1} = \answer{\arctan x} \]
for $-1 < x < 1$. 
\item This means
\[ \sum_{n=0}^\infty \frac{(-1)^n x^{2n}}{2n+1} = \answer{\frac{\arctan x}{x}}. \]
\item We conclude
\[  \sum_{n=0}^\infty \frac{(-1)^n}{2n+1} \frac{1}{3^n} = \answer{\frac{\pi \sqrt{3}}{6}}. \]
\end{itemize}
\end{exercise}



\begin{exercise}
Compute the sum of the series below.
\[ \sum_{n=0}^\infty n 3^{-n} = \answer{\frac{3}{4}}. \]
\begin{hint}
\[ \frac{1}{1-x} = \sum_{n=0}^\infty x^n \ \ \text{ when } -1 < x < 1. \]
\end{hint}
\begin{hint}
\[ \frac{1}{(1-x)^2} = \sum_{n=0}^\infty n x^{n-1} \ \ \text{ when } -1 < x < 1. \]
\end{hint}
\end{exercise}

\begin{exercise}
Compute the sum of the series below.
\[ \sum_{n=1}^\infty \frac{3}{n(n+1)} = \answer{3}. \]
\begin{hint}
You don't need Taylor series for this one.
\end{hint}
\begin{hint}
It's a telescoping series
\end{hint}
\end{exercise}

\section*{Sample Exam Questions}

\begin{question}%%%%%[2015C.02]

If it converges, find the sum of the series \(\displaystyle \sum_{n=0}^\infty \frac{(-1)^n \pi^{2n}}{3^{2n} (2n)!}\). \offline{If the series diverges, explain why.}
\begin{multiplechoice}
\choice{\(\ln 2\)}
\choice{\(\ln 3 - \ln 2\)}
\choice{\(e^{-2}\)}
\choice[correct]{\(\displaystyle \frac{1}{2}\)}
\choice{\(\displaystyle \frac{2}{e}\)}
\choice{diverges}
\end{multiplechoice}
\begin{feedback}
We recognize the Taylor series for cosine:
\[ \cos x = \sum_{n=0}^\infty \frac{(-1)^n \pi^{2n}}{3^{2n}(2n)!} \]
The series in question is exactly
\[ \sum_{n=0}^\infty \frac{(-1)^n (\pi/3)^{2n}}{(2n)!} = \cos \frac{\pi}{3} = \frac{1}{2}. \]
\end{feedback}

\end{question}

\begin{question}%%%%%[2015C.05]

What is the limit of the sequence \(\displaystyle \left\{ n^2 \left( 1 - \cos \frac{1}{n} \right) \right\}\)?
\begin{multiplechoice}
\choice{\(1\)}
\choice{\(-1\)}
\choice{\(\displaystyle \frac{\sqrt{3}}{2}\)}
\choice[correct]{\(\displaystyle \frac{1}{2}\)}
\choice{\(\displaystyle -\frac{\sqrt{3}}{2}\)}
\choice{diverges}
\end{multiplechoice}

\end{question}

\begin{question}%%%%%[2017C.10]

Find the limit of the sequence
\[ a_n = \left\{ n \left[ \ln (n+3) - \ln n \right] \right\}. \]
\begin{multiplechoice}
\choice{\(0\)}
\choice{\(1\)}
\choice{\(\ln 3\)}
\choice[correct]{\(3\)}
\choice{\(\infty\)}
\choice{the limit does not exist}
\end{multiplechoice}

\end{question}


\end{document}
