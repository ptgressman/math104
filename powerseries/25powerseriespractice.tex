\documentclass{ximera}
\graphicspath{
{./}
{volumes/}
{arclengths/}
{centroids/}
{techniques/}
{applications/}
{series/}
{powerseries/}
{odes/}
{lessons/}
}
\usepackage{booktabs}

\newcommand{\bigmath}[1]{$\displaystyle #1$}
\newcommand{\choicebreak}{}
\newenvironment{type}{}{}
\newenvironment{notes}{}{}
\newenvironment{keywords}{}{}
\newcommand{\offline}{}
\newenvironment{comments}{\begin{feedback}}{\end{feedback}}
\newenvironment{multiplechoice}{\begin{multipleChoice}}{\end{multipleChoice}}
\title{Exercises: Power Series and Convergence}
%%%%%\author{Philip T. Gressman}

\begin{document}
\begin{abstract}
Exercises relating to power series and their convergence properties.
\end{abstract}
\maketitle

\begin{exercise}
Compute the radius of convergence of the power series below.
\[ \sum_{n = 1}^\infty \frac{(x-3)^n}{2^n + 1} \]
\[ R = \answer{2} \]
\begin{hint}
When computing a limit like
\[ \lim_{n \rightarrow \infty} \frac{2^{n} + 1}{2^{n+1} + 1}, \]
remember the effect of orders of growth: $1 <\!\!< 2^n$, and so from the perspective of the limit, the $+1$'s in both the numerator and denominator are both negligible.
\end{hint}
\end{exercise}

\begin{exercise}
Compute the radius of convergence of the power series below.
\[ \sum_{n = 1}^\infty \frac{(-3)^n (\ln n) (x-3)^n}{4^n} \]
\[ R = \answer{\frac{4}{3}} \]
\begin{hint}
When computing a limit like
\[ \lim_{n \rightarrow \infty} (\ln n)^{1/n}, \]
remember that we already know $n^{1/n} \rightarrow 1$ and that $\ln n <\!\!<n$ as well, so we expect $(\ln n)^{1/n} \rightarrow 1$ as well.
\end{hint}
\end{exercise}

\begin{exercise}
Compute the radius of convergence of the power series below.
\[ \sum_{n = 1}^\infty \frac{(-2)^n n! (x-3)^n}{3^n+n^2} \]
\[ R = \answer{0} \]
\end{exercise}

\begin{exercise}
Compute the interval of convergence for the power series below.
\[ \sum_{n=1}^\infty \frac{(-2)^n (x-3)^n}{\ln n} \]
The left endpoint of the interval is $\answer{5/2}$; it \wordChoice{\choice{is}\choice[correct]{is not}} included in the interval of convergence.
The right endpoint of the interval is $\answer{7/2}$; it is \wordChoice{\choice[correct]{is}\choice{is not}} included in the interval of convergence.
\end{exercise}

\begin{exercise}
Reindex the series below:
\[ \sum_{n=1}^\infty \frac{x^{2n+1}}{n + 2} = \sum_{n = 0} \frac{x^{\answer{2n+3}}}{\answer{n+3}} \]
\end{exercise}

\begin{exercise}
Differentiate the series below term-by-term:
\[ \frac{d}{dx} \sum_{n=0}^\infty \frac{n}{n+1} x^n = \sum_{n=1} \answer{\frac{n^2}{n+1}} x^{\answer{n-1}}. \]
(Note that the $n=0$ term goes away because the derivative of a constant is zero.)
\end{exercise}

\begin{exercise}
Integrate the series below term-by-term:
\[ \int_0^x \left[ \sum_{n=0}^\infty \frac{x^n}{(2n+3)^2} \right]  = \sum_{n=0} \answer{\frac{1}{(n+1)(2n+3)^2}} x^{\answer{n+1}}. \]
\end{exercise}

\begin{exercise}
For each step below, apply a term-by-term operation, a multiplication by a monomial, or a substitution to derive a new series formula from a known formula.
\begin{itemize}
\item Use the formula \[ \frac{1}{1-x} = \sum_{n=0}^\infty x^n \]
to derive a summation formula for $1/(1+x^2)$, i.e.,
\[ \frac{1}{1+x^2} = \sum_{n=0}^\infty \answer{(-1)^n} x^{\answer{2n}}. \]
\item Use the formula you derived above to develop a power series expansion for arctangent:
\[ \arctan x = \sum_{n=0}^\infty \answer{\frac{(-1)^n}{(2n+1)}} x^{\answer{2n+1}}. \]
\item The radius of convergence of this last series is $R = \answer{1}$.
\end{itemize}
\end{exercise}

\begin{exercise}
For each step below, apply a term-by-term operation, a multiplication by a monomial, or a substitution to derive a new series formula from a known formula.
\begin{itemize}
\item Use the formula \[ \frac{1}{1-x}   = \sum_{n=0}^\infty x^n \] to determine the sum of the series below for $-1 < x < 1$:
\[ \sum_{n = 0}^\infty \frac{x^{n+1}}{n+1} = \answer{- \ln |1-x|}. \]
(Don't forget absolute values if you need them.)
\item Use the formula you just derived to determine the sum of the series
\[ \sum_{n=0}^\infty \frac{x^{2n+3}}{n+1} = \answer{- x \ln |1-x^2|.} \]
\item Use the formula you just derived to determine the sum of the series
\[ \sum_{n=0}^\infty \frac{2^{2n} x^{2n+3}}{n+1} = \answer{- \frac{1}{4} x \ln |1-4x^2|.} \]
\end{itemize}
\end{exercise}


\section*{Sample Quiz Questions}
\begin{question}%%%%%[PowerSerInterval001]

Find the full interval of convergence for the power series \[\sum_{m=2}^{\infty} \frac{(-3)^mm^2(x - 5)^m }{\ln m}.\]
(Hints will not be revealed until after you choose a response.)
\begin{multiplechoice}
\choice[correct]{\(\displaystyle \left(\frac{14}{3},\frac{16}{3}\right)\)}
\choice{\(\displaystyle \left[\frac{14}{3},\frac{16}{3}\right)\)}
\choice{\(\displaystyle \left(2,8\right]\)}
\choice{\(\displaystyle \left[2,8\right]\)}
\choice{\(\displaystyle \left(-\infty,\infty\right)\)}
\end{multiplechoice}
\begin{feedback}
First observe that 
\[\begin{aligned}
\frac{1}{R} & = \lim_{m\rightarrow \infty} \left| \frac{\frac{(-3)^{(m+1)}{(m+1)}^2 }{\ln {(m+1)}} }{\frac{(-3)^mm^2 }{\ln m}} \right| \\
& = \lim_{m\rightarrow \infty} \left| \frac{-3{(m+1)}^2\ln m }{m^2\ln {(m+1)} } \right| \\
& = 3
\end{aligned}\]
because \[\lim_{m \rightarrow \infty} \frac{(m+1)^2 \ln m}{m^2 \ln (m+1)} = \lim_{m \rightarrow \infty} \frac{(m+1)^2}{m^2} \lim_{m \rightarrow \infty} \frac{\ln m}{\ln (m+1)} = 1\] by virtue of l'Hospital's rule applied to both limits on the right-hand side.
\begin{hint}
This means that the radius equals \(1/3\). At the endpoint \(x=14/3\), the series equals \[\sum_{m=2}^{\infty} \frac{m^2 }{\ln m },\] which diverges by the \(n\)-th term divergence test because \(\lim_{m \rightarrow \infty} m^{2} / \ln m = \infty \neq 0\). At the endpoint \(x=16/3\), the series equals \[\sum_{m=2}^{\infty} (-1)^m\frac{m^2 }{\ln m },\] which diverges for the same reason as the other endpoint, i.e., the terms do not go to zero.
\end{hint}
\end{feedback}

\end{question}

\begin{question}%%%%%[PowerSerInterval003]

Find the full interval of convergence for the power series \[\sum_{m=2}^{\infty} \frac{(-4)^m(\ln m)(x - 1)^m }{(-2)^mm}.\]
(Hints will not be revealed until after you choose a response.)
\begin{multiplechoice}
\choice{\(\displaystyle \left(\frac{1}{2},\frac{3}{2}\right)\)}
\choice[correct]{\(\displaystyle \left[\frac{1}{2},\frac{3}{2}\right)\)}
\choice{\(\displaystyle \left(-1,3\right]\)}
\choice{\(\displaystyle \left[-1,3\right]\)}
\choice{\(\displaystyle \left(-\infty,\infty\right)\)}
\end{multiplechoice}
\begin{feedback}
First observe that 
\[\begin{aligned}
\frac{1}{R} & = \lim_{m\rightarrow \infty} \left| \frac{\frac{(-4)^{(m+1)}(\ln {(m+1)}) }{(-2)^{(m+1)}{(m+1)}} }{\frac{(-4)^m(\ln m) }{(-2)^mm}} \right| \\
& = \lim_{m\rightarrow \infty} \left| \frac{-4(\ln {(m+1)})m }{-2(\ln m){(m+1)} } \right| \\
& = 2
\end{aligned}\]
because \[\lim_{m \rightarrow \infty} \frac{m\ln (m+1)}{(m+1) \ln m} = \lim_{m \rightarrow \infty} \frac{m}{m+1} \lim_{m \rightarrow \infty} \frac{\ln (m+1)}{\ln m} = 1\] by virtue of l'Hospital's rule applied to both limits on the right-hand side. \begin{hint}
This means that the radius equals \(1/2\). At the endpoint \(x=3/2\), the series equals \[\sum_{m=2}^{\infty} \frac{(\ln m) }{m },\] which diverges by direct comparison to the harmonic series, i.e., the \(p\)-series with \(p = 1\). At the endpoint \(x=1/2\), the series equals \[\sum_{m=2}^{\infty} (-1)^m\frac{(\ln m) }{m },\] which converges by the alternating series test because the sign of the terms alternates and \(\ln m / m\) decreases to zero as \(m \rightarrow \infty\). \end{hint}
\end{feedback}

\end{question}

\begin{question}%%%%%[PowerSerInterval004]

Find the full interval of convergence for the power series \[\sum_{m=2}^{\infty} \frac{(-2)^m(\ln m)(x + 4)^m }{6^mm}.\]
(Hints will not be revealed until after you choose a response.)
\begin{multiplechoice}
\choice{\(\displaystyle \left(-\frac{13}{3},-\frac{11}{3}\right)\)}
\choice{\(\displaystyle \left[-\frac{13}{3},-\frac{11}{3}\right)\)}
\choice[correct]{\(\displaystyle \left(-7,-1\right]\)}
\choice{\(\displaystyle \left[-7,-1\right]\)}
\choice{\(\displaystyle \left(-\infty,\infty\right)\)}
\end{multiplechoice}
\begin{feedback}
First observe that 
\[\begin{aligned}
\frac{1}{R} & = \lim_{m\rightarrow \infty} \left| \frac{\frac{(-2)^{(m+1)}(\ln {(m+1)}) }{6^{(m+1)}{(m+1)}} }{\frac{(-2)^m(\ln m) }{6^mm}} \right| \\
& = \lim_{m\rightarrow \infty} \left| \frac{-2(\ln {(m+1)})m }{6(\ln m){(m+1)} } \right| \\
& = \frac{1}{3}
\end{aligned}\]
because \[\lim_{m \rightarrow \infty} \frac{m\ln (m+1)}{(m+1) \ln m} = \lim_{m \rightarrow \infty} \frac{m}{m+1} \lim_{m \rightarrow \infty} \frac{\ln (m+1)}{\ln m} = 1\] by virtue of l'Hospital's rule applied to both limits on the right-hand side. \begin{hint}
This means that the radius equals \(3\). At the endpoint \(x=-7\), the series equals \[\sum_{m=2}^{\infty} \frac{(\ln m) }{m },\] which diverges by direct comparison to the harmonic series, i.e., the \(p\)-series with \(p = 1\). At the endpoint \(x=-1\), the series equals \[\sum_{m=2}^{\infty} (-1)^m\frac{(\ln m) }{m },\] which converges by the alternating series test because the sign of the terms alternates and \(\ln m / m\) decreases to zero as \(m \rightarrow \infty\). \end{hint}
\end{feedback}

\end{question}

\begin{question}%%%%%[PowerSerInterval005]

Find the full interval of convergence for the power series \[\sum_{m=1}^{\infty} \frac{\sqrt[3]{m}(x - 2)^m }{m!}.\]
\begin{multiplechoice}
\choice{\(\displaystyle \left(1,3\right)\)}
\choice{\(\displaystyle \left[1,3\right)\)}
\choice{\(\displaystyle \left(1,3\right]\)}
\choice{\(\displaystyle \left[1,3\right]\)}
\choice[correct]{\(\displaystyle \left(-\infty,\infty\right)\)}
\end{multiplechoice}
\begin{feedback}
First observe that 
\[\begin{aligned}
\frac{1}{R} & = \lim_{m\rightarrow \infty} \left| \frac{\frac{\sqrt[3]{{(m+1)}} }{{(m+1)}!} }{\frac{\sqrt[3]{m} }{m!}} \right| \\
& = \lim_{m\rightarrow \infty} \left| \frac{\sqrt[3]{{(m+1)}} }{(m+1)\sqrt[3]{m} } \right| \\
& = 0
\end{aligned}\]
because \(m+1\) in the denominator tends to \(\infty\) and \[\lim_{m \rightarrow \infty} \frac{\sqrt[3]{m+1}}{\sqrt[3]{m}} = \lim_{m \rightarrow \infty} \left( 1 + m^{-1} \right)^{1/3} = \left( 1 + \lim_{m \rightarrow \infty} m^{-1} \right)^{1/3} = 1.\]
This means that the radius is infinite and the interval of convergence is \((-\infty,\infty)\).
\end{feedback}

\end{question}

\section*{Sample Exam Questions}

\begin{question}%%%%%[2015C.03]

For which values of \(x\) does the series \(\displaystyle \sum_{n=1}^\infty \frac{(-1)^{n+1}(x-1)^n}{n 4^n}\) converge?
\begin{multiplechoice}
\choice{\(-3 < x < 5\)}
\choice{\(-3 \leq x < 5\)}
\choice[correct]{\(-3 < x \leq 5\)}
\choice{\(-5 < x \leq 3\)}
\choice{\(-5 \leq x < 3\)}
\choice{\(-5 \leq x \leq 3\)}
\end{multiplechoice}

\end{question}

\begin{question}%%%%%[2016C.11]

Find the interval of convergence of the power series below.
\[ \sum_{n=1}^\infty \frac{(4x-1)^n}{n^\frac{3}{4} (n^2+2)} \]
\begin{multiplechoice}
\choice{\(\displaystyle \left( 0 , \frac{1}{2} \right]\)}
\choice[correct]{\(\displaystyle \left[ 0 , \frac{1}{2} \right]\)}
\choice{\(\displaystyle \left( 0 , \frac{1}{2} \right)\)}
\choice{\(\displaystyle \left[ 0 , \frac{1}{2} \right)\)}
\choice{\(\displaystyle \left( -\frac{1}{2} , 0\right]\)}
\choice{\(\left( - \infty, \infty \right)\)}
\end{multiplechoice}

\end{question}

\begin{question}%%%%%[2017C.13]

Find the interval of convergence of the power series \(\displaystyle \sum_{n=2}^\infty \frac{2^n (x+5)^n}{\sqrt[3]{n}}\).
\begin{multiplechoice}
\choice{\(\displaystyle \left[ -\frac{11}{2}, -\frac{9}{2} \right]\)}
\choice[correct]{\(\displaystyle \left[ -\frac{11}{2}, -\frac{9}{2} \right)\)}
\choice{\(\displaystyle \left( -\frac{11}{2}, -\frac{9}{2} \right)\)}
\choice{\(\displaystyle \left[ \frac{9}{2}, \frac{11}{2} \right)\)}
\choice{\(\displaystyle \left( \frac{9}{2}, \frac{11}{2} \right)\)}
\choice{\(\displaystyle \left[ \frac{9}{2}, \frac{11}{2} \right]\)}
\end{multiplechoice}

\end{question}


\end{document}
