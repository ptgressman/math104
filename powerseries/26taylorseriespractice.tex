\documentclass{ximera}
\graphicspath{
{./}
{volumes/}
{arclengths/}
{centroids/}
{techniques/}
{applications/}
{series/}
{powerseries/}
{odes/}
{lessons/}
}
\usepackage{booktabs}

\newcommand{\bigmath}[1]{$\displaystyle #1$}
\newcommand{\choicebreak}{}
\newenvironment{type}{}{}
\newenvironment{notes}{}{}
\newenvironment{keywords}{}{}
\newcommand{\offline}{}
\newenvironment{comments}{\begin{feedback}}{\end{feedback}}
\newenvironment{multiplechoice}{\begin{multipleChoice}}{\end{multipleChoice}}
\title{Exercises: Taylor Series}
%%%%%\author{Philip T. Gressman}

\begin{document}
\begin{abstract}
Exercises relating to Taylor series and their computation.
\end{abstract}
\maketitle


\begin{exercise}
Use the Maclaurin series for $\sin x$ to give a complete formula for the Maclaurin series for $x \sin x^2$:
\[ x \sin (x^2) = \sum_{m=0}^\infty  \answer{\frac{(-1)^m}{(2m+1)!}} x^{\answer{4m+3}} \]
\end{exercise}

\begin{exercise}
Use the Maclaurin series for $e^x$ and $e^{-x}$ to give a complete formula for the Maclaurin series for $\cosh x$:
\[ \cosh x = \sum_{n=0}^\infty \answer{\frac{1}{(2n)!}} x^{\answer{2n}}. \]
\begin{hint}
Your formula should account for the fact that there are only even powers of $x$.
\end{hint}
\end{exercise}

\begin{exercise}
Compute the Taylor series of the function $f(x) = \ln x$ centered at the point $x_0 = 2$.
\[ \answer{ \ln 2} + \answer{\frac{1}{2}} (x-2) + \answer{- \frac{1}{8}}(x-2)^2 + \answer{\frac{1}{24}}(x-2)^3 + \cdots = \ln 2 + \sum_{n=1}^\infty \answer{\frac{(-1)^{n-1}2^{-n}}{n}} (x-2)^n \]
\end{exercise}


\begin{exercise}
Compute the Maclaurin series of the function identified below.
\[ \int_0^x \frac{\sin t}{t} dt = \sum_{n=0}^\infty \answer{\frac{(-1)^n}{(2n+1)! (2n+1)}} x^{\answer{2n+1}} \]
\end{exercise}

\section*{Sample Quiz Questions}

\begin{question}%%%%%[TaylorExpand001]

Compute the first \(4\) nonzero terms in the Taylor series 
at \(x=0\) of the function \[\frac{d}{dx} \left[ x e^{x^2} \right].\]
\begin{multiplechoice}
\choice[correct]{\( \displaystyle 1 + 3x^{2} + \frac{5}{2}x^{4} + \frac{7}{6}x^{6}\)}
\choice{\( \displaystyle  - 1 - 3x^{2} + \frac{5}{2}x^{4} + \frac{7}{6}x^{6}\)}
\choice{\( \displaystyle 1 - 6x^{2} - \frac{5}{4}x^{4} + \frac{7}{6}x^{6}\)}
\choice{\( \displaystyle  - 1 - 6x^{2} + \frac{5}{4}x^{4} + \frac{7}{6}x^{6}\)}
\choice{\( \displaystyle 2 + 3x^{2} + \frac{5}{4}x^{4} + \frac{7}{6}x^{6}\)}
\choice{\( \displaystyle 2 - 3x^{2} - \frac{5}{4}x^{4} + \frac{7}{6}x^{6}\)}
\end{multiplechoice}
\begin{feedback}
Compute the series in stages beginning with substitution into known series:
\[e^{x^2} = 1 + x^{2} + \frac{1}{2}x^{4} + \frac{1}{6}x^{6} + \cdots \]
\[x e^{x^2} = x + x^{3} + \frac{1}{2}x^{5} + \frac{1}{6}x^{7} + \cdots \]
\[\frac{d}{dx} \left[ x e^{x^2} \right] = 1 + 3x^{2} + \frac{5}{2}x^{4} + \frac{7}{6}x^{6} + \cdots \]
\end{feedback}

\end{question}

\begin{question}%%%%%[TaylorExpand011]

Compute the first \(4\) nonzero terms in the Taylor series 
at \(x=0\) of the function \[\int_0^x \left( x \ln (1-x) \right) ~ dx.\]
\begin{multiplechoice}
\choice{\( \displaystyle  - \frac{1}{6}x^{3} - \frac{1}{8}x^{4} - \frac{2}{15}x^{5} - \frac{1}{24}x^{6}\)}
\choice{\( \displaystyle \frac{1}{6}x^{3} + \frac{1}{8}x^{4} - \frac{2}{15}x^{5} - \frac{1}{24}x^{6}\)}
\choice{\( \displaystyle  - \frac{1}{6}x^{3} + \frac{1}{4}x^{4} + \frac{1}{15}x^{5} - \frac{1}{24}x^{6}\)}
\choice{\( \displaystyle \frac{1}{6}x^{3} + \frac{1}{4}x^{4} - \frac{1}{15}x^{5} - \frac{1}{24}x^{6}\)}
\choice[correct]{\( \displaystyle  - \frac{1}{3}x^{3} - \frac{1}{8}x^{4} - \frac{1}{15}x^{5} - \frac{1}{24}x^{6}\)}
\choice{\( \displaystyle  - \frac{1}{3}x^{3} + \frac{1}{8}x^{4} + \frac{1}{15}x^{5} - \frac{1}{24}x^{6}\)}
\end{multiplechoice}
\begin{feedback}
Compute the series in stages beginning with substitution into known series:
\[\ln (1-x) =  - x - \frac{1}{2}x^{2} - \frac{1}{3}x^{3} - \frac{1}{4}x^{4} + \cdots \]
\[x \ln (1-x) =  - x^{2} - \frac{1}{2}x^{3} - \frac{1}{3}x^{4} - \frac{1}{4}x^{5} + \cdots \]
\[\int_0^x \left( x \ln (1-x) \right) ~ dx =  - \frac{1}{3}x^{3} - \frac{1}{8}x^{4} - \frac{1}{15}x^{5} - \frac{1}{24}x^{6} + \cdots \]
\end{feedback}

\end{question}

\section*{Sample Exam Questions}

\begin{question}%%%%%[2015C.01]

The first few nonzero terms of the Maclaurin series for \(f(x) = \ln ( 1 + \sin x)\) are:
\begin{multiplechoice}
\choice{\(\displaystyle 1 + \frac{1}{2} x - \frac{1}{8} x^2 + \frac{1}{24} x^3 + \cdots\)}
\choice{\(\displaystyle 1 + \frac{1}{2} x - \frac{1}{8} x^2 - \frac{1}{48} x^3 + \cdots\)} 
\choice{\(\displaystyle x - \frac{1}{2} x^2 + \frac{1}{8} x^3 - \frac{1}{24} x^4 + \cdots\)}
\choice{\(\displaystyle 1 + x + \frac{1}{2} x^2 + \frac{1}{3} x^3 + \frac{1}{6} x^4 \cdots\)} 
\choice[correct]{\(\displaystyle x - \frac{1}{2} x^2 + \frac{1}{6} x^3 - \frac{1}{12} x^4 + \cdots\)}
\choice{\(\displaystyle 1 + x + \frac{1}{2} x^2 + \frac{1}{3} x^3 - \frac{1}{12} x^4 + \cdots\)}
\end{multiplechoice}
(Hints will not be revealed until you choose your response.)
\begin{feedback}
The first few derivatives of \(f(x)\) are:
\[\begin{aligned}
f(x) & = \ln (1 + \sin(x)) \\
f'(x) & = \frac{ \cos x}{1 + \sin x} \\
f''(x) & = - \frac{\sin x}{1 + \sin x} - \frac{\cos^2 x}{(1 + \sin x)^2} \\
f'''(x) & = - \frac{\cos x}{1 + \sin x} + \frac{\sin x \cos x}{(1 + \sin x)^2} - \frac{2 \sin x \cos x}{(1 + \sin x)^2} + 2 \frac{\cos^3 x}{(1 + \sin x)^3}
\end{aligned}\]
\begin{hint}
Evaluating at \(x = 0\) gives \(f(0) = \ln 1 = 0\), \(f'(0) = 1\), \(f''(0) = -1\), and \(f'''(0) = 1\). Therefore the series starts with the terms \(x - \frac{1}{2} x^2 + \frac{1}{6} x^3 + \cdots\).
\end{hint}
\end{feedback}

\end{question}

\begin{question}%%%%%[2016C.12]

Find the Taylor polynomial of degree 2 for \(f(x) = \sqrt{x+16}\) centered at \(x=9\).
\begin{multiplechoice}
\choice{\(\displaystyle 5 + \frac{4}{5} x + \frac{9}{250} x^2\)}
\choice{\(\displaystyle 5 - \frac{3}{5} (x-5) + \frac{1}{125} (x-5)^2\)}
\choice[correct]{\(\displaystyle 5 + \frac{1}{10} (x-9) - \frac{1}{1000} (x-9)^2\)} 
\choice{\(\displaystyle 5 + \frac{3}{5} (x-5) + \frac{8}{125} (x-5)^2\)}
\choice{\(\displaystyle 5 + \frac{1}{5} (x-9) + \frac{16}{125} (x-9)^2\)}
\choice{none of these}
\end{multiplechoice}

\end{question}

\begin{question}%%%%%[2017C.14]

Use the Taylor polynomial of degree \(3\) for \(f(x) = \ln (1+x)\) centered at \(x_0 = 0\) to approximate the value of \(\displaystyle \ln \left( \frac{3}{2} \right)\).
\begin{multiplechoice}
\choice{\(\displaystyle \frac{2}{3}\)}
\choice{\(\displaystyle \frac{3}{2}\)}
\choice{\(\displaystyle \frac{15}{4}\)}
\choice[correct]{\(\displaystyle \frac{5}{12}\)}
\choice{\(\displaystyle \frac{9}{24}\)}
\choice{\(\displaystyle \frac{11}{24}\)}
\end{multiplechoice}

\end{question}

\begin{question}%%%%%[2017C.15]

Let \(F(x)\) be the unique function that satisfies \(F(0) = 0\), \(F'(0) = 0\), and \(F'(x) = \frac{1}{x} \sin x^3\) for all \(x \neq 0\). Find the Taylor Series of \(F(x)\) centered at \(x_0 = 0\).
\begin{multiplechoice}
\choice{\(\displaystyle \sum_{n=0}^\infty \frac{(-1)^n x^{6n+3}}{(2n+1)!}\)}
\choice{\(\displaystyle \sum_{n=0}^\infty \frac{(-1)^n (6n+3) x^{6n+2}}{(2n+1)!}\)}
\choice[correct]{\(\displaystyle \sum_{n=0}^\infty \frac{(-1)^n x^{6n+3}}{(6n+3)(2n+1)!}\)} 
\choice{\(\displaystyle \sum_{n=0}^\infty \frac{(-1)^n x^{6n+2}}{(2n+1)!}\)}
\choice{\(\displaystyle \sum_{n=0}^\infty \frac{(-1)^n (6n+2) x^{6n+2}}{(2n+1)!}\)}
\choice{\(\displaystyle \sum_{n=0}^\infty \frac{(-1)^n x^{2n+3}}{(6n+3)(2n+1)!}\)}
\end{multiplechoice}

\end{question}


\end{document}
