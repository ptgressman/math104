\documentclass{ximera}
\graphicspath{
{./}
{volumes/}
{arclengths/}
{centroids/}
{techniques/}
{applications/}
{series/}
{powerseries/}
{odes/}
}

\newcommand{\bigmath}[1]{$\displaystyle #1$}
\newcommand{\choicebreak}{}
\newenvironment{type}{}{}
\newenvironment{notes}{}{}
\newenvironment{keywords}{}{}
\newcommand{\offline}{}
\newenvironment{comments}{\begin{feedback}}{\end{feedback}}
\newenvironment{multiplechoice}{\begin{multipleChoice}}{\end{multipleChoice}}
\title{Exercises: Taylor Series Applications}
\author{Philip T. Gressman}

\begin{document}
\begin{abstract}
Various exercises relating to the application of Taylor Series to other problems of interest.
\end{abstract}
\maketitle

\section*{Sample Quiz Questions}

\begin{question}[TaylorExpand001]
\begin{type}
multiplechoice
\end{type}
Compute the first \(4\) nonzero terms in the Taylor series 
at \(x=0\) of the function \[\frac{d}{dx} \left[ x e^{x^2} \right].\]
\begin{multiplechoice}
\choice[correct]{\( \displaystyle 1 + 3x^{2} + \frac{5}{2}x^{4} + \frac{7}{6}x^{6}\)}
\choice{\( \displaystyle  - 1 - 3x^{2} + \frac{5}{2}x^{4} + \frac{7}{6}x^{6}\)}
\choice{\( \displaystyle 1 - 6x^{2} - \frac{5}{4}x^{4} + \frac{7}{6}x^{6}\)}
\choice{\( \displaystyle  - 1 - 6x^{2} + \frac{5}{4}x^{4} + \frac{7}{6}x^{6}\)}
\choice{\( \displaystyle 2 + 3x^{2} + \frac{5}{4}x^{4} + \frac{7}{6}x^{6}\)}
\choice{\( \displaystyle 2 - 3x^{2} - \frac{5}{4}x^{4} + \frac{7}{6}x^{6}\)}
\end{multiplechoice}
\begin{feedback}
Compute the series in stages beginning with substitution into known series:
\[e^{x^2} = 1 + x^{2} + \frac{1}{2}x^{4} + \frac{1}{6}x^{6} + \cdots \]
\[x e^{x^2} = x + x^{3} + \frac{1}{2}x^{5} + \frac{1}{6}x^{7} + \cdots \]
\[\frac{d}{dx} \left[ x e^{x^2} \right] = 1 + 3x^{2} + \frac{5}{2}x^{4} + \frac{7}{6}x^{6} + \cdots \]
\end{feedback}
\begin{keywords}
Taylor series computation,generation2
\end{keywords}
\end{question}

\begin{question}[TaylorExpand011]
\begin{type}
multiplechoice
\end{type}
Compute the first \(4\) nonzero terms in the Taylor series 
at \(x=0\) of the function \[\int_0^x \left( x \ln (1-x) \right) ~ dx.\]
\begin{multiplechoice}
\choice{\( \displaystyle  - \frac{1}{6}x^{3} - \frac{1}{8}x^{4} - \frac{2}{15}x^{5} - \frac{1}{24}x^{6}\)}
\choice{\( \displaystyle \frac{1}{6}x^{3} + \frac{1}{8}x^{4} - \frac{2}{15}x^{5} - \frac{1}{24}x^{6}\)}
\choice{\( \displaystyle  - \frac{1}{6}x^{3} + \frac{1}{4}x^{4} + \frac{1}{15}x^{5} - \frac{1}{24}x^{6}\)}
\choice{\( \displaystyle \frac{1}{6}x^{3} + \frac{1}{4}x^{4} - \frac{1}{15}x^{5} - \frac{1}{24}x^{6}\)}
\choice[correct]{\( \displaystyle  - \frac{1}{3}x^{3} - \frac{1}{8}x^{4} - \frac{1}{15}x^{5} - \frac{1}{24}x^{6}\)}
\choice{\( \displaystyle  - \frac{1}{3}x^{3} + \frac{1}{8}x^{4} + \frac{1}{15}x^{5} - \frac{1}{24}x^{6}\)}
\end{multiplechoice}
\begin{feedback}
Compute the series in stages beginning with substitution into known series:
\[\ln (1-x) =  - x - \frac{1}{2}x^{2} - \frac{1}{3}x^{3} - \frac{1}{4}x^{4} + \cdots \]
\[x \ln (1-x) =  - x^{2} - \frac{1}{2}x^{3} - \frac{1}{3}x^{4} - \frac{1}{4}x^{5} + \cdots \]
\[\int_0^x \left( x \ln (1-x) \right) ~ dx =  - \frac{1}{3}x^{3} - \frac{1}{8}x^{4} - \frac{1}{15}x^{5} - \frac{1}{24}x^{6} + \cdots \]
\end{feedback}
\begin{keywords}
Taylor series computation,generation2
\end{keywords}
\end{question}

\begin{question}[TAp0]
\begin{type}
multiplechoice
\end{type}
Use Taylor series to estimate the value of
\[\sqrt[3]{\frac{11}{10}}\] to within an error of at most \(1/900\).
\begin{multiplechoice}
\choice[correct]{\(\displaystyle \frac{31}{30}\)}
\choice{\(\displaystyle \frac{47}{45}\)}
\choice{\(\displaystyle \frac{19}{18}\)}
\choice{\(\displaystyle \frac{16}{15}\)}
\choice{\(\displaystyle \frac{97}{90}\)}
\choice{\(\displaystyle \frac{49}{45}\)}
\end{multiplechoice}
\begin{feedback}
We may use the remainder formula for Taylor series to approach this problem.
Suppose \(p_n(x)\) is the degree \(n\) Taylor polynomial of the function
\[ f(x) = \sqrt[3]{1+x}\]
with center \(a=0\). Then the error \(E_n(x)\), i.e., the difference between the polynomial and the function, does not exceed \(\frac{f^{(n+1)}(\xi)}{(n+1)!}x^{n+1}\), where \(\xi\) is some unknown point  in the range \(0 \leq \xi \leq x\). In this case one should take \(x = 1/10\) and determine how many derivatives are required to make this error estimate less than the given threshold. This means checking by hand for small numbers of derivatives. For the specific problem at hand, if we approximate \(f(x)\) by the Taylor polynomial of degree \( n = 1\), we have 
\[ \left| \frac{x^{n+1} }{(n+1)!} f^{(n+1)}(\xi) \right| = \left| \frac{x^{n+1}}{(n+1)!} \left( -\frac{2}{9} (\xi+1)^{-5/3} \right) \right| \leq \left| \frac{x^{n+1}}{(n+1)!} \left( \frac{2}{9} \right) \right| = \frac{1}{900} \] when \(x = 1/10\).
We conclude that the correct Taylor approximation is
\[ p_n \left( \frac{1}{10}\right) =  \left(\frac{1}{10}\right)^{0} + \frac{1}{3}\left(\frac{1}{10}\right)^{1} =  1 + \frac{1}{30} = \frac{31}{30}.\]
\end{feedback}
\begin{keywords}
Taylor series approximation,generation2
\end{keywords}
\end{question}

\begin{question}[TAp1]
\begin{type}
multiplechoice
\end{type}
Use Taylor series to estimate the value of
\[e^{-\frac{1}{3}}\] to within an error of at most \(1/162\).
\begin{multiplechoice}
\choice{\(\displaystyle \frac{5}{9}\)}
\choice[correct]{\(\displaystyle \frac{13}{18}\)}
\choice{\(\displaystyle \frac{8}{9}\)}
\choice{\(\displaystyle \frac{19}{18}\)}
\choice{\(\displaystyle \frac{11}{9}\)}
\choice{\(\displaystyle \frac{25}{18}\)}
\end{multiplechoice}
\begin{feedback}
We may use the remainder formula for Taylor series to approach this problem.
Suppose \(p_n(x)\) is the degree \(n\) Taylor polynomial of the function
\[ f(x) = e^{-x}\]
with center \(a=0\). Then the error \(E_n(x)\), i.e., the difference between the polynomial and the function, does not exceed \(\frac{f^{(n+1)}(\xi)}{(n+1)!}x^{n+1}\), where \(\xi\) is some unknown point  in the range \(0 \leq \xi \leq x\). In this case one should take \(x = 1/3\) and determine how many derivatives are required to make this error estimate less than the given threshold. This means checking by hand for small numbers of derivatives. For the specific problem at hand, if we approximate \(f(x)\) by the Taylor polynomial of degree \( n = 2\), we have 
\[ \left| \frac{x^{n+1} }{(n+1)!} f^{(n+1)}(\xi) \right| = \left| \frac{x^{n+1}}{(n+1)!} \left( e^{-\xi} \right) \right| \leq \left| \frac{x^{n+1}}{(n+1)!} \left( 1 \right) \right| = \frac{1}{162} \] when \(x = 1/3\).
We conclude that the correct Taylor approximation is
\[ p_n \left( \frac{1}{3}\right) =  \left(\frac{1}{3}\right)^{0} -1\left(\frac{1}{3}\right)^{1} + \frac{1}{2}\left(\frac{1}{3}\right)^{2} =  1 -\frac{1}{3} + \frac{1}{18} = \frac{13}{18}.\]
\end{feedback}
\begin{keywords}
Taylor series approximation,generation2
\end{keywords}
\end{question}


\end{document}
