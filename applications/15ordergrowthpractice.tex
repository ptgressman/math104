\documentclass{ximera}
\graphicspath{
{./}
{volumes/}
{arclengths/}
{centroids/}
{techniques/}
{applications/}
{series/}
{powerseries/}
{odes/}
{lessons/}
}
\usepackage{booktabs}

\newcommand{\bigmath}[1]{$\displaystyle #1$}
\newcommand{\choicebreak}{}
\newenvironment{type}{}{}
\newenvironment{notes}{}{}
\newenvironment{keywords}{}{}
\newcommand{\offline}{}
\newenvironment{comments}{\begin{feedback}}{\end{feedback}}
\newenvironment{multiplechoice}{\begin{multipleChoice}}{\end{multipleChoice}}
\title{Exercises: Orders of Growth}
%%%%%\author{Philip T. Gressman}

\begin{document}
\begin{abstract}
Various exercises relating to orders of growth.
\end{abstract}
\maketitle

\section*{Sample Quiz Questions}

\begin{question}%%%%%[GrowthOrdINF001]

Arrange the functions 
\[ x^{x} \qquad \frac{e^{x}}{\ln{x}} \qquad \ln{x} \]
in order from least rate of growth to greatest rate of growth as \(x \rightarrow \infty\). Compare on the basis of magnitude rather than sign, i.e., if a function is negative, take its absolute value first.
\begin{multiplechoice}
\choice{\(\displaystyle x^{x} < \! \! < \frac{e^{x}}{\ln{x}} < \! \! < \ln{x}\)}
\choice{\(\displaystyle \ln{x} < \! \! < x^{x} < \! \! < \frac{e^{x}}{\ln{x}}\)}
\choice{\(\displaystyle \frac{e^{x}}{\ln{x}} < \! \! < \ln{x} < \! \! < x^{x}\)} 
\choice{\(\displaystyle x^{x} < \! \! < \ln{x} < \! \! < \frac{e^{x}}{\ln{x}}\)}
\choice{\(\displaystyle \frac{e^{x}}{\ln{x}} < \! \! < x^{x} < \! \! < \ln{x}\)}
\choice[correct]{\(\displaystyle \ln{x} < \! \! < \frac{e^{x}}{\ln{x}} < \! \! < x^{x}\)}
\end{multiplechoice}
\begin{feedback}
General Remarks:
\begin{itemize} \item Higher powers of \(x\) grow faster at infinity than lower powers of \(x\).
\item As \(x \rightarrow \infty\), \(\ln x\) goes to infinity slower than \(x^p\) for any (presumably small) positive constant \(p\).
\item As \(x \rightarrow \infty\), \(e^x\) goes to infinity faster than \(x^n\) for any (presumably large) positive constant \(n\).
\item As \(x \rightarrow \infty\), \(x^x\) goes to infinity faster than any exponential of the form \(e^{cx}\) for any constant \(c\).
\end{itemize}
\end{feedback}

\end{question}

\begin{question}%%%%%[GrowthOrdINF002]

Arrange the functions 
\[ \frac{e^{- x}}{\ln{x}} \qquad x^{3} \ln{x} \qquad e^{x} \]
in order from least rate of growth to greatest rate of growth as \(x \rightarrow \infty\). Compare on the basis of magnitude rather than sign, i.e., if a function is negative, take its absolute value first.
\begin{multiplechoice}
\choice[correct]{\(\displaystyle \frac{e^{- x}}{\ln{x}} < \! \! < x^{3} \ln{x} < \! \! < e^{x}\)}
\choice{\(\displaystyle e^{x} < \! \! < \frac{e^{- x}}{\ln{x}} < \! \! < x^{3} \ln{x}\)}
\choice{\(\displaystyle x^{3} \ln{x} < \! \! < e^{x} < \! \! < \frac{e^{- x}}{\ln{x}}\)} 
\choice{\(\displaystyle \frac{e^{- x}}{\ln{x}} < \! \! < e^{x} < \! \! < x^{3} \ln{x}\)}
\choice{\(\displaystyle x^{3} \ln{x} < \! \! < \frac{e^{- x}}{\ln{x}} < \! \! < e^{x}\)}
\choice{\(\displaystyle e^{x} < \! \! < x^{3} \ln{x} < \! \! < \frac{e^{- x}}{\ln{x}}\)}
\end{multiplechoice}
\begin{feedback}
General Remarks:
\begin{itemize} \item Higher powers of \(x\) grow faster at infinity than lower powers of \(x\).
\item As \(x \rightarrow \infty\), \(\ln x\) goes to infinity slower than \(x^p\) for any (presumably small) positive constant \(p\).
\item As \(x \rightarrow \infty\), \(e^x\) goes to infinity faster than \(x^n\) for any (presumably large) positive constant \(n\).
\end{itemize}
\end{feedback}

\end{question}


\begin{question}%%%%%[GrowthOrdZERO001]

Arrange the functions 
\[ \left( \ln \frac{1}{x} \right)^2 \qquad \frac{e^{- x}}{x^{3}} \ln{x} \qquad x^{3} e^{x} \]
in order from least rate of growth to greatest rate of growth as \(x \rightarrow 0^+\). Compare on the basis of magnitude rather than sign, i.e., if a function is negative, take its absolute value first.
\begin{multiplechoice}
\choice{\(\displaystyle \left( \ln \frac{1}{x} \right)^2 < \! \! < \frac{e^{- x}}{x^{3}} \ln{x} < \! \! < x^{3} e^{x}\)}
\choice[correct]{\(\displaystyle x^{3} e^{x} < \! \! < \left( \ln \frac{1}{x} \right)^2 < \! \! < \frac{e^{- x}}{x^{3}} \ln{x}\)}
\choice{\(\displaystyle \frac{e^{- x}}{x^{3}} \ln{x} < \! \! < x^{3} e^{x} < \! \! < \left( \ln \frac{1}{x} \right)^2\)} 
\choice{\(\displaystyle \left( \ln \frac{1}{x} \right)^2 < \! \! < x^{3} e^{x} < \! \! < \frac{e^{- x}}{x^{3}} \ln{x}\)}
\choice{\(\displaystyle \frac{e^{- x}}{x^{3}} \ln{x} < \! \! < \left( \ln \frac{1}{x} \right)^2 < \! \! < x^{3} e^{x}\)}
\choice{\(\displaystyle x^{3} e^{x} < \! \! < \frac{e^{- x}}{x^{3}} \ln{x} < \! \! < \left( \ln \frac{1}{x} \right)^2\)}
\end{multiplechoice}
\begin{feedback}
General Remarks:
\begin{itemize} \item Lower powers of \(x\) grow faster as \(x \rightarrow 0^+\) than higher powers of \(x\).
\item As \(x \rightarrow 0^+\), \(-\ln x = \ln x^{-1}\) goes to \(\infty\) slower than \(x^{-p}\) for any (presumably small) positive \(p\).
\item As \(x \rightarrow 0^+\), \(e^{x} \rightarrow 1\) and so does not influence the growth rate.
\item As \(x \rightarrow 0^+\), \(e^{-x} \rightarrow 1\) and so does not influence the growth rate.
\end{itemize}
\end{feedback}

\end{question}

\begin{question}%%%%%[GrowthOrdZERO005]

Arrange the functions 
\[ \frac{x^{3} e^{x}}{\ln{x}} \qquad \frac{e^{- x}}{x^{3}} \qquad \frac{e^{- x}}{x^{3} \ln{x}} \]
in order from least rate of growth to greatest rate of growth as \(x \rightarrow 0^+\). Compare on the basis of magnitude rather than sign, i.e., if a function is negative, take its absolute value first.
\begin{multiplechoice}
\choice{\(\displaystyle \frac{x^{3} e^{x}}{\ln{x}} < \! \! < \frac{e^{- x}}{x^{3}} < \! \! < \frac{e^{- x}}{x^{3} \ln{x}}\)}
\choice{\(\displaystyle \frac{e^{- x}}{x^{3} \ln{x}} < \! \! < \frac{x^{3} e^{x}}{\ln{x}} < \! \! < \frac{e^{- x}}{x^{3}}\)}
\choice{\(\displaystyle \frac{e^{- x}}{x^{3}} < \! \! < \frac{e^{- x}}{x^{3} \ln{x}} < \! \! < \frac{x^{3} e^{x}}{\ln{x}}\)} 
\choice[correct]{\(\displaystyle \frac{x^{3} e^{x}}{\ln{x}} < \! \! < \frac{e^{- x}}{x^{3} \ln{x}} < \! \! < \frac{e^{- x}}{x^{3}}\)}
\choice{\(\displaystyle \frac{e^{- x}}{x^{3}} < \! \! < \frac{x^{3} e^{x}}{\ln{x}} < \! \! < \frac{e^{- x}}{x^{3} \ln{x}}\)}
\choice{\(\displaystyle \frac{e^{- x}}{x^{3} \ln{x}} < \! \! < \frac{e^{- x}}{x^{3}} < \! \! < \frac{x^{3} e^{x}}{\ln{x}}\)}
\end{multiplechoice}
\begin{feedback}
General Remarks:
\begin{itemize} \item Lower powers of \(x\) grow faster as \(x \rightarrow 0^+\) than higher powers of \(x\).
\item As \(x \rightarrow 0^+\), \(-\ln x = \ln x^{-1}\) goes to \(\infty\) slower than \(x^{-p}\) for any (presumably small) positive \(p\).
\item As \(x \rightarrow 0^+\), \(e^{x} \rightarrow 1\) and so does not influence the growth rate.
\item As \(x \rightarrow 0^+\), \(e^{-x} \rightarrow 1\) and so does not influence the growth rate.
\end{itemize}
\end{feedback}

\end{question}






\end{document}
