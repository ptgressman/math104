\documentclass{ximera}
\graphicspath{
{./}
{volumes/}
{arclengths/}
{centroids/}
{techniques/}
{applications/}
{series/}
{powerseries/}
{odes/}
{lessons/}
}
\usepackage{booktabs}

\newcommand{\bigmath}[1]{$\displaystyle #1$}
\newcommand{\choicebreak}{}
\newenvironment{type}{}{}
\newenvironment{notes}{}{}
\newenvironment{keywords}{}{}
\newcommand{\offline}{}
\newenvironment{comments}{\begin{feedback}}{\end{feedback}}
\newenvironment{multiplechoice}{\begin{multipleChoice}}{\end{multipleChoice}}
\title{Exercises: Improper Integrals}
%%%%%\author{Philip T. Gressman}

\begin{document}
\begin{abstract}
Various exercises relating to improper integrals.
\end{abstract}
\maketitle

\begin{exercise}%[APEX0607IMPRP07]
Evaluate the improper integral: \(\displaystyle \int_0^\infty e^{5-2x}\ dx = \answer{(e^5)/2}.\)
%
%
\end{exercise}

\begin{exercise}%[APEX0607IMPRP10]
Evaluate the given improper integral: \(\displaystyle \int_{-\infty}^\infty \frac{1}{x^2+9}\ dx = \answer{\pi/3}.\)
%
%
\end{exercise}

\begin{exercise}
Evaluate the integral: \[ \int_0^1 \frac{e^{\sqrt{x}}}{\sqrt{x}} \ dx = \answer{2(e-1)}. \]
This integral is \wordChoice{\choice{not improper}\choice[correct]{improper}} because of the behavior of the integrand near $x = 0$.
\end{exercise}

\begin{exercise}%[APEX0607IMPRP14]
Evaluate the given improper integral. \(\displaystyle \int_{3}^\infty\frac{1}{x^2-4}\ dx = \answer{\frac{\ln 5}{4}}.\)
%
%
\end{exercise}

\begin{exercise}%[APEX0607IMPRP35]
Use the Direct Comparison Test or the Limit Comparison Test to determine whether the integral converges or diverges:
 \(\displaystyle \int_{10}^\infty \frac{3}{\sqrt{3x^2+2x-5}} \ dx.\)
 Answer: the integral 
 \wordChoice{\choice{converges}\choice[correct]{diverges}} by \wordChoice{\choice{direct}\choice[correct]{limit}} comparison with the function $\displaystyle \frac{1}{x^{\answer{1}}}.$
%
%
\end{exercise}

\begin{exercise}
Use the Direct Comparison Test or the Limit Comparison Test to determine whether the integral converges or diverges:
 \(\displaystyle \int_{2}^\infty \frac{2}{\sqrt{7x^3-x}} \ dx.\)
 Answer: the integral 
 \wordChoice{\choice[correct]{converges}\choice{diverges}} by \wordChoice{\choice{direct}\choice[correct]{limit}} comparison with the function $\displaystyle \frac{1}{x^{\answer{3/2}}}$ (select the largest exponent for the denominator which makes the statement true).
\end{exercise}


\begin{exercise}
Use the Direct Comparison Test or the Limit Comparison Test to determine whether the integral converges or diverges:
 \(\displaystyle \int_{1}^\infty e^{-x} \ln x \ dx.\)
 Answer: the integral 
 \wordChoice{\choice[correct]{converges}\choice{diverges}} by direct comparison with the function 
 \begin{multipleChoice}
 \choice{$e^{-x}$}
 \choice[correct]{$x e^{-x}$}
 \choice{$e^{-x}/x$}
 \end{multipleChoice}
\end{exercise}

\begin{exercise}
Use the Direct Comparison Test or the Limit Comparison Test to determine whether the integral converges or diverges:
 \(\displaystyle \int_{1}^\infty e^{-x^2 + 3x + 1} \ dx.\)
 Answer: the integral 
 \wordChoice{\choice[correct]{converges}\choice{diverges}} by direct comparison with the function 
 \begin{multipleChoice}
 \choice{$e^{-x^2}$}
 \choice[correct]{$e^{-x}$}
 \choice{$e^{3x+1}$}
 \end{multipleChoice}
\end{exercise}

\begin{exercise}
Use the Direct Comparison Test or the Limit Comparison Test to determine whether the integral converges or diverges:
 \(\displaystyle \int_{1}^\infty \frac{x}{x^2+\cos x} \ dx.\)
 Answer: the integral 
 \wordChoice{\choice{converges}\choice[correct]{diverges}} by \wordChoice{\choice{direct}\choice[correct]{limit}} comparison with the function 
 \begin{multipleChoice}
 \choice[correct]{$1/x$}
 \choice{$x/\cos x$}
 \choice{$1/(x^2+\cos x)$}
 \end{multipleChoice}
\end{exercise}


\begin{exercise}
Use the Direct or Limit Comparison Test to determine whether the integral converges or diverges:
\[ \int_0^{1/e} \frac{(\ln x)^2-1}{x^3 + x^2 + x} dx \]
Answer: The integral \wordChoice{\choice{converges}\choice[correct]{diverges}} by \wordChoice{\choice{direct}\choice[correct]{limit}} comparison with the function 
\begin{multipleChoice}
\choice{$\displaystyle \frac{(\ln x)^2}{x^3}$}
\choice{$\displaystyle \frac{(\ln x)^2}{x^2}$}
\choice[correct]{$\displaystyle \frac{(\ln x)^2}{x}$}
\choice{$\displaystyle \frac{1}{x^3}$}
\choice{$\displaystyle \frac{1}{x^2}$}
\end{multipleChoice}
\end{exercise}

\begin{exercise}
Use the Direct or Limit Comparison Test to determine whether the integral converges or diverges:
\[ \int_0^{1/e} \frac{(\ln x)^2-1}{x + \sqrt{x} + e^{-1/x}} dx \]
Answer: The integral \wordChoice{\choice[correct]{converges}\choice{diverges}} by direct comparison with the function 
\begin{multipleChoice}
\choice{$\displaystyle \frac{(\ln x)^2}{x}$}
\choice[correct]{$\displaystyle \frac{(\ln x)^2}{\sqrt{x}}$}
\choice{$\displaystyle \frac{(\ln x)^2}{e^{-1/x}}$}
\choice{$\displaystyle \frac{1}{x}$}
\choice{$\displaystyle \frac{1}{\sqrt{x}}$}
\choice{$\displaystyle \frac{1}{e^{-1/x}}$}
\end{multipleChoice}
\end{exercise}




\section*{Sample Quiz Questions}

\begin{question}%%%%%[ImpropCD01]

Which of the following improper integrals is convergent? \offline{Show how you used comparison tests to justify your answer.}
\[ \mathrm{I}: \ \int_0^1\frac{\sqrt{{1}+{x^2}{e^{-x}}}}{{(\cos x)}{x^2}}~dx \qquad   \mathrm{II}: \ \int_{2}^\infty\frac{{e^{x}}}{{x}{e^{x}}+{x^2}}~dx \qquad  \mathrm{III}: \ \int_{2}^\infty\frac{{x^2}}{{x^4}+{1}}~dx\]
\begin{multiplechoice}
\choice{only \(\mathrm{I}\) converges}
\choice{only \(\mathrm{II}\) converges}
\choice[correct]{only \(\mathrm{III}\) converges} 
\choice{\(\mathrm{I}\) and \(\mathrm{II}\) converge}
\choice{\(\mathrm{II}\) and \(\mathrm{III}\) converge}
\choice{\(\mathrm{I}\) and \(\mathrm{III}\) converge}
\end{multiplechoice}
\begin{feedback}
Integral \(\mathrm{I}\) is divergent by direct comparison to the function \(\displaystyle \frac{{1}}{{x^2}}\). 
Integral \(\mathrm{II}\) is divergent by limit comparison to the function \(\displaystyle \frac{{1}}{{x}}\). 
Integral \(\mathrm{III}\) is convergent by direct comparison to the function \(\displaystyle \frac{{1}}{{x^2}}\).
\end{feedback}

\end{question}

\begin{question}%%%%%[ImpropCD08]

Which of the following improper integrals is convergent? \offline{Show how you used comparison tests to justify your answer.}
\[ \mathrm{I}: \ \int_0^1\frac{\sqrt{{e^{2x}}+{x^3}}}{{x}}~dx \qquad   \mathrm{II}: \ \int_0^1\frac{{x^2}}{{x^2}{\sqrt{x}}+{x^3}}~dx \qquad  \mathrm{III}: \ \int_{2}^\infty\frac{{x^2}{\ln x}}{-{x}+{x^4}}~dx\]
\begin{multiplechoice}
\choice{only \(\mathrm{I}\) converges}
\choice{only \(\mathrm{II}\) converges}
\choice{only \(\mathrm{III}\) converges} 
\choice{\(\mathrm{I}\) and \(\mathrm{II}\) converge}
\choice[correct]{\(\mathrm{II}\) and \(\mathrm{III}\) converge}
\choice{\(\mathrm{I}\) and \(\mathrm{III}\) converge}
\end{multiplechoice}
\begin{feedback}
Integral \(\mathrm{I}\) is divergent by direct comparison to the function \(\displaystyle \frac{{1}}{{x}}\). 
Integral \(\mathrm{II}\) is convergent by direct comparison to the function \(\displaystyle \frac{{1}}{{\sqrt{x}}}\). 
Integral \(\mathrm{III}\) is convergent by limit comparison to the function \(\displaystyle \frac{{\ln x}}{{x^2}}\).
\end{feedback}

\end{question}

\section*{Sample Exam Questions}

\begin{question}%%%%%[2016C.06]

Only one of the following four improper integrals diverges. Choose that improper integral\offline{ and justify why it diverges}. \offline{(You need NOT justify why the other integrals converge.)}
\begin{multiplechoice}
\choice{\(\displaystyle \int^{\infty}_{2} \frac{\arctan x}{1+x^3} dx\)}
\choice{\(\displaystyle \int^{\infty}_{2} \frac{1}{\sqrt{x^4+x^2}} dx\)}
\choice{\(\displaystyle \int^{\infty}_{2} \frac{1+\sin x}{x^2} dx\)}
\choice[correct]{\(\displaystyle \int_{2}^{\infty} \frac{1}{\sqrt[3]{x^2-1}} dx\)}
\end{multiplechoice}


\end{question}


\end{document}
