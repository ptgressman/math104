\documentclass{ximera}
\graphicspath{
{./}
{volumes/}
{arclengths/}
{centroids/}
{techniques/}
{applications/}
{series/}
{powerseries/}
{odes/}
{lessons/}
}
\usepackage{booktabs}

\newcommand{\bigmath}[1]{$\displaystyle #1$}
\newcommand{\choicebreak}{}
\newenvironment{type}{}{}
\newenvironment{notes}{}{}
\newenvironment{keywords}{}{}
\newcommand{\offline}{}
\newenvironment{comments}{\begin{feedback}}{\end{feedback}}
\newenvironment{multiplechoice}{\begin{multipleChoice}}{\end{multipleChoice}}
\title{Exercises: Probability}
\author{Philip T. Gressman}

\begin{document}
\begin{abstract}
Various exercises relating to probability.
\end{abstract}
\maketitle


\section*{Sample Quiz Questions}
\begin{question}%%%%%[ProbabAssort001]

Find the value of \(c\) which makes the function
\[f(x) = \frac{1}{2}e^{-x} - ce^{-2x}\]
a probability density function on the interval \([0, \infty)\). 
What is the value of the mean \(\mu\) of the corresponding random variable?
\begin{multiplechoice}
\choice{\(\displaystyle c = 1, ~ \mu = \frac{1}{2}\)}
\choice[correct]{\(\displaystyle c = -1, ~ \mu = \frac{3}{4}\)}
\choice{\(\displaystyle c = 1, ~ \mu = 1\)} 
\choice{\(\displaystyle c = -1, ~ \mu = \frac{5}{4}\)}
\choice{\(\displaystyle c = 1, ~ \mu = \frac{3}{2}\)}
\choice{\(\displaystyle c = -1, ~ \mu = \frac{7}{4}\)}
\end{multiplechoice}
\begin{feedback}
To compute the constant \(c\), we use the fact that the integral of a probability density function must equal \(1\), so 
\[ \mu = \int_{0}^{\infty} \left( \frac{1}{2}e^{-x} - ce^{-2x} \right) ~ dx = 1.\]
This gives the equation \[\frac{1}{2} - \frac{1}{2}c = 1,\]
which then implies that \(c = -1\). To compute the mean \(\mu\), we use the formula
\[ \mu = \int_{0}^{\infty} x \left( \frac{1}{2}e^{-x} + e^{-2x} \right) ~ dx.\]
 Calculating the integral gives \(\mu = 3/4.\)
\end{feedback}

\end{question}

\begin{question}%%%%%[ProbEVsinibp001]

A certain random variable \(X\) takes values in the interval \(\left[2 \pi , \frac{5}{2} \pi \right]\).
If the probability density function is given by
\[ A \sin {x} \]
 for some appropriate value of the constant \(A\),
compute the expected value \(\mu\) of \(X\).
\begin{multiplechoice}
\choice{\(\displaystyle \mu = -1 + 2 \pi\)}
\choice{\(\displaystyle \mu = 1 + \frac{3}{2} \pi\)}
\choice{\(\displaystyle \mu = 2 \pi\)}
\choice{\(\displaystyle \mu = -1 + \frac{5}{2} \pi\)}
\choice[correct]{\(\displaystyle \mu = 1 + 2 \pi\)}
\choice{\(\displaystyle \mu = \frac{5}{2} \pi\)}
\end{multiplechoice}
\begin{feedback}
The constant \(A\) will be the reciprocal of the integral
\[ \int_{2 \pi}^{\frac{5}{2} \pi} \sin {x}\, dx \]
 One can check that
\[ \begin{aligned} \int_{2 \pi}^{\frac{5}{2} \pi} \sin {x}\, dx & = 1. \end{aligned} \]
To compute the expected value \(\mu\) we also need to compute the integral 
\[ \int_{2 \pi}^{\frac{5}{2} \pi} x \sin {x}\, dx \]
 To compute the integralwe can use integration by parts. A reasonable strategy is to integrate  \(\sin {x}\) and differentiate  \(x\).
 This gives the equality
\[ \begin{aligned} \int x \sin {x}\, dx & = - x \cos {x} - \int \left(- \cos {x}\right)\, dx \\
 & = - x \cos {x} + \sin {x}. \end{aligned} \]
Therefore 
\[ \begin{aligned} \int_{2 \pi}^{\frac{5}{2} \pi} x \sin {x}\, dx & = \left. \left[- x \cos {x} + \sin {x} \right] \right|_{2 \pi}^{\frac{5}{2} \pi}\\ & = 1 - \left(- 2 \pi \right) = 1 + 2 \pi. \end{aligned} \]
Therefore the expected value is the ratio of the integrals, i.e.,
\[ \begin{aligned} \mu & = \frac{1 + 2 \pi}{1} = 1 + 2 \pi. \end{aligned} \]
\end{feedback}

\end{question}

\section*{Sample Exam Questions}

\begin{question}%[2019ProbImprop]
A certain random variable $X$ has values in $(1,\infty)$ and has the property that there is some constant $C$ such that
\[ P( X > a) = C \ln \frac{a^3+1}{a^3} \]
for every $a > 1$. Compute the value of $C$ and determine whether the expected value $\mu$ of $X$ is finite or infinite. [Hint: There is enough information given to compute $C$ without calculating any integrals.]
\begin{multipleChoice}
\choice{$C = \ln 2$ and $\mu < \infty$}
\choice{$C = 1$ and $\mu < \infty$} 
\choice[correct]{$C = (\ln 2)^{-1}$ and $\mu < \infty$} 
\choice{$C = \ln 2$ and $\mu = \infty$}
\choice{$C = 1$ and $\mu  = \infty$}
\choice{$C = (\ln 2)^{-1}$ and $\mu = \infty$}
\end{multipleChoice}


\begin{feedback}
We know that $X$ is always greater than one, so
\[ 1 = P(X > 1) = C \ln \frac{1+1}{1}, \]
which gives $C = (\ln 2)^{-1}$.  If we let $f(x)$ denote the probability density function of $X$, then
\[ \frac{1}{\ln 2} \ln \frac{a^3+1}{a^3} = P(X > a) = \int_a^\infty f(x) dx. \]
Differentiating both sides with respect to $a$ gives
\[ \frac{1}{\ln 2} \left[ \frac{3 a^2}{a^3+1} - \frac{3}{a} \right] = - f(a) \]
so 
\[ f(a) = \frac{1}{\ln 2} \left[ \frac{3}{a} - \frac{3a^2}{a^3+1} \right] = \frac{3}{a(a^3+1) \ln 2}. \]
The expected value of $X$ must equal
\[ \int_1^\infty \frac{3a}{a(a^3+1) \ln 2} da = \frac{3}{\ln 2} \int_1^\infty \frac{da}{a^3+1}. \]
This integral will be finite by direct comparison to the convergent integral $\int_1^\infty a^{-3} da$.
\end{feedback}
\end{question}

\begin{question}%%%%%[2015C.08]

The function 
\[ f(x) = \begin{cases} \displaystyle \frac{k}{x^3} & 1 \leq x < \infty \\ 0 & \text{otherwise} \end{cases} \]
is a probability density function for a certain value of \(k\). For that probability density function, find the probability that \(x > 2\).
\begin{multiplechoice}
\choice{\(\displaystyle \frac{1}{2}\)}
\choice{\(\displaystyle \frac{1}{3}\)}
\choice[correct]{\(\displaystyle \frac{1}{4}\)}
\choice{\(\displaystyle \frac{2}{3}\)}
\choice{\(\displaystyle \frac{1}{5}\)}
\choice{\(\displaystyle \frac{1}{6}\)}
\end{multiplechoice}

\end{question}

\begin{question}%%%%%[2016C.07]

For a certain real number \(k\), the function
\[ f(X) = \begin{cases} \displaystyle \frac{k}{X^2+1} & \text{if } X \geq 0 \\ 0 & \text{otherwise} \end{cases} \]
is a probability density function for a continuous random variable \(X\). For this value of \(k\), find the probability that \(X > 1\).
\begin{multiplechoice}
\choice{\(0\)}
\choice{\(\displaystyle \frac{1}{3}\)}
\choice{\(\displaystyle \frac{2}{3}\)}
\choice{\(1\)}
\choice[correct]{\(\displaystyle \frac{1}{2}\)}
\choice{\(\displaystyle \frac{1}{4}\)}
\end{multiplechoice}

\end{question}

\begin{question}%%%%%[2017C.09]

Let \[ f(r) = \begin{cases} C r^2 e^{-2r/b} & r \geq 0 \\ 0 & r < 0 \end{cases}. \]
Find \(C\) so that this is a probability density function (pdf) for the random variable \(r\). Here \(b\) is a positive constant. This function is used to model the distance between the nucleus and the electron in a hydrogen atom. The constant \(b\) is called the \textit{Bohr length}. Find the mean of the pdf.
\begin{multiplechoice}
\choice{\(\displaystyle C = \frac{b^3}{4}\), mean \(\displaystyle = b\)}
\choice{\(\displaystyle C = \frac{4}{b^2}\), mean \(\displaystyle = b\)}
\choice{\(\displaystyle C = \frac{4}{b}\), mean \(\displaystyle = b^2\)} 
\choice[correct]{\(\displaystyle C = \frac{4}{b^3}\), mean \(\displaystyle = \frac{3}{2}b\)}
\choice{\(\displaystyle C = \frac{4}{b^2}\), mean \(\displaystyle = \frac{3}{2}b^2\)}
\choice{\(\displaystyle C = \frac{4}{b}\), mean \(\displaystyle = \frac{3}{2}b^3\)}
\end{multiplechoice}

\end{question}


\end{document}
