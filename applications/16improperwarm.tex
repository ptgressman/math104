\documentclass{ximera}
\graphicspath{
{./}
{volumes/}
{arclengths/}
{centroids/}
{techniques/}
{applications/}
{series/}
{powerseries/}
{odes/}
}

\newcommand{\bigmath}[1]{$\displaystyle #1$}
\newcommand{\choicebreak}{}
\newenvironment{type}{}{}
\newenvironment{notes}{}{}
\newenvironment{keywords}{}{}
\newcommand{\offline}{}
\newenvironment{comments}{\begin{feedback}}{\end{feedback}}
\newenvironment{multiplechoice}{\begin{multipleChoice}}{\end{multipleChoice}}
\title{Improper Integrals}
%%%%%\author{Philip T. Gressman}

\begin{document}
\begin{abstract}
We study the concept of improper integrals.
\end{abstract}
\maketitle

\section*{(Video) Calculus: Single Variable}
\youtube{NJKB7QRQ2yY}

\section*{Online Texts}
\begin{itemize}
\item \link[OpenStax II 3.7: Improper Integrals]{https://openstax.org/books/calculus-volume-2/pages/3-7-improper-integrals}
\item \link[Ximera OSU: Improper Integrals]{https://ximera.osu.edu/mooculus/calculus2/improperIntegrals/titlePage}
\item \link[Community Calculus: Improper Integrals]{https://www.whitman.edu/mathematics/calculus_online/section09.07.html}
\end{itemize}

\section*{Examples}

\begin{example}
Compute the improper integral
\[ \int_0^1 \ln x \, dx \]
if it is convergent.
\begin{itemize}
\item First off, we note that the integral \textit{is} in fact improper because $\ln x$ has a vertical asymptote at $x = 0$.  We can find the antiderivative of $\ln x$ by integration by parts:
\[ \int \ln x \, dx = \answer{x \ln x - x} + C\]
(your answer should be the usual thing given by integration by parts. In particular, it equals $0$ at $x = e$.)
\item The correct interpretation of the improper integral is 
\[ \int_0^1 \ln x \, dx = \lim_{b \rightarrow 0^+} \int_{\answer{b}}^{\answer{1}} \ln x \, dx. \]
\item Using the antiderivative you found above (with $C = 0$), use the Fundamental Theorem of Calculus to conclude
\[ \int_0^1 \ln x \, dx = \lim_{b \rightarrow 0^+} \int_{\answer{b}}^{\answer{1}} \ln x \, dx = \lim_{b \rightarrow 0^+} \left[ \left( \answer{-1} \right) - \left(\answer{ b \ln b - b} \right) \right]. \]
\item As $b \rightarrow 0^+$, $b \rightarrow 0$ and $\ln b \rightarrow - \infty$. Using knowledge of orders of growth, we know that the product $b \ln b$
\begin{multipleChoice}
\choice{tends to $-\infty$ because $\ln b$ grows faster than any negative power of $b$ as $b \rightarrow 0^+$.}
\choice{tends to some finite constant as $b \rightarrow 0^+$ because $\ln b$ has the same growth rate as $b^{-1}$ as $b \rightarrow 0^+$.}
\choice[correct]{tends to zero because $\ln b$ grows slower than any negative power of $b$ as $b \rightarrow 0^+$.}
\end{multipleChoice}
\item We take the limit as $b \rightarrow 0^+$ to conclude
\[ \int_{0}^1 \ln x \, dx = \answer{-1} \]
(write N/A if the limit diverges).
\end{itemize}
\end{example}

\begin{example}
Without finding an antiderivative, determine whether the integral below is convergent or divergent.
\[ \int_1^\infty \frac{dx}{x - \ln x} \]
\begin{itemize}
\item The dominant term in the denominator as $x \rightarrow \infty$ will be 
\begin{multipleChoice}
\choice[correct]{$x$}
\choice{$\ln x$}
\end{multipleChoice}
which suggests doing a comparison to the integral
\begin{multipleChoice}
\choice[correct]{$\displaystyle \int_1^\infty \frac{dx}{x}$}
\choice{$\displaystyle \int_1^\infty \frac{dx}{- \ln x}$}
\end{multipleChoice}
Since $1/(x- \ln x)$ is always \wordChoice{\choice{less than}\choice[correct]{greater than}} $1/x$, the simplest test which applies is the \wordChoice{\choice[correct]{direct comparison}\choice{limit comparison}} test and it indicates \wordChoice{\choice{convergence}\choice[correct]{divergence}\choice{nothing--it does not apply}}.
\end{itemize}
\end{example}

\begin{example}
Without finding an antiderivative, determine whether the integral below is convergent or divergent.
\[ \int_0^1 \frac{dx}{2\sqrt{x} - x^2}. \]
\begin{itemize}
\item The dominant term in the denominator as $x \rightarrow 0^+$ is
\begin{multipleChoice}
\choice{$x^2$}
\choice[correct]{$2 \sqrt{x}$}
\end{multipleChoice}
which suggests doing a comparison to the integral
\begin{multipleChoice}
\choice[correct]{$\displaystyle \int_0^1 \frac{dx}{2 \sqrt{x}}$}
\choice{$\displaystyle \int_0^1 \frac{dx}{- x^2}$}
\end{multipleChoice}
Since $1/(2 \sqrt{x} -x^2)$ is always \wordChoice{\choice{less than}\choice[correct]{greater than}} $1/(2 \sqrt{x})$, the simplest test which applies is the \wordChoice{\choice{direct comparison}\choice[correct]{limit comparison}} test and it indicates \wordChoice{\choice[correct]{convergence}\choice{divergence}\choice{nothing--it does not apply}}.
\end{itemize}
\end{example}



\end{document}
