\documentclass{ximera}
\graphicspath{
{./}
{volumes/}
{arclengths/}
{centroids/}
{techniques/}
{applications/}
{series/}
{powerseries/}
{odes/}
}

\newcommand{\bigmath}[1]{$\displaystyle #1$}
\newcommand{\choicebreak}{}
\newenvironment{type}{}{}
\newenvironment{notes}{}{}
\newenvironment{keywords}{}{}
\newcommand{\offline}{}
\newenvironment{comments}{\begin{feedback}}{\end{feedback}}
\newenvironment{multiplechoice}{\begin{multipleChoice}}{\end{multipleChoice}}
\title{Exercises: Improper Integrals}
\author{Philip T. Gressman}

\begin{document}
\begin{abstract}
Various exercises relating to improper integrals.
\end{abstract}
\maketitle


\section*{Sample Quiz Questions}

\begin{question}[ImpropCD01]
\begin{type}
multiplechoice
\end{type}
Which of the following improper integrals is convergent? \offline{Show how you used comparison tests to justify your answer.}
\[ \mathrm{I}: \ \int_0^1\frac{\sqrt{{1}+{x^2}{e^{-x}}}}{{(\cos x)}{x^2}}~dx \qquad   \mathrm{II}: \ \int_{2}^\infty\frac{{e^{x}}}{{x}{e^{x}}+{x^2}}~dx \qquad  \mathrm{III}: \ \int_{2}^\infty\frac{{x^2}}{{x^4}+{1}}~dx\]
\begin{multiplechoice}
\choice{only \(\mathrm{I}\) converges}
\choice{only \(\mathrm{II}\) converges}
\choice[correct]{only \(\mathrm{III}\) converges} \choicebreak
\choice{\(\mathrm{I}\) and \(\mathrm{II}\) converge}
\choice{\(\mathrm{II}\) and \(\mathrm{III}\) converge}
\choice{\(\mathrm{I}\) and \(\mathrm{III}\) converge}
\end{multiplechoice}
\begin{feedback}
Integral \(\mathrm{I}\) is divergent by direct comparison to the function \(\displaystyle \frac{{1}}{{x^2}}\). 
Integral \(\mathrm{II}\) is divergent by limit comparison to the function \(\displaystyle \frac{{1}}{{x}}\). 
Integral \(\mathrm{III}\) is convergent by direct comparison to the function \(\displaystyle \frac{{1}}{{x^2}}\).
\end{feedback}
\begin{keywords}
improper integrals,improper integral convergence, generation2
\end{keywords}
\end{question}

\begin{question}[ImpropCD08]
\begin{type}
multiplechoice
\end{type}
Which of the following improper integrals is convergent? \offline{Show how you used comparison tests to justify your answer.}
\[ \mathrm{I}: \ \int_0^1\frac{\sqrt{{e^{2x}}+{x^3}}}{{x}}~dx \qquad   \mathrm{II}: \ \int_0^1\frac{{x^2}}{{x^2}{\sqrt{x}}+{x^3}}~dx \qquad  \mathrm{III}: \ \int_{2}^\infty\frac{{x^2}{\ln x}}{-{x}+{x^4}}~dx\]
\begin{multiplechoice}
\choice{only \(\mathrm{I}\) converges}
\choice{only \(\mathrm{II}\) converges}
\choice{only \(\mathrm{III}\) converges} \choicebreak
\choice{\(\mathrm{I}\) and \(\mathrm{II}\) converge}
\choice[correct]{\(\mathrm{II}\) and \(\mathrm{III}\) converge}
\choice{\(\mathrm{I}\) and \(\mathrm{III}\) converge}
\end{multiplechoice}
\begin{feedback}
Integral \(\mathrm{I}\) is divergent by direct comparison to the function \(\displaystyle \frac{{1}}{{x}}\). 
Integral \(\mathrm{II}\) is convergent by direct comparison to the function \(\displaystyle \frac{{1}}{{\sqrt{x}}}\). 
Integral \(\mathrm{III}\) is convergent by limit comparison to the function \(\displaystyle \frac{{\ln x}}{{x^2}}\).
\end{feedback}
\begin{keywords}
improper integrals,improper integral convergence, generation2
\end{keywords}
\end{question}

\section*{Sample Exam Questions}

\begin{question}[2016C.06]
\begin{type}
multiplechoice
\end{type}
Only one of the following four improper integrals diverges. Choose that improper integral\offline{ and justify why it diverges}. \offline{(You need NOT justify why the other integrals converge.)}
\begin{multiplechoice}
\choice{\(\displaystyle \int^{\infty}_{2} \frac{\arctan x}{1+x^3} dx\)}
\choice{\(\displaystyle \int^{\infty}_{2} \frac{1}{\sqrt{x^4+x^2}} dx\)}
\choice{\(\displaystyle \int^{\infty}_{2} \frac{1+\sin x}{x^2} dx\)}
\choice[correct]{\(\displaystyle \int_{2}^{\infty} \frac{1}{\sqrt[3]{x^2-1}} dx\)}
\end{multiplechoice}
\begin{notes}
same question as 2018.S.16
\end{notes}
\begin{keywords}
handwritten,improper integrals
\end{keywords}
\end{question}


\end{document}
