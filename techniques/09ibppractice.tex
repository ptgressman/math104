\documentclass{ximera}
\graphicspath{
{./}
{volumes/}
{arclengths/}
{centroids/}
{techniques/}
{applications/}
{series/}
{powerseries/}
{odes/}
{lessons/}
}
\usepackage{booktabs}

\newcommand{\bigmath}[1]{$\displaystyle #1$}
\newcommand{\choicebreak}{}
\newenvironment{type}{}{}
\newenvironment{notes}{}{}
\newenvironment{keywords}{}{}
\newcommand{\offline}{}
\newenvironment{comments}{\begin{feedback}}{\end{feedback}}
\newenvironment{multiplechoice}{\begin{multipleChoice}}{\end{multipleChoice}}
\title{Exercises: Integration by Parts}
\author{Philip T. Gressman}

\begin{document}
\begin{abstract}
Various exercises relating to integration by parts.
\end{abstract}
\maketitle


Compute the indefinite integrals below. Since there are many possible answers (which differ by constant values), use the given instructions if needed to choose which possible answer to use.

\begin{exercise}%[APEX0602IBP05]
\[  \int x\sin x\ dx = \answer{\sin x - x\cos x} +C\]
(Add a constant to your answer if needed so that it equals $0$ at $x = 0$.)
%
%
\end{exercise}

\begin{exercise}%[APEX0602IBP06]
\[ \int xe^{-x}\ dx = \answer{-e^{-x}-xe^{-x}}+C\]
(Add a constant to your answer if needed so that it equals $-1$ at $x = 0$.)
%
%
\end{exercise}

\begin{exercise}%[APEX0602IBP10]
\[  \int x^3e^{x}\ dx\ = \answer{x^3e^x-3x^2e^x+6xe^x-6e^x}+C\]
(Write your answer so that it has no constant term.)
%
%
\end{exercise}

\begin{exercise}%[APEX0602IBP25]
\[ \int x^2\ln |x| \ dx = \answer{\frac{1}{3} x^3 \ln |x|-\frac{x^3}{9}}+C\]
Don't forget absolute values in your logarithm.
(Add a constant to your answer as necessary so that it equals $-1/9$ at $x = 1$.)
%
%
\end{exercise}

\begin{exercise}%[APEX0602IBP12]
\[ \int e^x\sin x\ dx = \answer{\frac{1}{2} e^x(\sin x-\cos x)}+C\]
(Add a constant to your answer if needed so that it equals $-1/2$ at $x = 0$.)
\begin{hint}
This is a case in which you need to treat the integration by parts as an equation and \textit{solve} for the answer.
\end{hint}
%
%
\end{exercise}

\begin{exercise}%[APEX0602IBP17]
\[ \int \arcsin x\ dx = \answer{\sqrt{1-x^2}+x \arcsin(x)}+C \]
(Add a constant to your answer if needed so that it equals $1$ at $x = 0$.)
\begin{hint}
Write $\arcsin x = 1 \cdot \arcsin x$.
\end{hint}
%
%
\end{exercise}

\section*{Sample Quiz Questions}
\begin{question}%%%%%[IBPibp:explin1413]

Compute the definite integral 
\[\int_1^4 e^{3x} (x + 1) ~ dx.\]
\begin{multiplechoice}
\choice[correct]{\(\displaystyle \frac{14}{9} e^{12} - \frac{5}{9} e^{3}\)}
\choice{\(\displaystyle \frac{17}{9} e^{12} - \frac{5}{9} e^{3}\)}
\choice{\(\displaystyle \frac{17}{9} e^{12} - \frac{8}{9} e^{3}\)}
\choice{\(\displaystyle \frac{20}{9} e^{12} - \frac{8}{9} e^{3}\)}
\choice{\(\displaystyle \frac{20}{9} e^{12} - \frac{11}{9} e^{3}\)}
\choice{\(\displaystyle \frac{23}{9} e^{12} - \frac{11}{9} e^{3}\)}
\end{multiplechoice}
\begin{feedback}
Integrate by parts, integrating the exponential and differentiating polynomials.
\[\begin{aligned}
    \int_1^4 e^{3x} &  (x + 1) ~ dx \\
    & = \left. \frac{e^{3x}}{3} (x+1) \right|_1^4 - \int_1^4 \frac{e^{3x}}{3} dx \\
    & = \frac{5}{3} e^{12} - \frac{2}{3} e^{3} - \left. \frac{e^{3x}}{9} \right|_1^4 \\
    & = \frac{14}{9} e^{12} - \frac{5}{9} e^{3}
\end{aligned}\]
\end{feedback}

\end{question}
\begin{question}%%%%%[IBPibp:fourier2alt4]

Compute the definite integral 
\[\int_{\pi}^{2 \pi} x \sin 4 x ~ dx.\]
\begin{multiplechoice}
\choice{\(\displaystyle 0\)}
\choice{\(\displaystyle \frac{\pi}{4}\)}
\choice[correct]{\(\displaystyle -\frac{\pi}{4}\)}
\choice{\(\displaystyle \frac{3 \pi}{4}\)}
\choice{\(\displaystyle -\frac{3 \pi}{4}\)}
\choice{\(\displaystyle \frac{7 \pi}{4}\)}
\end{multiplechoice}
\begin{feedback}
Integrate by parts, integrating the trig functions and differentiating polynomials.
\[ \begin{aligned}
    \int_{\pi}^{2 \pi} x & \sin 4 x ~ dx \\
    & = \left. - \frac{\cos 4x}{4} x \right|_{\pi}^{2 \pi} + \int_{\pi}^{2 \pi} \frac{\cos 4x}{4} dx \\
    & = - \frac{2 \pi }{4} + \frac{(-1)^{4} \pi }{4} - \left. \frac{\sin 4x}{16} \right|_{\pi}^{2 \pi} = -\frac{\pi}{4}
\end{aligned}\]
\end{feedback}

\end{question}
\begin{question}%%%%%[IBPibp:miscarctan5]

Compute the indefinite integral 
\[\int \arctan 5x ~ dx.\]
\begin{multiplechoice}
\choice[correct]{\(\displaystyle x \arctan 5x - \frac{1}{10} \ln | 1 + 25x^2|  + C\)}
\choice{\(\displaystyle x \arctan 5x - \frac{1}{12} \ln | 1 + 25x^2|  + C\)}
\choice{\(\displaystyle x \arctan 5x - \frac{1}{14} \ln | 1 + 25x^2|  + C\)}
\choice{\(\displaystyle x \arctan 5x + \frac{1}{10} \ln | 1 + 25x^2|  + C\)}
\choice{\(\displaystyle x \arctan 5x + \frac{1}{12} \ln | 1 + 25x^2|  + C\)}
\choice{\(\displaystyle x \arctan 5x + \frac{1}{14} \ln | 1 + 25x^2|  + C\)}
\end{multiplechoice}
\begin{feedback}
Integrate by parts, integrating the coefficient 1 and differentiating arctangent.
\[\begin{aligned}
    \int & \arctan 5x ~ dx \\
    & = x \arctan 5x - \int \frac{5x}{1+25x^2} ~ dx \\
    & = x \arctan 5x - \frac{1}{10} \int \frac{50x}{1+25x^2} ~ dx \\
    & = x \arctan 5x - \frac{1}{10} \ln | 1 + 25x^2|  + C
\end{aligned}\]
\end{feedback}

\end{question}
\begin{question}%%%%%[IBPibp:expcos35]

Compute the indefinite integral 
\[\int e^{3x}  \cos 5x ~ dx.\]
\begin{multiplechoice}
\choice{\(\displaystyle \frac{e^{3x} (2 \cos 5x + 5 \sin 5x)}{29} + C\)}
\choice[correct]{\(\displaystyle \frac{e^{3x} (3 \cos 5x + 5 \sin 5x)}{34} + C\)}
\choice{\(\displaystyle \frac{e^{3x} ( \cos 5x + 3 \sin 5x)}{20} + C\)}
\choice{\(\displaystyle \frac{e^{3x} ( \cos 5x + 2 \sin 5x)}{15} + C\)}
\choice{\(\displaystyle \frac{e^{3x} (2 \cos 5x + 7 \sin 5x)}{53} + C\)}
\choice{\(\displaystyle \frac{e^{3x} (3 \cos 5x + 7 \sin 5x)}{58} + C\)}
\end{multiplechoice}
\begin{feedback}
Integrate by parts, integrating the exponential and differentiating cosine.
\[ \begin{aligned}
    \int e^{3x} & \cos 5x ~ dx \\
    & = \frac{e^{3x}}{3} \cos 5x - \int \frac{e^{3x}}{3} (-5 \sin 5x) ~ dx \\
    & = \frac{e^{3x} \cos 5x}{3} + \frac{5}{3} \int e^{3x} \sin 5x ~ dx \\
    & = \frac{e^{3x} \cos 5x}{3} + \frac{5}{3} \frac{e^{3x}}{3} \sin 5x - \frac{5}{3} \int \frac{e^{3x}}{3} (5 \cos 5x) ~ dx \\
    & = \frac{e^{3x} (3 \cos 5x + 5 \sin 5x)}{9} - \frac{25}{9} \int e^{3x} \cos 5x ~ dx  \\
    \Rightarrow   & \qquad \frac{34}{9}  \int e^{3x}  \cos 5x ~ dx  = \frac{e^{3x} (3 \cos 5x + 5 \sin 5x)}{9} \\
    \Rightarrow & \qquad  \int  e^{3x}  \cos 5x ~ dx = \frac{e^{3x} (3 \cos 5x + 5 \sin 5x)}{34} + C
\end{aligned}\]
\end{feedback}

\end{question}

\section*{Sample Exam Questions}

\begin{question}%%%%%[2015C.11]

Compute the integral below.
\[ \int_{\frac{1}{2}}^\infty \frac{ \ln (2x)}{x^2} dx \]
\begin{multiplechoice}
\choice{\(1 - \ln 2\)}
\choice[correct]{\(2\)}
\choice{\(\displaystyle \ln 2 - \frac{1}{2}\)}
\choice{\(\displaystyle \frac{1}{2}\)}
\choice{\(2 - 2 \ln 2\)}
\choice{the integral diverges}
\end{multiplechoice}

\end{question}

\begin{question}%%%%%[2016C.05]

Compute the indefinite integral indicated below. [Hint: Write \(\displaystyle \frac{1}{\cos^2 \theta} = \sec^2 \theta\) and integrate by parts.]
\[ \int \left( 1 + \frac{\ln | \sin \theta|}{\cos^2 \theta} \right) d \theta \]
\begin{multiplechoice}
\choice{\(\displaystyle (\sin \theta) \ln |\sin \theta| + C\)}
\choice{\(\displaystyle (\cos \theta) \ln |\sin \theta| + C\)}
\choice[correct]{\(\displaystyle (\tan \theta) \ln |\sin \theta| + C\)} 
\choice{\(\displaystyle (\csc \theta) \ln |\sin \theta| + C\)}
\choice{\(\displaystyle (\sec \theta) \ln |\sin \theta| + C\)}
\choice{\(\displaystyle (\cot \theta) \ln |\sin \theta| + C\)}
\end{multiplechoice}

\end{question}

\end{document}
