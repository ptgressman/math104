\documentclass{ximera}
\graphicspath{
{./}
{volumes/}
{arclengths/}
{centroids/}
{techniques/}
{applications/}
{series/}
{powerseries/}
{odes/}
{lessons/}
}
\usepackage{booktabs}

\newcommand{\bigmath}[1]{$\displaystyle #1$}
\newcommand{\choicebreak}{}
\newenvironment{type}{}{}
\newenvironment{notes}{}{}
\newenvironment{keywords}{}{}
\newcommand{\offline}{}
\newenvironment{comments}{\begin{feedback}}{\end{feedback}}
\newenvironment{multiplechoice}{\begin{multipleChoice}}{\end{multipleChoice}}
\title{Exercises: Partial Fractions}
\author{Philip T. Gressman}

\begin{document}
\begin{abstract}
Various exercises relating to partial fractions and integration.
\end{abstract}
\maketitle


Compute the indefinite integrals below. Since there are many possible answers (which differ by constant values), use the given instructions if needed to choose which possible answer to use. Do not forget absolute value signs inside logarithms when they are needed.

\begin{exercise}%[APEX0604PF07]
\[ \frac{7x+7}{x^2+3x-10} = \frac{\answer{3}}{x-2} + \frac{\answer{4}}{x+5} \]
\[ \int \frac{7x+7}{x^2+3x-10}\ dx = \answer{3 \ln |x-2|+4 \ln |x+5|}+C\]
(Do not include any constant terms in your antiderivative.)
%%
%%
\end{exercise}

\begin{exercise}%[APEX0604PF08]
\[ \int \frac{7x-2}{x^2+x}\ dx = \answer{9 \ln |x+1|-2 \ln |x|}+C\]
(Do not include any constant terms in your antiderivative.)
%
%
\end{exercise}

\begin{exercise}%[APEX0604PF11]
\[ \frac{x+7}{(x+5)^2} = \frac{\answer{1}}{x+5} + \frac{\answer{2}}{(x+5)^2} \]
\[ \int \frac{x+7}{(x+5)^2}\ dx = \answer{\ln |x+5|-\frac{2}{x+5}}+C\]
(Do not include any constant terms in your antiderivative.)
%
%
\end{exercise}


\begin{exercise}%[APEX0604PF13]
\[ \int \frac{9x^2+11x+7}{x(x+1)^2}\ dx = \answer{\frac{5}{x+1}+7 \ln |x|+2 \ln |x+1|}+C\]
(Do not include any constant terms in your answer.)
%
%
\end{exercise}

\begin{exercise}%[APEX0604PF16]
\[ \int \frac{x^2+x+1}{x^2+x-2}\ dx = \answer{x+\ln |x-1|-\ln |x+2|}+C \]
(Do not include any constant terms in your answer.)
\begin{hint}
Don't forget polynomial long division; it is needed in this case because the degree of the numerator is at least as large as the degree of the denominator.
\end{hint}
%
%
\end{exercise}

\begin{exercise}[APEX0604PF20]
\[ \int \frac{x^2+x+5}{x^2+4x+10}\ dx = \answer{-\frac{3}{2} \ln \left|x^2+4 x+10\right|+x+\frac{\arctan \left(\frac{x+2}{\sqrt{6}}\right)}{\sqrt{6}}}+C \]
(Do not include any constant terms in your answer.)
\begin{hint}
\[ \frac{x^2+x+5}{x^2+4x+10} = \answer{1} + \frac{\answer{-3} x + \answer{-5}}{x^{2} + 4 x + 10}  \]
Since the derivative of the denominator is $2x + 4$, we should rewrite the numerator of the big fraction to have $x+2$'s if possible:
\[  \frac{x^2+x+5}{x^2+4x+10} = \answer{1} + \frac{\answer{-3}(x+2) + \answer{1}}{x^2 + 4x + 10}. \]
For expressions like
\[ \int \frac{x+2}{x^2 +4x+10} dx \]
we should do a substitution. For terms like
\[ \int \frac{1}{x^2 + 4x + 10} dx \]
we should first complete the square: $x^2 + 4x + 10 =  (x + 2)^2 + 6$ and then make the substitution $x + 2 = u \sqrt{6}$.
\end{hint}
%
%
\end{exercise}

\begin{exercise}%[APEX0604PF23]
\[ \int \frac{2x^2+x+1}{(x+1)(x^2+9)}\ dx = \answer{\frac{9}{10} \ln \left|x^2+9\right|+\frac{1}{5} \ln |x+1|-\frac{4}{15} \arctan \left(\frac{x}{3}\right)} +C\]
(Do not include any constant terms in your answer.)
%
%
\end{exercise}

\section*{Sample Quiz Questions}

\begin{question}%%%%%[PartialFracHeav001v2]

Compute the integral
\[\int_{3}^{4}\frac{2x-3}{x^2-3x+2}~dx.\]
\begin{multiplechoice}
\choice{\(\displaystyle \ln 2\)}
\choice[correct]{\(\displaystyle \ln 3\)}
\choice{\(\displaystyle \ln 4\)}
\choice{\(\displaystyle \ln 5\)}
\choice{\(\displaystyle \ln 6\)}
\choice{\(\displaystyle \ln 7\)}
\end{multiplechoice}
\begin{feedback}
First factor the denominator of the integrand: \(x^2-3x+2 = (x-1)(x-2)\). Since the roots are distinct, it is possible to use the Heaviside cover-up method.  The partial fractions expansion will take the form \[\frac{A}{x-1} + \frac{B}{x-2}, \] where the coefficient \(A\) can be computed by cancelling the factor of \(x-1\) in the denominator and evaluating the result at \(x = 1\), i.e., \[A = \frac{2(1)-3}{(1)-2} = 1. \] Similarly, \[B = \frac{2(2)-3}{(2)-1} = 1,\] which gives that \[\frac{2x-3}{(x-1)(x-2)} = \frac{1}{x-1} + \frac{1}{x-2}. \] Therefore
\[ \begin{aligned} 
\int_{3}^{4}\frac{2x-3}{x^2-3x+2}~dx & = \int_{3}^{4}\left(\frac{1}{x-1} + \frac{1}{x-2}\right)~dx \\
 & = \left(\ln |4-1| + \ln |4-2| \right) - \left(\ln |3-1| + \ln |3-2| \right)\\
 & = \ln 3 + \ln 2 + \ln \frac{1}{2} + 0 = \ln 3.\end{aligned}\]
\end{feedback}

\end{question}

\section*{Sample Exam Questions}

\begin{question}%%%%%[2019WrittenByClass01]
Compute the volume of the solid of revolution obtained by revolving around the $y$-axis the region below the graph
\[ y = \frac{1}{(x-1)^2}, \]
above $y=0$, and between $x=2$ and $x=3$.
\begin{multiplechoice}
\choice{\(\displaystyle \pi\)}
\choice{\(\displaystyle \pi ( \ln 2 + 3)\)}
\choice[correct]{\(\displaystyle \pi (2 \ln 2 + 1)\)}
\choice{\(\displaystyle \pi (2 \ln 3 + 1)\)}
\choice{\(\displaystyle \pi (3 \ln 2 + 1)\)}
\choice{\(\displaystyle \pi (3 \ln 3 + 1)\)}
\end{multiplechoice}
\begin{feedback}
Choosing $x$ as the variable of integration, slices will be parallel to the $y$-axis, indicating that the shell method should be used. The radius of a shell is $x$ (because the axis lies to the left of the region) and the height will be $(x-1)^{-2}$, so 
\[ V = \int_2^3 \frac{2 \pi x}{(x-1)^2} dx = 2 \pi \int_2^3 \frac{x}{(x-1)^2} dx. \]
The integral can be computed by partial fractions; the expansion has the form
\[ \frac{x}{(x-1)^2} = \frac{A}{x-1} + \frac{B}{(x-1)^2}. \]
The coefficients $A$ and $B$ can be found by usual methods (but note that the Heaviside cover up method will \textit{not} work in this case), but it is also possible to find them directly by carefully rewriting the numerator of the fraction in terms of $x-1$:
\[ \frac{x}{(x-1)^2} = \frac{(x-1) + 1}{(x-1)^2} = \frac{(x-1)}{(x-1)^2} + \frac{1}{(x-1)^2} = \frac{1}{x-1} + \frac{1}{(x-1)^2}. \]
Therefore
\[ \begin{aligned} V & = 2 \pi \int_2^3 \left[ \frac{1}{x-1} + \frac{1}{(x-1)^2} \right] dx = 2 \pi \left. \left[ \ln |x-1| - \frac{1}{x-1} \right] \right|_{2}^3 \\
& = 2 \pi \left( \ln 2 - \frac{1}{2} \right) - 2 \pi \left( 0 - 1 \right) = \pi(2 \ln 2 + 1). \end{aligned}\]
\end{feedback}

\end{question}

\begin{question}%%%%%[2016C.04]

Compute the constants \(A\) and \(B\) in the partial fractions expansion indicated below. \offline{To receive full credit, it is not necessary to compute \(C, D,\) or \(E\).}
\[ \frac{x^4 + 16}{x^4 - 16} =A +  \frac{B}{x-2} + \frac{C}{x+2} + \frac{Dx + E}{x^2 + 4} \]
\begin{multiplechoice}
\choice{\(A=-1, B=1\)}
\choice{\(A = 0, B = 1\)}
\choice[correct]{\(A = 1, B = 1\)} 
\choice{\(A=-1, B=-1\)}
\choice{\(A = 0, B = -1\)}
\choice{\(A = 1, B = -1\)}
\end{multiplechoice}
\begin{comments}
\[  \frac{x^4 + 16}{x^4 - 16} = 1 + \frac{1}{x-2} - \frac{1}{x+2} - \frac{4}{x^2+4} \]
\end{comments}


\end{question}

\begin{question}%%%%%[2017C.04]

Evaluate \(\displaystyle \int_1^2 \frac{x^2+x+1}{x^2+x} dx\).
\begin{multiplechoice}
\choice{\(0\)}
\choice{\(1\)}
\choice[correct]{\(\displaystyle 1 + \ln \left(\frac{4}{3}\right)\)}
\choice{\(2\)}
\choice{\(\displaystyle 2 + \ln \left(\frac{8}{3}\right)\)}
\choice{none of these}
\end{multiplechoice}

\end{question}




\end{document}
