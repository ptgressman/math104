\documentclass{ximera}
\graphicspath{
{./}
{volumes/}
{arclengths/}
{centroids/}
{techniques/}
{applications/}
{series/}
{powerseries/}
{odes/}
{lessons/}
}
\usepackage{booktabs}

\newcommand{\bigmath}[1]{$\displaystyle #1$}
\newcommand{\choicebreak}{}
\newenvironment{type}{}{}
\newenvironment{notes}{}{}
\newenvironment{keywords}{}{}
\newcommand{\offline}{}
\newenvironment{comments}{\begin{feedback}}{\end{feedback}}
\newenvironment{multiplechoice}{\begin{multipleChoice}}{\end{multipleChoice}}
\title{Exercises: Trigonometric Substitions}
\author{Philip T. Gressman}

\begin{document}
\begin{abstract}
Various exercises relating to trigonometric substitutions.
\end{abstract}
\maketitle


Compute the indefinite integrals below. Since there are many possible answers (which differ by constant values), use the given instructions if needed to choose which possible answer to use. Do not forget absolute value signs inside logarithms when they are needed.

\begin{exercise}%[APEX0608TRIGSB20]
\[ \int x^2\sqrt{1-x^2}\ dx = \answer{\frac{1}{8}\arcsin x-\frac{1}{8}x\sqrt{1-x^2}(1-2x^2)}+C\] 
(Choose your answer to equal $0$ at $x = 0$.)
%
%
\end{exercise}

\begin{exercise}%[APEX0608TRIGSB18]
\[  \int \frac {1}{(x^2+1)^2}\ dx = \answer{\frac{1}{2}\left(\arctan x+\frac{x}{x^2+1}\right)} +C\]
(Choose your answer to equal $0$ at $x = 0$.)
%
%
\end{exercise}

\begin{exercise}%[APEX0608TRIGSB26]
\[  \int \frac{x^2}{\sqrt{x^2+3}}\ dx = \answer{\frac{1}{2}x\sqrt{x^2+3}-\frac{3}{2}\ln\left|\frac{\sqrt{x^2+3}}{\sqrt{3}}+\frac{x}{\sqrt{3}}\right|} +C\]
(Choose your answer to equal $0$ at $x = 0$.)
%
%
\end{exercise}

\section*{Sample Quiz Questions}
\begin{question}%%%%%[TrigSub001]

Compute the integral 
\[\int_{-2}^{2}\frac{5}{(5-x^2)^{3/2}}~dx.\]
\begin{multiplechoice}
\choice{\(\displaystyle 2\)}
\choice{\(\displaystyle 3\)}
\choice[correct]{\(\displaystyle 4\)}
\choice{\(\displaystyle 5\)}
\choice{\(\displaystyle 6\)}
\choice{\(\displaystyle 7\)}
\end{multiplechoice}
\begin{feedback}
Begin by making the trig substitution \(x=\sqrt{5}\sin \theta\). \begin{hint} It follows that 
\[ \begin{aligned} \int\frac{5}{(5-x^2)^{3/2}}~dx & = \int \frac{5}{(5-(\sqrt{5}\sin \theta)^2)^{3/2}} \cdot (\sqrt{5}\cos \theta)~d \theta \\
 & = \int (\cos \theta)^{-2}~d \theta \\ & = \int (\sec \theta)^{2} ~ d \theta = (\tan \theta) + C. \end{aligned} \] \begin{hint}
To finish, use the inversion identity \[\tan \theta = \frac{x}{\sqrt{5-x^2}}.\]
Therefore \[\int_{-2}^{2}\frac{5}{(5-x^2)^{3/2}}~dx = \left.\frac{x}{\sqrt{5-x^2}}\right|_{-2}^{2} = \left(2\right) - \left(-2\right) = 4.\] \end{hint} \end{hint}
\end{feedback}

\end{question}

\begin{question}%%%%%[TrigSub017]

Compute the integral 
\[\int_{-1}^{1}\frac{3}{(3+x^2)^{3/2}}~dx.\]
\begin{multiplechoice}
\choice{\(\displaystyle \frac{1}{2}\)}
\choice[correct]{\(\displaystyle 1\)}
\choice{\(\displaystyle \frac{3}{2}\)}
\choice{\(\displaystyle 2\)}
\choice{\(\displaystyle \frac{5}{2}\)}
\choice{\(\displaystyle 3\)}
\end{multiplechoice}
\begin{feedback}
Begin by making the trig substitution \(x=\sqrt{3}\tan \theta\). \begin{hint} It follows that 
\[ \begin{aligned} \int\frac{3}{(3+x^2)^{3/2}}~dx & = \int \frac{3}{(3+(\sqrt{3}\tan \theta)^2)^{3/2}} \cdot (\sqrt{3}\sec^2 \theta)~d \theta \\
 & = \int (\sec \theta)^{-1}~d \theta \\ & = \int (\cos \theta) ~ d \theta = (\sin \theta) + C. \end{aligned} \] \begin{hint}
To finish, use the inversion identity \[\cos \theta = \frac{x}{\sqrt{3+x^2}}.\]
Therefore \[\int_{-1}^{1}\frac{3}{(3+x^2)^{3/2}}~dx = \left.\frac{x}{\sqrt{3+x^2}}\right|_{-1}^{1} = \left(\frac{1}{2}\right) - \left(-\frac{1}{2}\right) = 1.\] \end{hint} \end{hint}
\end{feedback}

\end{question}

\begin{question}%%%%%[TrigSub033]

Compute the integral 
\[\int_{4}^{5}\frac{16\sqrt{x^2-16}}{x^{4}}~dx.\]
\begin{multiplechoice}
\choice{\(\displaystyle \frac{4}{125}\)}
\choice{\(\displaystyle \frac{1}{25}\)}
\choice{\(\displaystyle \frac{6}{125}\)}
\choice{\(\displaystyle \frac{7}{125}\)}
\choice{\(\displaystyle \frac{8}{125}\)}
\choice[correct]{\(\displaystyle \frac{9}{125}\)}
\end{multiplechoice}
\begin{feedback}
Begin by making the trig substitution \(x=4\sec \theta\). \begin{hint} It follows that 
\[ \begin{aligned} \int\frac{16\sqrt{x^2-16}}{x^{4}}~dx & = \int \frac{16\sqrt{(4\sec \theta)^2-16}}{(4\sec \theta)^{4}} \cdot (4\sec \theta\tan \theta)~d \theta \\
 & = \int (\sec \theta)^{-3}(\tan \theta)^{2}~d \theta \\ & = \int (\sin \theta)^{2}(\cos \theta) ~ d \theta = \frac{1}{3}(\sin \theta)^{3} + C. \end{aligned} \] \begin{hint}
To finish, use the inversion identity \[\sin \theta = \frac{\sqrt{x^2-16}}{x}.\]
Therefore \[\int_{4}^{5}\frac{16\sqrt{x^2-16}}{x^{4}}~dx = \left.\frac{1}{3}\frac{(x^2-16)^{3/2}}{x^{3}}\right|_{4}^{5} = \left(\frac{9}{125}\right) - \left(0\right) = \frac{9}{125}.\] \end{hint} \end{hint}
\end{feedback}

\end{question}



\section*{Sample Exam Questions}

\begin{question}%%%%%[2016C.03]

Compute the value of the integral below.
\[ \int_0^{\frac{1}{\sqrt{2}}} \frac{1}{(1-x^2)^{\frac{3}{2}}} dx \]
\begin{multiplechoice}
\choice{\(0\)}
\choice[correct]{\(1\)}
\choice{\(2\)}
\choice{\(3\)}
\choice{\(4\)}
\choice{none of these}
\end{multiplechoice}

\end{question}

\begin{question}%%%%%[2017C.06]

Evaluate \(\displaystyle \int_0^3 \frac{dx}{(25-x^2)^{3/2}}\).
\begin{multiplechoice}
\choice{\(0\)}
\choice{\(\displaystyle \frac{1}{100}\)}
\choice[correct]{\(\displaystyle \frac{3}{100}\)}
\choice{\(\displaystyle \frac{5}{100}\)}
\choice{\(\displaystyle \frac{7}{100}\)}
\choice{none of these}
\end{multiplechoice}

\end{question}


\end{document}
