\documentclass{ximera}
\graphicspath{
{./}
{volumes/}
{arclengths/}
{centroids/}
{techniques/}
{applications/}
{series/}
{powerseries/}
{odes/}
}

\newcommand{\bigmath}[1]{$\displaystyle #1$}
\newcommand{\choicebreak}{}
\newenvironment{type}{}{}
\newenvironment{notes}{}{}
\newenvironment{keywords}{}{}
\newcommand{\offline}{}
\newenvironment{comments}{\begin{feedback}}{\end{feedback}}
\newenvironment{multiplechoice}{\begin{multipleChoice}}{\end{multipleChoice}}
\title{Trigonometric Integrals}
%%%%%\author{Philip T. Gressman}

\begin{document}
\begin{abstract}
We learn various techniques for integrating certain combinations of trigonometric functions.
\end{abstract}
\maketitle

\section*{Online Texts}
\begin{itemize}
\item \link[OpenStax II 3.2: Trigonometric Integrals]{https://openstax.org/books/calculus-volume-2/pages/3-2-trigonometric-integrals}
\item \link[Ximera OSU: Trigonometric Integrals]{https://ximera.osu.edu/mooculus/calculus2/trigonometricIntegrals/titlePage}
\item \link[Community Calculus 8.2: Powers of Sine and Cosine]{https://www.whitman.edu/mathematics/calculus_online/section08.02.html}
\end{itemize}

\section*{Examples}

\begin{example}
Compute the indefinite integrals
\[ \int \cos^5 3x \sin^2 3x \, dx \ \text{ and } \ \ \int \cos^2 3x \sin^5 3x \, dx. \]
\begin{itemize}
\item When dealing with products of sines and cosines of the same quantity (in this case, the $3x$ is the same inside both the sine and the cosine), we can use a substitution. We look for one with an odd power and build a substitution using the other.   In this case, we would use the substitution \wordChoice{\choice{$u = \cos 3x$}\choice[correct]{$u = \sin 3x$}} in the first integral and \wordChoice{\choice[correct]{$u = \cos 3x$}\choice{$u = \sin 3x$}} in the second.
\item Using the substitutions just identified, we have $du = \answer{3 \cos 3x} \, dx$ in the former case and $du = \answer{-3 \sin 3x}\, dx$ in the latter. This means the integrals become
\[ \frac{1}{3} \int \cos^4 3x \sin^2 3x \, du \ \ \text{ and } \ \ - \frac{1}{3} \int \cos^2 3x \sin^4 3x \, du. \]
\item We continue to simplify, using the trig identity $\cos^2 \theta + \sin^2 \theta = 1$ to completely eliminate all references to the variable $x$ in the integrand:
\[ \frac{1}{3} \int (\answer{1-u^2})^2 u^2 \, du \ \ \text{ and } \ \ - \frac{1}{3} \int u^2 (\answer{1-u^2})^2 \, du. \]
\item Now we calculate the integral, giving 
\[ \frac{1}{3} \int (\answer{1-u^2})^2 u^2 \, du = \answer{\frac{u^3}{9} - \frac{2 u^5}{15} + \frac{u^7}{21}} + C \]
in the first case and
\[ -\frac{1}{3} \int u^2 (\answer{1-u^2})^2 \, du = \answer{-\frac{u^3}{9} + \frac{2 u^5}{15} - \frac{u^7}{21}} + C \]
in the second.
\item To conclude, we reverse the substitution, so that
\[  \int \cos^5 3x \sin^2 3x \, dx = \answer{\frac{(\sin 3x)^3}{9} - \frac{2 (\sin 3x)^5}{15} + \frac{(\sin 3x)^7}{21}} + C \]
and
\[ \int \cos^2 3x \sin^5 3x \, dx = \answer{-\frac{(\cos 3x)^3}{9} + \frac{2 (\cos 3x)^5}{15} - \frac{(\cos 3x)^7}{21}} + C. \]
\end{itemize}
\end{example}

\begin{example}
Compute the indefinite integral
\[ \int \sin^2 3x \cos^2 3x \, dx. \]
\begin{itemize}
\item When both powers are even, your only option is to use a trigonometric identity to reduce the power. In this case, the identities are
\[ \sin^2 \theta = \frac{1 - \cos 2 \theta}{2}  \ \ \text{ and } \ \ \cos^2 \theta = \frac{1 + \cos 2 \theta}{2}. \]
\item 
This means that 
\[ \sin^2 3x \cos^2 3 x = \frac{1}{4} - \answer{ \frac{(\cos 6 x)^2}{4}}. \]
\item
Because we still have only even powers, we should use power reduction again:
\[ \sin^2 3x \cos^2 3x = \frac{1}{4} - \frac{1}{4} \left[ \frac{1}{2} + \answer{\frac{1}{2} \cos 12x} \right] = \frac{1}{8} - \frac{1}{8} \answer{\cos 12x}. \]
\item
Integrating this last expression gives
\[ \int \sin^2 3x \cos^2 3x \, dx = \answer{\frac{x}{8} - \frac{1}{96} \sin 12x} + C. \]
\end{itemize}
\end{example}

\begin{example}
Compute the indefinite integrals
\[ \int \sec^2 3x \tan^2 3x \, dx \ \ \text{ and } \ \ \int \sec^3 3x \tan^3 3x \, dx. \]
\begin{itemize}
\item In this case we look for either an even power of $secant$, which indicates a tangent substitution, or an odd number of both secant and tangent, which indicates a secant substitution. So we substitute $u = \tan 3x$ in the first integral and $u = \sec 3x$ in the second. To simplify and write the integrands in terms of $u$ only, use the identity $\sec^2 \theta = \tan^2 \theta + 1$.  This gives
\[ \int \answer{\frac{1}{3} u^2} \, du \ \ \text{ and } \ \ \int \answer{\frac{1}{3} u^2 (u^2 - 1)} \, du \]
for the two integrals (don't forget the extra factor of $1/3$ coming from the chain rule).
\item Integrating and reversing the substitution gives
\[ \int \sec^2 3x \tan^2 3x \, dx = \answer{ \frac{(\tan 3x)^3}{9}} + C \]
and
\[ \int \sec^3 3x \tan^3 3x \, dx = \answer{ \frac{(\sec 3x)^5}{15} - \frac{(\sec 3x)^3}{9}} + C. \]
\end{itemize}
\end{example}

\begin{example}
When there are are odd powers of secant and even (or just zero) powers of tangent, there's a trick: integrate by parts letting $dv = \sec^2 x$ and $u$ being everything else. \label{trig:reduce_example}
\[ \begin{aligned}
\int \sec^7 x \, dx & = \int \sec^{\answer{5}} x  \sec^2 x \, dx \\
& = \sec^5 x \answer{\tan x} - \int \answer{5 (\sec x)^4 \sec x (\tan x)^2} dx \\
& = \sec^5 x \answer{\tan x} - 5 \int \sec^7 x \, dx + 5 \int \sec^5 x \, dx.
\end{aligned} \]
Just like for other integration-by-parts examples, we can now \textit{solve} for the integral in terms of a simpler integral:
\[ \int \sec^7 x \, dx = \answer{\frac{1}{6}} \left[ \answer{(\sec x)^5 \tan x} + 5 \int \sec^5 x \, dx \right]. \]
This is an example of a \textit{reduction formula}, because the integral on the right-hand side is similar to the one we started with, but simpler. We could then compute the integral of $\sec^5 x$ in terms of the integral of $\sec^3 x$ and ultimately to $\sec x$, which is one which we can evaluate (or look up in a table).
\end{example}




\end{document}
