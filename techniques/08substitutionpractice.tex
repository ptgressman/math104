\documentclass{ximera}
\graphicspath{
{./}
{volumes/}
{arclengths/}
{centroids/}
{techniques/}
{applications/}
{series/}
{powerseries/}
{odes/}
{lessons/}
}
\usepackage{booktabs}

\newcommand{\bigmath}[1]{$\displaystyle #1$}
\newcommand{\choicebreak}{}
\newenvironment{type}{}{}
\newenvironment{notes}{}{}
\newenvironment{keywords}{}{}
\newcommand{\offline}{}
\newenvironment{comments}{\begin{feedback}}{\end{feedback}}
\newenvironment{multiplechoice}{\begin{multipleChoice}}{\end{multipleChoice}}
\title{Exercises: Substitution and Tables}
\author{Philip T. Gressman}

\begin{document}
\begin{abstract}
Various exercises relating to substitution and the use of integral tables.
\end{abstract}
\maketitle

Compute the indefinite integrals below. Since there are many possible answers (which differ by constant values), use the given instructions if needed to choose which possible answer to use.

\begin{exercise}%[APEX0601SUBS06]
\[ \int (12 x+14) \left(3 x^2+7 x-1\right)^5 dx = \answer{\frac{1}{3}(3 x^2+7 x-1)^6}+C \]
(Add a constant to your answer if needed so that it equals $1/3$ at $x = 0$.)
%
%
\end{exercise}

\begin{exercise}%[APEX0601SUBS11]
\[ \int \frac{e^{\sqrt{x}}}{\sqrt{x}} dx  = \answer{2 e^{\sqrt{x}}}+C \]
(Add a constant to your answer if needed so that it equals $2$ at $x = 0$.)
%
%
\end{exercise}

\begin{exercise}%[APEX0601SUBS09]
\[ \int \frac{x}{\sqrt{x+3}} dx  = \answer{\frac{2}{3} (x-6) \sqrt{x+3}}+C\]
(Add a constant to your answer if needed so that it equals $0$ at $x=6$.)
%
%
\end{exercise}

\begin{exercise}%[APEX0601SUBS33]
\[ \int \frac{\ln |x|}{x} dx  = \answer{\frac{1}{2}\ln^2 |x|}+C\]
Remember absolute value in your logarithm.
(Add a constant to your answer if needed so that it equals $0$ at $x = 1$.)
%
%
\end{exercise}

\begin{exercise}%[APEX0601SUBS57]
\[ \int \sin x \sqrt{\cos x} dx =  \answer{-\frac{2}{3} \cos ^{\frac{3}{2}}(x)}+C\]
(Add a constant to your answer if needed so that it equals $-2/3$ at $x = 0$.)
%
%
\end{exercise}

\begin{exercise}%[APEX0601SUBS63]
\[  \int \frac{9 (2 x+3)}{3 x^2+9 x+7} dx = \answer{3 \ln \left|3 x^2+9 x+7\right|}+C \]
Use absolute values as needed in logarithms. (Add a constant to your answer if needed so that it equals $3 \ln 7$ at $x = 0$.)
%
%
\end{exercise}

\begin{exercise}%[APEX0601SUBS65]
\[ \int \frac{x}{x^4+81} dx = \answer{\frac{1}{18} \arctan \left(\frac{x^2}{9}\right)}+C \]
(Add a constant to your answer if needed so that it equals $0$ at $x = 0$.)
\begin{hint}
Make a substitution $x^2 = 9 u$.
\end{hint}
%
%
\end{exercise}

\begin{exercise}%[APEX0601SUBS83]
Evaluate the definite integral \(\displaystyle \int_{-2}^{-1} (x+1)e^{x^2+2x+1}\ dx. \)
\begin{center}
\begin{prompt}
Value = \(\answer{(1-e)/2}\)
\end{prompt}
\end{center}
%
%
\end{exercise}



\end{document}
