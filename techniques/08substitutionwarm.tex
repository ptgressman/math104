\documentclass{ximera}
\graphicspath{
{./}
{volumes/}
{arclengths/}
{centroids/}
{techniques/}
{applications/}
{series/}
{powerseries/}
{odes/}
}

\newcommand{\bigmath}[1]{$\displaystyle #1$}
\newcommand{\choicebreak}{}
\newenvironment{type}{}{}
\newenvironment{notes}{}{}
\newenvironment{keywords}{}{}
\newcommand{\offline}{}
\newenvironment{comments}{\begin{feedback}}{\end{feedback}}
\newenvironment{multiplechoice}{\begin{multipleChoice}}{\end{multipleChoice}}
\title{Substitution and Tables}
\author{Philip T. Gressman}

\begin{document}
\begin{abstract}
We review substitution and the use of integral tables.
\end{abstract}
\maketitle


\section*{(Videos) Calculus: Single Variable}
\textbf{Note: The section on the Gompertz Model (7:08--11:31) relates to ideas that we will study later but have not yet seen. It is recommended that you do your best to understand it now and then come back to it again when we study ODEs.}
\youtube{q_CmUbctfzo}
\youtube{T-WgpMM813E}

\section*{Online Texts}
\begin{itemize}
\item \href{https://openstax.org/books/calculus-volume-2/pages/1-5-substitution}{OpenStax II 1.5: Substitution} and  \href{https://openstax.org/books/calculus-volume-2/pages/3-5-other-strategies-for-integration}{OpenStax II 3.5: Tables and Computer Systems}
\item \href{https://ximera.osu.edu/mooculus/calculus1/substitution/titlePage}{Ximera OSU: Substitution}
\item \href{https://www.whitman.edu/mathematics/calculus_online/section08.01.html}{Community Calculus 8.1: Substitution}
\end{itemize}

\section*{Examples}

\begin{example}%[washeralgsub2001]
The region in the plane given by \[\left|{\frac{x}{2} - \frac{1}{2 \sqrt{\frac{x^{2}}{3} + 1}}}\right| \leq y \leq \frac{x}{2} + \frac{1}{2 \sqrt{\frac{x^{2}}{3} + 1}}\] and \(0 \leq x \leq 3\) is revolved around the \(x\)-axis. Compute the volume of the resulting solid.

If the variable \(x\) is used for slicing, then slices are perpendicular to the axis of rotation, which indicates the washer method should be used.
The inequalities for \(y\) give the outer and inner radii, and \[ \left(\frac{x}{2} + \frac{1}{2 \sqrt{\frac{x^{2}}{3} + 1}}\right)^2 - \left( \left|{\frac{x}{2} - \frac{1}{2 \sqrt{\frac{x^{2}}{3} + 1}}}\right|\right)^2 = \frac{\sqrt{3} x}{\sqrt{x^{2} + 3}}. \] (Note that the absolute values go away when the radius is squared.)
This leads to the conclusion
\[ V=  \int_{0}^{3} \frac{\sqrt{3} \pi x}{\sqrt{x^{2} + 3}} dx. \]
Among the options below, the best choice for a potential substitution is
\begin{multipleChoice}
\choice{$u = \sqrt{3x}$}
\choice{$u = 1/\sqrt{x^2+3}$}
\choice{$u = \sqrt{x^2 + 3}$}
\choice[correct]{$u = x^2+3$}
\end{multipleChoice}
Using the substitution above, the differentials satisfy \(du = \answer{2x}dx\). This gives the equality
\[  \int \frac{\sqrt{3} \pi x}{\sqrt{x^{2} + 3}}\, dx  = \int \answer{\frac{\sqrt{3} \pi}{2 \sqrt{u}}} du = \answer{\sqrt{3} \pi \sqrt{u}}.  \]
Reversing the substitution gives
\[ \begin{aligned} \int_{0}^{3} \frac{\sqrt{3} \pi x}{\sqrt{x^{2} + 3}}\, dx & = \left. \left[\answer{\sqrt{3} \pi \sqrt{x^{2} + 3}} \right] \right|_{0}^{3}\\ & =  \answer{3 \pi}. \end{aligned} \]
\textbf{Note:} We would not have to reverse the substitution if we also determined the new bounds. In this case, if $u = x^2 + 3$ and $x = 0$, then $u = \answer{3}$. Likewise if $x = 3$, then $u = \answer{12}$. Thus we could also have carried out the calculation by changing bounds:
\[ \int_{0}^{3} \frac{\sqrt{3} \pi x}{\sqrt{x^{2} + 3}}\, dx = \int_{\answer{3}}^{\answer{12}} \answer{\frac{\sqrt{3} \pi}{2 \sqrt{u}}} du. \]
\end{example}


\begin{example}%%From the corresponding worksheet
Using the table of integrals below, compute the indefinite integral
\[ \int \frac{d \theta}{(1-\theta) \sqrt{\theta^2 - 2 \theta - 3}}.\]
\begin{itemize}
\item When dealing with quadratic expressions such as the ones appearing in this integrand, it is often necessary to complete the square before appealing to a table. In this case, 
\[ \theta^2 - 2 \theta - 3 = \left( \theta - \answer{1} \right)^2 - \answer{4}. \]
\item Using the most appropriate entry of the table and plugging in the correct value of $a$ gives
\[ \int \frac{dx}{x \sqrt{x^2 - \answer{4}}} = \frac{1}{\answer{2}} \mathrm{arcsec} \frac{x}{\answer{2}} + C. \]
\item Based on the results of completing the square, we make a substitution $x = \theta - \answer{1}$ and conclude
\[ \int \frac{d \theta}{(1-\theta) \sqrt{\theta^2 - 2 \theta - 3}} =  \int \frac{dx}{\answer{-x} \sqrt{x^2 - \answer{4}}} = \answer{- \frac{1}{2}} \mathrm{arcsec} \answer{\frac{\theta-1}{2}} + C. \]
\end{itemize}
\end{example}


\begin{example}%[2019MT1.integraltable]
Use a table of integrals to compute the antiderivative below.
\[ \int \frac{x^2 ~ dx}{\sqrt{16 + x^6}} \]
\begin{itemize}
\item The key is to make the substitution $u = x^{\answer{3}}$ so that the expression $16 + x^6$ can be understood as a quadratic function of $u$. 
\item Specifically $du = \answer{3x^2} dx$, so
\[ \int \frac{x^2 ~ dx}{\sqrt{16 + x^6}} = \frac{1}{3} \int \frac{du}{\sqrt{16+u^2}}. \] 
Aside from the factor of $1/3$, the $u$ integral belongs to the table:
\[ \int \frac{du}{\sqrt{16+u^2}} = \answer{\ln | u + \sqrt{16 + u^2}| } + C \]
\item We conclude that
\[  \int \frac{x^2 ~ dx}{\sqrt{16 + x^6}} = \frac{1}{3} \int \frac{du}{\sqrt{16+u^2}} = \frac{\ln ( x^3 + \sqrt{16 + x^6})}{3} + C. \]
Absolute values are not needed in the logarithm because $x^3 + \sqrt{16 + x^6}$ can never be negative. 
\end{itemize}
\end{example}


\begin{center}
\textbf{A Basic Table of Integrals} \\ \label{table_of_integrals}
(Note: To use these tables, $k$ and $a$ must represent \textit{constants} in the integral you wish to compute and cannot depend on the variable of integration.)
\begin{tabular}{|lr|}
\hline
\vspace{-10pt}&   \\
$\displaystyle \int x^k dx = \frac{x^{k+1}}{k+1} + C$ & $(k \neq -1)$ \\ $\displaystyle \int \frac{1}{x} dx = \ln |x| + C$ & \\
$\displaystyle \int a^x dx = \frac{a^x}{\ln a} + C$  &  $(a > 0, a \neq 1)$  \\   $\displaystyle \int e^x dx = e^x + C$ & \\
$\displaystyle \int x^k \ln |x| dx = \frac{x^{k+1}}{k+1} \ln |x| - \frac{x^{k+1}}{(k+1)^2} + C$  \hspace{-8pt} & $(k \neq -1)$  \\  $\displaystyle \int \frac{dx}{x^2 + a^2} = \frac{1}{a} \arctan \frac{x}{a} + C$ & $(a \neq 0)$ \\
$\displaystyle \int \frac{dx}{\sqrt{a^2 + x^2}} = \ln \left|x + \sqrt{a^2 + x^2}\right| + C$ & \\ $\displaystyle \int \frac{dx}{\sqrt{a^2 - x^2}} = \arcsin \frac{x}{a} + C$ & $(a > 0, |x| < a)$  \\
$\displaystyle \int \frac{dx}{\sqrt{x^2 - a^2}} = \ln \left| x + \sqrt{x^2-a^2} \right| + C$ & $(|x| > |a|)$ \\ $\displaystyle \int \frac{dx}{x \sqrt{x^2 - a^2}} = \frac{1}{a} \mathrm{arcsec} \frac{x}{a} + C$ & $(a > 0, x > a)$ \\
$\displaystyle \int \sin x ~dx = - \cos x + C$ & \\ $\displaystyle \int \cos x ~dx = \sin x + C$ & \\
$\displaystyle \int \tan x ~dx = \ln |\sec x| + C$ & \\ $\displaystyle \int \csc x ~dx = - \ln |\csc x + \cot x| + C$ \hspace{-18pt} & \\
$\displaystyle \int \sec x ~dx = \ln |\sec x + \tan x| + C$ & \\ $\displaystyle \int \cot x ~dx = \ln |\sin x| + C$ & \\
$\displaystyle \int \sec^2 x ~ dx = \tan x + C$ & \\ $\displaystyle \int \csc^2 x ~dx = - \cot x + C$ & \\
$\displaystyle \int \sec x \tan x dx = \sec x + C$  & \\ $\displaystyle \int \csc x \cot x dx = - \csc x + C$ &  \vspace{-8pt} \\
&   \\
\hline
\end{tabular}
\end{center}




\end{document}
