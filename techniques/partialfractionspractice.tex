\documentclass{ximera}
\graphicspath{
{./}
{volumes/}
{arclengths/}
{centroids/}
{techniques/}
{applications/}
{series/}
{powerseries/}
{odes/}
{lessons/}
}
\usepackage{booktabs}

\newcommand{\bigmath}[1]{$\displaystyle #1$}
\newcommand{\choicebreak}{}
\newenvironment{type}{}{}
\newenvironment{notes}{}{}
\newenvironment{keywords}{}{}
\newcommand{\offline}{}
\newenvironment{comments}{\begin{feedback}}{\end{feedback}}
\newenvironment{multiplechoice}{\begin{multipleChoice}}{\end{multipleChoice}}
\title{Exercises: Partial Fractions}
\author{Philip T. Gressman}

\begin{document}
\begin{abstract}
Various exercises relating to partial fractions and integration.
\end{abstract}
\maketitle

\section*{Sample Quiz Questions}

\begin{question}%%%%%[PartialFracHeav001v2]

Compute the integral
\[\int_{3}^{4}\frac{2x-3}{x^2-3x+2}~dx.\]
\begin{multiplechoice}
\choice{\(\displaystyle \ln 2\)}
\choice[correct]{\(\displaystyle \ln 3\)}
\choice{\(\displaystyle \ln 4\)}
\choice{\(\displaystyle \ln 5\)}
\choice{\(\displaystyle \ln 6\)}
\choice{\(\displaystyle \ln 7\)}
\end{multiplechoice}
\begin{feedback}
First factor the denominator of the integrand: \(x^2-3x+2 = (x-1)(x-2)\). Since the roots are distinct, it is possible to use the Heaviside cover-up method.  The partial fractions expansion will take the form \[\frac{A}{x-1} + \frac{B}{x-2}, \] where the coefficient \(A\) can be computed by cancelling the factor of \(x-1\) in the denominator and evaluating the result at \(x = 1\), i.e., \[A = \frac{2(1)-3}{(1)-2} = 1. \] Similarly, \[B = \frac{2(2)-3}{(2)-1} = 1,\] which gives that \[\frac{2x-3}{(x-1)(x-2)} = \frac{1}{x-1} + \frac{1}{x-2}. \] Therefore
\[ \begin{aligned} 
\int_{3}^{4}\frac{2x-3}{x^2-3x+2}~dx & = \int_{3}^{4}\left(\frac{1}{x-1} + \frac{1}{x-2}\right)~dx \\
 & = \left(\ln |4-1| + \ln |4-2| \right) - \left(\ln |3-1| + \ln |3-2| \right)\\
 & = \ln 3 + \ln 2 + \ln \frac{1}{2} + 0 = \ln 3.\end{aligned}\]
\end{feedback}

\end{question}

\section*{Sample Exam Questions}

\begin{question}%[2019WrittenByClass01]
Compute the volume of the solid of revolution obtained by revolving around the $y$-axis the region below the graph
\[ y = \frac{1}{(x-1)^2}, \]
above $y=0$, and between $x=2$ and $x=3$.
\begin{multipleChoice}
\choice{\(\displaystyle \pi (3 \ln 3 + 1)\)}
\choice{\(\displaystyle \pi (3 \ln 3 + 1)\)}
\choice[correct]{\(\displaystyle \pi (3 \ln 3 + 1)\)}
\choice{\(\displaystyle \pi (3 \ln 3 + 1)\)}
\choice{\(\displaystyle \pi (3 \ln 3 + 1)\)}
\choice{\(\displaystyle \pi (3 \ln 3 + 1)\)}
\end{multipleChoice}

\begin{feedback}
Choosing $x$ as the variable of integration, slices will be parallel to the $y$-axis, indicating that the shell method should be used. The radius of a shell is $x$ (because the axis lies to the left of the region) and the height will be $(x-1)^{-2}$, so 
\[ V = \int_2^3 \frac{2 \pi x}{(x-1)^2} dx = 2 \pi \int_2^3 \frac{x}{(x-1)^2} dx. \]
The integral can be computed by partial fractions; the expansion has the form
\[ \frac{x}{(x-1)^2} = \frac{A}{x-1} + \frac{B}{(x-1)^2}. \]
The coefficients $A$ and $B$ can be found by usual methods (but note that the Heaviside cover up method will \textit{not} work in this case), but it is also possible to find them directly by carefully rewriting the numerator of the fraction in terms of $x-1$:
\[ \frac{x}{(x-1)^2} = \frac{(x-1) + 1}{(x-1)^2} = \frac{(x-1)}{(x-1)^2} + \frac{1}{(x-1)^2} = \frac{1}{x-1} + \frac{1}{(x-1)^2}. \]
Therefore
\[ \begin{aligned} V & = 2 \pi \int_2^3 \left[ \frac{1}{x-1} + \frac{1}{(x-1)^2} \right] dx = 2 \pi \left. \left[ \ln |x-1| - \frac{1}{x-1} \right] \right|_{2}^3 \\
& = 2 \pi \left( \ln 2 - \frac{1}{2} \right) - 2 \pi \left( 0 - 1 \right) = \pi(2 \ln 2 + 1). \end{aligned}\]
\end{feedback}
\end{question}

\begin{question}%%%%%[2016C.04]

Compute the constants \(A\) and \(B\) in the partial fractions expansion indicated below. \offline{To receive full credit, it is not necessary to compute \(C, D,\) or \(E\).}
\[ \frac{x^4 + 16}{x^4 - 16} =A +  \frac{B}{x-2} + \frac{C}{x+2} + \frac{Dx + E}{x^2 + 4} \]
\begin{multiplechoice}
\choice{\(A=-1, B=1\)}
\choice{\(A = 0, B = 1\)}
\choice[correct]{\(A = 1, B = 1\)} 
\choice{\(A=-1, B=-1\)}
\choice{\(A = 0, B = -1\)}
\choice{\(A = 1, B = -1\)}
\end{multiplechoice}
\begin{comments}
\[  \frac{x^4 + 16}{x^4 - 16} = 1 + \frac{1}{x-2} - \frac{1}{x+2} - \frac{4}{x^2+4} \]
\end{comments}


\end{question}

\begin{question}%%%%%[2017C.04]

Evaluate \(\displaystyle \int_1^2 \frac{x^2+x+1}{x^2+x} dx\).
\begin{multiplechoice}
\choice{\(0\)}
\choice{\(1\)}
\choice[correct]{\(\displaystyle 1 + \ln \left(\frac{4}{3}\right)\)}
\choice{\(2\)}
\choice{\(\displaystyle 2 + \ln \left(\frac{8}{3}\right)\)}
\choice{none of these}
\end{multiplechoice}

\end{question}




\end{document}
