\documentclass{ximera}
\graphicspath{
{./}
{volumes/}
{arclengths/}
{centroids/}
{techniques/}
{applications/}
{series/}
{powerseries/}
{odes/}
}

\newcommand{\bigmath}[1]{$\displaystyle #1$}
\newcommand{\choicebreak}{}
\newenvironment{type}{}{}
\newenvironment{notes}{}{}
\newenvironment{keywords}{}{}
\newcommand{\offline}{}
\newenvironment{comments}{\begin{feedback}}{\end{feedback}}
\newenvironment{multiplechoice}{\begin{multipleChoice}}{\end{multipleChoice}}
\title{Exercises: Cumulative}
\author{Philip T. Gressman}

\begin{document}
\begin{abstract}
Exercises relating to various topics we have studied.
\end{abstract}
\maketitle


\section*{Sample Quiz Questions}

\begin{question}%%%%%[shelllogibp001]

The region in the plane between the  \(x\)-axis and the graph
\[ y = \frac{1}{2} \ln{x} + 1 \]
 in the range \(\frac{1}{5} \leq x \leq 1\) is revolved around the axis \(x = \frac{1}{10}\). Compute the volume of the resulting solid.
\begin{multiplechoice}
\choice{\(\displaystyle \frac{11}{25} \pi\)}
\choice{\(\displaystyle \frac{14}{25} \pi\)}
\choice[correct]{\(\displaystyle \frac{16}{25} \pi\)}
\choice{\(\displaystyle \frac{19}{25} \pi\)}
\choice{\(\displaystyle \frac{21}{25} \pi\)}
\choice{\(\displaystyle \frac{24}{25} \pi\)}
\end{multiplechoice}
\begin{feedback}
If the variable \(x\) is used for slicing, then slices are parallel to the axis of rotation, which indicates the shell method should be used.
The radius of a shell is \(x - \frac{1}{10}\). The height of a shell is exactly \(\frac{1}{2} \ln{x} + 1\).
The volume of the region is therefore given by
\[ \int_{\frac{1}{5}}^{1} \frac{\pi}{10} \left(10 x - 1\right) \left(\ln{x} + 2\right)\, dx. \]
 To compute the integralwe can use integration by parts. A reasonable strategy is to integrate  \(x - \frac{1}{10}\) and differentiate  \(\ln{x} + 2\).
 This gives the equality
\[ \begin{aligned} \pi \int \left(x - \frac{1}{10}\right) \left(\ln{x} + 2\right)\, dx & = \pi \left(\left(\frac{x^{2}}{2} - \frac{x}{10}\right) \left(\ln{x} + 2\right) - \int \frac{1}{x} \left(\frac{x^{2}}{2} - \frac{x}{10}\right)\, dx\right) \\
 & = - \pi \left(\frac{x^{2}}{4} - \frac{x}{10}\right) + \pi \left(\frac{x^{2}}{2} - \frac{x}{10}\right) \left(\ln{x} + 2\right). \end{aligned} \]
Therefore 
\[ \begin{aligned} \pi \int_{\frac{1}{5}}^{1} \left(x - \frac{1}{10}\right) \left(\ln{x} + 2\right)\, dx & = \left. \left[- \pi \left(\frac{x^{2}}{4} - \frac{x}{10}\right) + \pi \left(\frac{x^{2}}{2} - \frac{x}{10}\right) \left(\ln{x} + 2\right) \right] \right|_{\frac{1}{5}}^{1}\\ & = \left(\frac{13}{20} \pi \right) - \left(\frac{\pi}{100} \right) = \frac{16}{25} \pi. \end{aligned} \]
\end{feedback}

\end{question}

\begin{question}%%%%%[COMsinibp001]

Consider the region given by \(2 \pi \leq x \leq \frac{5}{2} \pi\) and \(0 \leq y \leq \sin {x}\).
Compute the \(x\)-coordinate of the centroid (i.e., assuming constant density).
\begin{multiplechoice}
\choice{\(\displaystyle -1 + \frac{5}{2} \pi\)}
\choice[correct]{\(\displaystyle 1 + 2 \pi\)}
\choice{\(\displaystyle \frac{5}{2} \pi\)}
\choice{\(\displaystyle -1 + 3 \pi\)}
\choice{\(\displaystyle 3 \pi\)}
\choice{\(\displaystyle 4 \pi\)}
\end{multiplechoice}
\begin{feedback}
The mass  \(M\) will be given by the integral
\[ \int_{2 \pi}^{\frac{5}{2} \pi} \sin {x}\, dx \]
 One can check that
\[ \begin{aligned} \int_{2 \pi}^{\frac{5}{2} \pi} \sin {x}\, dx & = 1. \end{aligned} \]
To compute the  \(x\)-coordinate of the centroid, we also need to compute the integral 
\[ \int_{2 \pi}^{\frac{5}{2} \pi} x \sin {x}\, dx \]
 To compute the integralwe can use integration by parts. A reasonable strategy is to integrate  \(\sin {x}\) and differentiate  \(x\).
 This gives the equality
\[ \begin{aligned} \int x \sin {x}\, dx & = - x \cos {x} - \int \left(- \cos {x}\right)\, dx \\
 & = - x \cos {x} + \sin {x}. \end{aligned} \]
Therefore 
\[ \begin{aligned} \int_{2 \pi}^{\frac{5}{2} \pi} x \sin {x}\, dx & = \left. \left[- x \cos {x} + \sin {x} \right] \right|_{2 \pi}^{\frac{5}{2} \pi}\\ & = 1 - \left(- 2 \pi \right) = 1 + 2 \pi. \end{aligned} \]
The corret answer is the ratio of the integrals, i.e.,
\[ \begin{aligned} \overline{x} & = \frac{1 + 2 \pi}{1} = 1 + 2 \pi. \end{aligned} \]
\end{feedback}

\end{question}

\section*{Sample Exam Questions}

\begin{question}%%%%%[2015C.09]

An object moves in such a way that its acceleration at time \(t\) seconds is \((t^2 + 5t + 6)^{-1}\) meters per second squared. If the initial velocity of the object is \(2/3\) meters per second, what is the limit of its velocity as \(t \rightarrow \infty\)?
\begin{multiplechoice}
\choice{\(\displaystyle \ln \frac{3}{2}\) meters per second}
\choice{\(\displaystyle \ln 6\) meters per second}
\choice{\(1\) meters per second}
\choice{\(\displaystyle \ln \frac{4}{9}\) meters per second}
\choice{\(\displaystyle \ln \frac{9}{4}\) meters per second}
\choice[correct]{\(0\) meters per second}
\end{multiplechoice}

\end{question}

\begin{question}%%%%%[2017C.01]

Find the volume of the solid generated by revolving the region bounded above by \(y = \sin x\) and bounded below by \(y = 0\) for \(0 \leq x \leq \pi\) about the line \(x = \pi\).
\begin{multiplechoice}
\choice{\(\pi^2\)}
\choice[correct]{\(2\pi^2\)}
\choice{\(4\pi^2\)}
\choice{\(\displaystyle\frac{\pi^2}{2}\)}
\choice{\(\displaystyle\frac{\pi^2}{4}\)}
\choice{none of these}
\end{multiplechoice}

\end{question}

\begin{question}%%%%%[2017C.05]

Evaluate \(\displaystyle \int_1^2 x \ln (x^2 + 1) dx\).
\begin{multiplechoice}
\choice{\(0\)}
\choice{\(1\)}
\choice{\(\ln 2\)}
\choice{\(\displaystyle \frac{1}{2}\)}
\choice{\(\displaystyle \ln(2) - \frac{1}{2}\)}
\choice[correct]{none of these}
\end{multiplechoice}
\begin{feedback}
This integral can be computed via integration by parts. If we integrate \(x\) and differentiate \(\ln (x^2+1)\), we get
\[\begin{aligned}
\int_1^2 x \ln (x^2+1) dx & = \left. \frac{x^2}{2} \ln (x^2+1) \right|_{1}^2 - \int_1^2 \frac{x^2}{2} \frac{2x}{x^2+1} dx \\
& = 2 \ln 5 - \frac{1}{2} \ln 2 - \int_1^2 \frac{x^3}{x^2+1} dx.
\end{aligned}\]
The latter integral can be simplified using polynomial long division: \(\displaystyle \frac{x^3}{x^2+1} = x - \frac{x}{x^2+1}\).
Therefore
\[\begin{aligned}
\int_1^2 x \ln (x^2+1) dx & = 2 \ln 5 - \frac{1}{2} \ln 2 - \int_1^2 x dx + \int_1^2 \frac{x}{x^2+1} dx \\
& = 2 \ln 5 - \frac{1}{2} \ln 2 - \left. \frac{x^2}{2} \right|_1^2 + \left. \frac{1}{2} \ln (x^2+1) \right|_1^2 \\
& = 2 \ln 5 - \frac{1}{2} \ln 2 - 2 + \frac{1}{2} + \frac{1}{2} \ln 5 - \frac{1}{2} \ln 2 \\
& = \frac{5}{2} \ln 5 - \frac{2}{2} \ln 4 - \frac{3}{2} - \frac{3}{2} = \ln \left( \frac{5^5}{4} \right) - \frac{3}{2}.
\end{aligned}\]
\end{feedback}

\end{question}


\end{document}
