\documentclass{ximera}
\graphicspath{
{./}
{volumes/}
{arclengths/}
{centroids/}
{techniques/}
{applications/}
{series/}
{powerseries/}
{odes/}
}

\newcommand{\bigmath}[1]{$\displaystyle #1$}
\newcommand{\choicebreak}{}
\newenvironment{type}{}{}
\newenvironment{notes}{}{}
\newenvironment{keywords}{}{}
\newcommand{\offline}{}
\newenvironment{comments}{\begin{feedback}}{\end{feedback}}
\newenvironment{multiplechoice}{\begin{multipleChoice}}{\end{multipleChoice}}
\title{Exercises: Trigonometric Integral}
\author{Philip T. Gressman}

\begin{document}
\begin{abstract}
Various exercises relating to the integration of trigonometric functions.
\end{abstract}
\maketitle


Compute the indefinite integrals below. Since there are many possible answers (which differ by constant values), use the given instructions if needed to choose which possible answer to use.

\begin{exercise}%[APEX0603TRIG05]
\[  \int \sin x\cos^4x\ dx = \answer{-\frac{1}{5} \cos ^5(x)}+C\]
(Your answer should not include any constant term.)
%
%
\end{exercise}

\begin{exercise}%[APEX0603TRIG08]
\[ \int \sin^3 x\cos^3 x\ dx = \answer{\frac{1}{6}\cos^6x-\frac{1}{4}\cos^4x}+C\]
(Your answer should not include any constant terms.)
%
%
\end{exercise}

\begin{exercise}%[APEX0603TRIG19]
\[ \int \tan^4x\sec^2x\ dx = \answer{\frac{\tan ^5(x)}{5}}+ C\]
(Add a constant to your answer if needed so that it equals $0$ at $x = 0$.)
%
%
\end{exercise}

\begin{exercise}%[APEX0603TRIG23]
\[ \int \tan^3x\sec^3x\ dx = \answer{\frac{\sec ^5(x)}{5}-\frac{\sec ^3(x)}{3}}+ C\]
(Add a constant to your answer if needed so that it equals $-2/15$ at $x = 0$.)
%
%
\end{exercise}

\begin{exercise}%[APEX0603TRIG10]
\[ \int \sin^2 x\cos^7 x\ dx = \answer{-\frac{1}{9} \sin ^{9}(x)+\frac{3 \sin^7(x)}{7}-\frac{3\sin ^5(x)}{5}+\frac{\sin ^3(x)}{3}}+ C\]
(Your answer should not include any constant terms.)
\begin{hint}
\[ (1-u^2)^3 = 1 - 3 u^2 + 3 u^4 - u^6. \]
\end{hint}
%
%
\end{exercise}

\begin{exercise}%[APEX0603TRIG13]
\[ \int \sin(5x)\cos(3x)\ dx = \answer{\frac{1}{2}\left(-\frac{1}{8}\cos(8x)-\frac{1}{2}\cos(2x)\right)}+ C\]
(Your answer should not include any constant terms.)
%
%
\end{exercise}

\begin{exercise}%[APEX0603TRIG11]
\[ \int \sin^2 x\cos^2 x\ dx = \answer{\frac{x}{8}-\frac{1}{32} \sin (4 x)}+ C \]
(Your answer should not include any constant terms and should equal $0$ at $x = 0$.)
\begin{hint}
Use power reduction formulas.
\end{hint}
%
%
\end{exercise}


\section*{Sample Quiz Questions}

\begin{question}%%%%%[TrigInt001]

Compute the value of the integral
\[\int_{0}^{\frac{\pi}{4}} \sin^{3} 2x ~dx.\]
(Hints won't be revealed until after you choose a response.)
\begin{multiplechoice}
\choice{\(\displaystyle \frac{1}{5}\)}
\choice[correct]{\(\displaystyle \frac{1}{3}\)}
\choice{\(\displaystyle \frac{1}{2}\)}
\choice{\(\displaystyle 1\)}
\choice{\(\displaystyle 2\)}
\choice{\(\displaystyle 3\)}
\end{multiplechoice}
\begin{feedback}
To simplify the calculation, begin with a substitution which replaces \(x\) with \(x/2\). The question reduces to computing
\[\frac{1}{2}\int_{0}^{\frac{\pi}{2}} \sin^{3} x ~dx.\]
This integral is compatible with the substitution \(u = \cos x\). \begin{hint} By the substitution formula, this means \(dx = - du / \sin x\), and one must also replace \(\sin^2 x\) by \(1 - u^2\). Furthermore, by virtue of the special angle formulas \(\cos 0 = 1\) and \(\cos \frac{\pi}{2} = 0\), the problem is reduced to computing the integral
\[-\frac{1}{2}\int_{1}^{0} (1 - u^2) ~du.\]
\begin{hint}
Carrying out this calculation in the usual way gives a final answer of \(\frac{1}{3}\).
\end{hint}
\end{hint}
\end{feedback}

\end{question}

\begin{question}%%%%%[TrigInt010]

Compute the value of the integral
\[\int_{\frac{\pi}{6}}^{\frac{\pi}{2}} \tan^{-6} x  \sec^{5} x ~dx.\]
(Hints won't be revealed until after you choose a response.)
\begin{multiplechoice}
\choice{\(\displaystyle \frac{17}{5}\)}
\choice{\(\displaystyle \frac{19}{5}\)}
\choice{\(\displaystyle \frac{23}{5}\)}
\choice{\(\displaystyle \frac{29}{5}\)}
\choice[correct]{\(\displaystyle \frac{31}{5}\)}
\choice{\(\displaystyle \frac{37}{5}\)}
\end{multiplechoice}
\begin{feedback}
Since the power of secant is odd and the power of tangent is even, try rewriting the integral in terms of sine and cosine. This gives 
\[\int_{\frac{\pi}{6}}^{\frac{\pi}{2}} \sin^{-6} x  \cos x ~dx.\]
This integral is compatible with the substitution \(u = \sin x\). \begin{hint} By the substitution formula, this means \(dx = du / \cos x\). Furthermore, by virtue of the special angle formulas \(\sin \frac{\pi}{6} = \frac{1}{2}\) and \(\sin \frac{\pi}{2} = 1\), the problem is reduced to computing the integral
\[\int_{\frac{1}{2}}^{1} u^{-6} ~du.\] \begin{hint}
Carrying out this calculation in the usual way gives a final answer of \(\frac{31}{5}\). \end{hint} \end{hint}
\end{feedback}

\end{question}

\begin{question}%%%%%[TrigInt020]

Compute the value of the integral
\[\int_{\frac{\pi}{6}}^{\frac{\pi}{2}} \sin^{-2} x  \cos^{3} x ~dx.\]
(Hints won't be revealed until after you choose a response.)
\begin{multiplechoice}
\choice{\(\displaystyle \frac{1}{5}\)}
\choice{\(\displaystyle \frac{1}{3}\)}
\choice[correct]{\(\displaystyle \frac{1}{2}\)}
\choice{\(\displaystyle 1\)}
\choice{\(\displaystyle 2\)}
\choice{\(\displaystyle 3\)}
\end{multiplechoice}
\begin{feedback}
This integral is compatible with the substitution \(u = \sin x\). By the substitution formula, this means \(dx = du / \cos x\), and one must also replace \(\cos^2 x\) by \(1-u^2\). Furthermore, by virtue of the special angle formulas \(\sin \frac{\pi}{6} = \frac{1}{2}\) and \(\sin \frac{\pi}{2} = 1\), the problem is reduced to computing the integral \begin{hint}
\[\int_{\frac{1}{2}}^{1} u^{-2}(1-u^2) ~du.\] \begin{hint}
Carrying out this calculation in the usual way gives a final answer of \(\frac{1}{2}\). \end{hint} \end{hint}
\end{feedback}

\end{question}

\section*{Sample Exam Questions}

\begin{question}%%%%%[2015C.10]

Compute the integral below.
\[ \int_0^{\frac{\pi}{8}} \tan^4 2x \sec^4 2x ~ dx \]
\begin{multiplechoice}
\choice{\(\displaystyle \frac{4}{9}\)}
\choice{\(\displaystyle \frac{7}{24}\)}
\choice{\(\displaystyle \frac{5}{14}\)}
\choice{\(\displaystyle \frac{9}{28}\)}
\choice[correct]{\(\displaystyle \frac{6}{35}\)}
\choice{\(\displaystyle \frac{1}{7}\)}
\end{multiplechoice}

\end{question}


\end{document}
