\documentclass{ximera}

\title{Welcome: About This Course}
%%%%%\author{Philip T. Gressman}
 
\begin{document}
\begin{abstract}
  This course is built in Ximera.
\end{abstract}\maketitle

The philosophy behind Ximera is to make it possible to learn mathematics \textit{actively} online. \textbf{Wrong answers never hurt you.} You should go through these pages with paper and pencil in hand; take notes, draw pictures, do calculations. A number of exercises have hints: you are strongly encouraged (and sometimes structurally required) to make your best effort to answer an exercise before referring to the hints.  Remember: \textbf{these exercises are for you to gain insight and strength by working through carefully} and are not simple boxes to check off or hoops to jump through.

\section*{The Mechanics of Ximera}

See these links for information and examples of how this system is used:
 \begin{itemize}
 \item \link[How to Use Ximera]{https://ximera.osu.edu/mooculus/calculus1/ximeraTutorial/howToUseXimera}
 \item \link[How Work is Scored in Ximera]{https://ximera.osu.edu/mooculus/calculus1/ximeraTutorial/howIsMyWorkScored}
 \end{itemize}
  
  
 \section*{The Structure of This Course}
 
 Each topic in this course is addressed in a two-part way. You will first encounter a page with links to videos and online texts that should be the first place you go to learn new material. \textbf{There is a wealth of high-quality, open-source material on the web, and it is not the purpose of this course to reinvent the wheel.} The main sources that you are encouraged to use are
 
 \begin{itemize}
 \item \link[Calculus: Single Variable]{https://www.youtube.com/playlist?list=PLKc2XOQp0dMwj9zAXD5LlWpriIXIrGaNb}, a series of short YouTube videos by Penn professor Robert Ghrist
 \item \link[OpenStax Calculus II]{https://openstax.org/details/books/calculus-volume-2}, a web-based, open-source, modern Calculus textbook. It is approved by the Open Textbook Initiative of the American Institute of Mathematics.
 \end{itemize}
 
 In most cases, these materials should be sufficient for you to actively engage with the material we will study. \textbf{For every topic, it is expected that you will view videos and read textbook sections on your own \textit{before we discuss these topics in class}}. You should further work through all the included ``Examples'' on a topic before class. In particular, these same examples will often appear on pre-class Canvas quizzes.
 
After each set of examples, there is a list of exercises which are a mix of original content, questions drawn from open-source texts: OpenStax Calculus II, \link[Calculus: Early Transcendentals]{https://www.whitman.edu/mathematics/calculus_online/}, and \link[APEX Calculus]{http://www.apexcalculus.com/}; and old Penn calculus exam questions.  These exercises are generally more challenging than the initial examples and may take more time and some thought inside the classroom as well as outside in order to complete. The exercises sections will have sample questions from quizzes and old exams when relevant. It is highly recommended that you carefully work such questions and avoid guessing answers, as they are there for you primarily to test your own understanding of how to solve problems and not simply there for you to know their answers.

 
\end{document}